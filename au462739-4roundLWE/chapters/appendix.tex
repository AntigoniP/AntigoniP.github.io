% !TEX root =../main-optimal.tex

% !TEX root =../main-optimal.tex

% !TEX root =../main-optimal.tex		--- NOT TRUE ANYMORE
% % !TEX root =./appendix.tex

\section{Secure Computation Definitions}
\label{sec:mpc}

For completeness, we recall the definition of secure computation based on \cite[Chapter 7]{Goldreich04} here. We only recall the two party case as it is most relevant to our proofs. The description naturally extends to multi-party case as well (details can be found in \cite{Goldreich04}).


\paragraph{Two-party computation.} A two-party protocol problem is cast by specifying a random process
that maps pairs of inputs to pairs of outputs (one for each party). We refer to such a process as
a functionality and denote it $\mathbf{\func} : \bit^* \times \bit^*\rightarrow\bit^* \times \bit^*$ where $\mathbf{\func} = (F_1, F_2)$. That is,
for every pair of inputs $(x, y)$, the output-pair is a random variable $(F_1(x, y), F_2(x, y))$ ranging over
pairs of strings. The first party (with input $x$) wishes to obtain $F_1(x, y)$ and the second party (with
input $y$) wishes to obtain $F_2(x, y)$.


\paragraph{Adversarial behavior.} Loosely speaking, the aim of a secure two-party protocol is to protect
an honest party against dishonest behavior by the other party. In this paper, we consider malicious
adversaries who may arbitrarily deviate from the specified protocol. When considering malicious
adversaries, there are certain undesirable actions that cannot be prevented. Specifically, a party
may refuse to participate in the protocol, may substitute its local input (and use instead a different
input) and may abort the protocol prematurely. One ramification of the adversary's ability to
abort, is that it is impossible to achieve {\em fairness}. That is, the adversary may obtain its output
while the honest party does not. In this work we consider a static corruption model, where one of the parties is adversarial and the other is honest, and this is fixed before the execution begins.


\paragraph{Communication channel.} In our results we consider a secure simultaneous message exchange channel in which all parties can simultaneously send messages over the channel at the same communication round. Moreover, we assume an asynchronous network\footnote{The fact that the network is asynchronous means that the messages are not necessarily delivered in the order which
they are sent.} where the communication is open (i.e. all the communication between the parties is seen by the adversary) and delivery of messages is not guaranteed. For simplicity, we assume that the delivered messages are authenticated. This can be achieved using standard methods.



\paragraph{Security of protocols (informal).} The security of a protocol is analyzed by comparing what an
adversary can do in the protocol to what it can do in an ideal scenario that is secure by definition.
This is formalized by considering an ideal computation involving an incorruptible trusted third
party to whom the parties send their inputs. The trusted party computes the functionality on the
inputs and returns to each party its respective output. Loosely speaking, a protocol is secure if
any adversary interacting in the real protocol (where no trusted third party exists) can do no more
harm than if it was involved in the above-described ideal computation.



\paragraph{Execution in the ideal model. } As we have mentioned, some malicious behavior cannot be
prevented (for example, early aborting). This behavior is therefore incorporated into the ideal
model. An ideal execution proceeds as follows:
\BDE
\item {\bf Inputs:} Each party obtains an input, denoted $w$ ($w = x$ for $\sen$, and $w = y$ for $\rec$).
\item  {\bf Send inputs to trusted party:} An honest party always sends $w$ to the trusted party. A malicious
party may, depending on $w$, either abort or send some $w' \in\bit^{|w|}$ to the trusted party.
\item  {\bf Trusted party answers first party:} In case it has obtained an input pair $(x, y)$, the trusted
party first replies to the first party with $F_1(x, y)$. Otherwise (i.e., in case it receives only one
valid input), the trusted party replies to both parties with a special symbol $\bot$.
\item  {\bf Trusted party answers second party: } In case the first party is malicious it may, depending on
its input and the trusted party's answer, decide to stop the trusted party by sending it $\bot$ after
receiving its output. In this case the trusted party sends $\bot$ to the second party. Otherwise
(i.e., if not stopped), the trusted party sends $F_2(x, y)$ to the second party.
\item  {\bf Outputs:} An honest party always outputs the message it has obtained from the trusted party. A
malicious party may output an arbitrary (probabilistic polynomial-time computable) function
of its initial input and the message obtained from the trusted party.
\EDE
Let $\mathbf{\func} : \bit^* \times \bit^*\rightarrow\bit^* \times \bit^*$ be a functionality where $\mathbf{\func} = (F_1, F_2)$ and let $\cS =
(\cS_1, \cS_2)$ be a pair of non-uniform probabilistic expected polynomial-time machines (representing
parties in the ideal model). Such a pair is {\em admissible} if for at least one $i\in \{1, 2\}$ we have that $\cS_i$
is honest (i.e., follows the honest party instructions in the above-described ideal execution). Then,
the {\em joint execution of $F$ under $\cS$ in the  ideal model} (on input pair $(x, y)$ and security parameter $\kappa$), denoted $\ideall_{\mathbf{\func},\cS}(\kappa,x,y)$
is defined as the output pair of $\cS_1$ and $\cS_2$ from the above ideal execution.


\paragraph{Execution in the real model.} We next consider the real model in which a real (two-party)
protocol is executed (and there exists no trusted third party). In this case, a malicious party
may follow an arbitrary feasible strategy; that is, any strategy implementable by non-uniform
probabilistic polynomial-time machines. In particular, the malicious party may abort the execution
at any point in time (and when this happens prematurely, the other party is left with no output).
Let $\mathbf{\func}$ be as above and let $\Pi$ be a two-party protocol for computing $\mathbf{\func}$. Furthermore, let $\cA =
(\cA_1, \cA_2)$ be a pair of non-uniform probabilistic polynomial-time machines (representing parties in
the real model). Such a pair is {\em admissible} if for at least one $i \in \{1, 2\}$ we have that $\cA_i$
is honest (i.e., follows the strategy specified by $\Pi$). Then, the {\em joint execution of $\Pi$ under $\cA$ in the real model}, denoted $\reall_{\Pi,\cA}(\kappa,x,y)$, is defined as the output pair of $\cA_1$ and $\cA_2$ resulting
from the protocol interaction.

\paragraph{Security as emulation of a real execution in the ideal model.} Having defined the ideal
and real models, we can now define security of protocols. Loosely speaking, the definition asserts
that a secure two-party protocol (in the real model) emulates the ideal model (in which a trusted
party exists). This is formulated by saying that admissible pairs in the ideal model are able to
simulate admissible pairs in an execution of a secure real-model protocol.



\BD[secure two-party computation] Let $\mathbf{\func}$ and $\Pi$ be as above. Protocol $\Pi$ is said to
securely compute $\mathbf{\func}$ (in the malicious model) if for every pair of admissible non-uniform probabilistic
polynomial-time machines $\cA =
(\cA_1, \cA_2)$ for the real model, there exists a pair of admissible non-uniform
probabilistic expected polynomial-time machines $\cS =(\cS_1, \cS_2)$  for the ideal model, such that:

$$
\left\{\ideall_{\mathbf{\func},\cS}(\kappa,x,y)\right\}_{\kappa\in\NN,x,y ~{\text  s.t. } ~|x|=|y|} \indist
\left\{\reall_{\Pi,\cA}(\kappa,x,y)\right\}_{\kappa\in\NN,x,y ~{\text  s.t. } ~|x|=|y|}
$$

%Namely, the two distributions are computationally indistinguishable.
\ED

We note that the above definition assumes that the parties know the input lengths (this can be
seen from the requirement that $|x| = |y|$). Some restriction on the input lengths is unavoidable,
see \cite[Section 7.1]{Goldreich04} for discussion. We also note that we allow the ideal adversary/simulator to
run in expected (rather than strict) polynomial-time. This is essential for constant-round
protocols \cite{BL04}.
%We denote the security parameter by $\kappa$ and, for the sake of simplicity, unify it with the length
%of the inputs (thus we consider security for ``all sufficiently long inputs''). 

%The above naturally extends to the multi-party computation setting. 
\remove{\section{General Definitions}
\subsection{Witness Relations}

We recall the definition of a witness relation for an NP language \cite{Goldbook,STOC:LinPas11}.
\BD[Witness relation] A \emph{witness relation} for a language $L \in \NP$ is a binary relation
$R_L$ that is polynomially bounded, polynomial time recognizable and characterizes $L$ by $L = \{x: ~\exists~ y ~\text{s.t.} ~(x,y)\in R_L\}$
\ED
We say that $y$ is a witness for the membership $x \in L$ if  $(x,y)\in R_L$. We will also let $R_L(x)$
denote the set of witnesses for the membership $x \in L$, i.e., $R_L(x) = \{y:(x,y)\in L \}$. In the
following, we assume a fixed witness relation $R_L$ for each language $L \in \NP$.

\subsection{Interactive Proofs}
%We use the standard definitions of interactive proofs (and interactive Turing machines) [GMR89]
%and arguments (a.k.a. computationally-sound proofs) [BCC88]. 

Given a pair of interactive Turing
machines, $P$ and $V$ , we denote by $\langle P(w), V \rangle(x)$ the random variable representing the (local) output
of $V$, on common input $x$, when interacting with machine $P$ with private input $w$, when the random
input to each machine is uniformly and independently chosen.

\BD [Interactive Proof System] A pair of interactive machines $\langle P, V \rangle$ is called an \emph{interactive proof system} for a language $L$ if there is a negligible function $\mu(\cdot)$ such that the following two conditions hold:
\BI
\item Completeness: For every $x \in L$, and every $w\in R_L(x)$, $Pr [\langle P(w), V \rangle(x)= 1] = 1$. 
\item  Soundness: For every $x \notin L$, and every interactive machine $P^*, Pr [\langle P^*, V \rangle(x) = 1] \leq\mu(\kappa)$
\EI
In case the soundness condition is required to hold only with respect to a computationally
bounded prover, the pair $\langle P, V \rangle$ is called an interactive argument system.
\ED

\subsection{Zero-Knowledge}
We recall the standard definition of ZK proofs. Loosely speaking, an interactive proof is said to be
zero-knowledge (ZK) if a verifier $V$ learns nothing beyond the validity of the assertion being proved,
it could not have generated on its own. As ``feasible''computation in general is defined though
the notion of probabilistic polynomial-time, this notion is formalized by requiring that the output
of every (possibly malicious) verifier interacting with the honest prover $P$ can be ``simulated" by
a probabilistic expected polynomial-time machine $\cS$ (a.k.a. the simulator). The idea behind this
definition is that whatever $V^*$ might have learned from interacting with $P$, he could have learned
by himself by running the simulator $\cS$. 

\BD [ZK]. Let $L$ be a language in $\NP$, $R_L$ a witness relation for $L$, $(P,V )$ an interactive
proof (argument) system for $L$. We say that $(P, V )$ is \emph{statistical/computational ZK}, if for every
probabilistic polynomial-time interactive machine $V$ there exists a probabilistic algorithm $\cS$ whose
expected running-time is polynomial in the length of its first input, such that the following ensembles
are statistically close/computationally indistinguishable over $L$.
\BI
\item $\{\langle P(y), V(z) \rangle(x)\}_{\kappa\in \NN\,x\in \bit^\kappa\cap L, y \in R_L(x), z\in\bit^*}$
\item $\{\cS (x,z)\}_{\kappa\in \NN\,x\in \bit^\kappa\cap L, y \in R_L(x), z\in\bit^*}$
\EI

where $\langle P(y), V(z) \rangle(x)$ denotes the view of $V$ in interaction with $P$ on common input $x$ and private
inputs $y$ and $z$, respectively.

\ED

\subsection{Witness Indistinguishability}
An interactive proof (or argument) is said to be witness indistinguishable (WI) if the verifier's
output is ``computationally'' independent of the witness used by the prover for proving the statement.
In this context, we focus on languages $L \in \NP$ with a corresponding witness relation $R_L$.
Namely, we consider interactions in which, on common input $x$, the prover is given a witness in
$R_L(x)$. By saying that the output is computationally independent of the witness, we mean that
for any two possible $\NP$-witnesses that could be used by the prover to prove the statement $x \in L$,
the corresponding outputs are computationally indistinguishable.

\BD [Witness-indistinguishability]. Let $\langle P, V \rangle$ be an interactive proof (or argument) sys-
tem for a language $L \in \NP$. We say that $\langle P, V \rangle$ is \emph{witness-indistinguishable} for $R_L$, if for every
probabilistic polynomial-time interactive machine $V^*$ and for every two sequences $\{w^1_{\kappa,x}\}_{\kappa\in \NN,x\in L}$ and $\{w^2_{\kappa,x}\}_{\kappa\in \NN,x\in L}$, such that $w^1_{\kappa,x}, w^2_{\kappa,x}\in R_L(x)$ for every $x \in L\cap \bit^\kappa$, the following probability
ensembles are computationally indistinguishable over $\kappa\in \NN$.
\ED

\BI
\item $\{\langle P(w^1_{\kappa,x}), V^*(z) \rangle(x)\}_{\kappa\in \NN\,x\in \bit^\kappa\cap L, z\in\bit^*}$
\item $\{\langle P(w^2_{\kappa,x}), V^*(z) \rangle(x)\}_{\kappa\in \NN\,x\in \bit^\kappa\cap L, z\in\bit^*}$
\EI
\subsection{Proofs (Arguments) of Knowledge}
Loosely speaking, an interactive proof is a proof of knowledge if the prover convinces the verifier
that it possesses, or can feasibly compute, a witness for the statement proved. The notion of a
proof of knowledge is essentially formalized as follows: an interactive proof of $x \in L$ is a proof of
knowledge if there exists a probabilistic expected polynomial-time extractor machine $E$, such that
for any prover $P$, $E$ on input the description of $P$ and any statement $x \in L$ readily outputs a valid
witness for $x \in L$ if $P$ succeeds in convincing the Verifier that $x \in L$. Formally,

\BD[Proof of knowledge] Let $(P, V )$ be an interactive proof system for the language
$L$. We say that $(P, V )$ is a proof of knowledge for the witness relation $R_L$ for the language $L$ it there
exists an probabilistic expected polynomial-time machine $E$, called the extractor, and a negligible
function $\mu(\cdot)$ such that for every machine $P^*$, every statement $x \in \bit^\kappa$, every random tape
$x \in \bit^*$, and every auxiliary input $z \in \bit^*$,
$$Pr [ \langle P^*_r(z), V \rangle(x) = 1]\leq Pr[E^{P^*_r(x,z)}(x) \in R_L(x)] + \mu(\kappa)$$
\ED

An interactive argument system $\langle P, V \rangle$ is an argument of knowledge if
the above condition holds w.r.t. probabilistic polynomial-time provers.

\paragraph{Special-sound WI proofs.} A $4$-round public-coin interactive proof for the language $L \in \NP$ with
witness relation $R_L$ is special-sound with respect to $R_L$, if for any two transcripts $(\delta,\alpha,\beta, \gamma)$ and $(\delta',\alpha',\beta', \gamma')$ such that the initial two messages, $(\delta,\delta')$ and $(\alpha,\alpha')$ are the same but the challenges $(\beta,\beta')$ are different, there is a deterministic procedure to extract the witness from the two transcripts
and runs in polynomial time. Special-sound WI proofs for languages in $\NP$ can be based on
the existence of $2$-round commitment schemes, which in turn can be based on one-way functions.
%[GMW91, FS90, HILL99, Nao91].
\paragraph{Adaptive-Input Witness Indistinguishability.}
The most recent $3$-round adaptive-Input Witness Indistinguishability we can use appears in the work of \cite{COSV16}. 
The notion of adaptive-input WI formalizes security of the prover with respect to an adversarial
verifier $A$ that adaptively chooses the input instance to the protocol, after seeing the first message of the prover. More specifically, for a delayed-input $3$-round complete protocol , we
consider game ${\sf ExpAWI}$ between a challenger $C$ and an adversary $A$ in which the instance $x$ and
two witnesses $w_0$ and $w_1$ for $x$ are chosen by $A$ after seeing the first message of the protocol played
by the challenger. The challenger then continues the game by randomly selecting one of the two
witnesses and by computing the third message by running the prover's algorithm on input the
instance $x$, the selected witness $w_b$ and the challenge received from the adversary. The adversary
wins the game if she can guess which of the two witnesses was used by the challenger.
The experiment is parameterized by a delayed-input $3$-round complete protocol  $(P, V)$ for a relation $R$ and by a $\ppt$
adversary $A$. The experiment has as input the security parameter and auxiliary information for $A$. 
The experiment ${\sf ExpAWI}$ proceeds as follows: 
\BE
\item $C$ randomly selects coin tosses $r$ and runs P on input $(1^\kappa; r)$ to obtain $a$;
\item $A$, on input $a$ and $aux$, outputs instance $x$, witnesses $w_0$ and $w_1$ such that
$(x,w_0), (x,w_1) \in R$, challenge $c$ and internal state $\sf state$;
\item $C$ randomly selects $b\leftarrow \bit$ and runs $P$ on input $(x,w_b, c)$ to obtain $z$;
\item $b'  \leftarrow A((a, c, z), aux, \sf state)$;
\item if $b = b'$ then output $1$ else output $0$.
\EE
 
\BD[Adaptive-Input Witness Indistinguishability]. A delayed input $3$-round complete
protocol is adaptive-input WI if for any $\ppt$ adversary $A$ there exists a negligible function $\mu(\cdot)$ such that for any $aux \in\bit^*$ it holds that ${\sf AdvAWI}_A(\kappa, aux)\leq \mu(\kappa)$ where ${\sf AdvAWI}_A(\kappa, aux) = |Pr[{\sf AdvAWI}_A(\kappa, aux)=1]-1/2|$
\ED}

%% !TEX root =./main-optimal.tex
\section{Proof Systems}
\label{sec:proofsystems}

We provide a detailed description of the proof systems used in this work.

\subsection{Protocol $\Pi_\WIPOK$} 
This is essentially the Feige-Lapidot-Shamir protocol, slightly reworded in \cite{KatzO04}, mostly for notational convenience.  We recall this protocol here. We denote the messages of this protocol by $(\WIPOKmsg_1, \WIPOKmsg_2,\WIPOKmsg_3)$.

We will be working with the NP-complete language $\it HC$ of graph Hamiltonicity,
and thus assume statements to be proven take the form of graphs,
while witnesses correspond to Hamilton cycles. If $\thmm$ is a graph, we abuse notation
and also let $\thmm$ denote the statement ``$\thmm \in {\it HC}$''. We show how the
proof system can be used to prove the following statement: $\thmm \land \thmm'$, where
$\thmm$ will be included as part of the first message, while $\thmm'$
is only decided in the last round. The
proof system $\Pi_\WIPOK$ runs $\kappa$ parallel executions of the following $3$-round protocol:

\BE

\item The prover commits to two adjacency matrices for two randomly-chosen cycle
graphs $G,G'$. The commitment is done bit-by-bit using a perfectly-binding
commitment scheme.
\item The verifier responds with a single bit $b$, chosen at random.
\item If $b = 0$, the prover opens all commitments. If $b = 1$, the prover sends two
permutations mapping the cycle in $\thmm$ (resp., $\thmm'$) to $G$ (resp., $G'$). For
each non-edge in $\thmm$ (resp., $\thmm'$), the prover opens the commitment at the
corresponding position in $G$ (resp., $G'$).
\item[] The verifier checks that all commitments were opened correctly. If $b = 0$, the
verifier additionally checks whether both decommitted graphs are indeed
cycle graphs. If $b = 1$, the verifier checks whether each non-edge in $\thmm$
(resp., $\thmm'$) corresponds to a non-edge in $G$ (resp., $G'$).
\EE

Note that the prover does not need to know either $\thmm$ or $\thmm'$ (or the corresponding
witnesses) until the beginning of the third round. In the above proof system, we assume that 
$\thmm$ is fixed as part of the first-round message enabling us to
claim stronger properties about the proof system. In particular, $\Pi_\WIPOK$ proof system is complete and sound. More specifically, the probability that an all-powerful
prover can cause a verifier to accept when either $\thmm$ or $\thmm'$
are not true
is at most $2^{-\kappa}$. We stress that this holds even if the prover can adaptively
choose  $\thmm'$ after viewing the second-round message of the verifier. Moreover, $\Pi_\WIPOK$ is witness indistinguishable and it is a proof of knowledge for $\thmm$. (More formally, we can achieve a notion
similar to that of ``witness-extended emulation'' \cite{C:Lindell01} for $\thmm$.) Note also that the first round of the above proof system (as well as the
internal state of the prover immediately following this round) is independent of
$\thmm$ or the associated witness.


\subsection{Protocol $\Pi_\FLS$}\label{sec:fs}

As noted in Section \ref{sec:prelim}, this is essentially the four round zero-knowledge protocol of Feige-Shamir, except that we use {\em non-malleable commitments} in the first three round of the protocol. Following the discussion in Section \ref{sec:prelim}, we let \nmcom\ be a non-malleable commitment scheme, and make the simplifying assumption that \nmcom\ has just three rounds and the first round is committing. Again, these are purely for notational convenience and can easily be removed (as discussed earlier).

We now simply list all the steps of this protocol following \cite{KatzO04}, but using \nmcom. The messages of this protocol are denoted by 
$(\FLSmsg_1, \FLSmsg_2,\FLSmsg_3, \FLSmsg_4)$. It allows the prover to prove $\thmm \land \thmm'$ where  $\thmm$ is sent as part of the second round yet $\thmm'$ is only sent as part of the last round. (Intuitively, statements $\thmm,\thmm'$ will correspond to statements $\st_1,\st_2$ of $\Pi_\WIPOK$ described above.)

The proof system $\Pi_\FLS$ proceeds as follows: 



\BE

\item The first round is as in the original Feige-Shamir protocol but augmented with an $\nmCom$ scheme.  Explicitly, the verifier $V$ selects randomly and independently two values $\sigma_1$ and $\sigma_2$ and computes the first message of two independent executions of $\nmCom$ for $\sigma_1$ and $\sigma_2$, with randomness $\rho_1,\rho_2$ respectively. Let $\nm_1^{\sigma_1}$ and $\nm_2^{\sigma_2}$ be these messages, which $V$ sends to $P$. %corresponding to $\sigma_1$ and $\sigma_2$ respectively. %The transcripts of these executions up to the $\rnm-2$ round are denoted by $\nmCom(\sigma_{1})$ and $\nmCom(\sigma_{2})$, respectively. 

Moreover, $V$ sends the first message $\WIPOKmsg_{1}$ of a WIPOK proof system.

\item The prover $P$ chooses a random challenge $R \in \bit^{2\kappa}$ and computes ${\sf C_R}=\qCom(R;\zeta)$. Let $\ok$ denote the statement that $\qCom$ was formed correctly.


\item[] Let $\tthm$ denote the statement: $(\thmm\land \ok)\lor(\nm_1^{\sigma_1}=\nmCom_1(\sigma_{1};\rho_1))\lor(\nm_1^{\sigma_2}=\nmCom_1(\sigma_{2};\rho_2))$ (this statement is reduced to a single graph \tthm). Then, $P$ sends ${\sf C_R}$ and also
the first message $\tilde\WIPOKmsg_1$ of a separate WIPOK proof system and message $\WIPOKmsg_2$ of $V$'s proof.

\item $V$ sends the last message $\WIPOKmsg_3$ of his WIPOK proof system and completes the proof for the knowledge of the values in $\nmCom$ (which is also completed along with the first and second rounds \footnote{If $\rnm>3$ then $V$ completes its WIPOK after the completion of $\nmCom$.}). $V$ additionally sends a random $R' \in \bit^{2\kappa}$ and message $\tilde\WIPOKmsg_2$ of $P$'s proof
\item $P$ decommits to $R$. Let $\prg$ be the statement that
$r = R \oplus R'$ is pseudorandom (i.e., $\exists s ~{\text s.t.} ~{\sf PRG}(s) = r$, where $\sf PRG$ is a pseudorandom function). Let $\tthm'$
be the statement $\thmm' \lor \prg$ (reduced to a single graph $\tthm'$). The prover send the last message $\tilde\WIPOKmsg_3$ of the $\Pi_\WIPOK$ proof system and completes the proof for the statement $\tthm \land \tthm'$.
\item[] $V$ checks the decommitment of $R$, and verifies the proof.
\EE

As claimed in \cite{KatzO04} $\Pi_\FLS$  proof system satisfies the following properties. It is complete and sound (for a poly-time prover) for $\thmm$ and $\thmm'$. Rounds $2-4$ constitute a proof of knowledge for $\tthm$. If a poly-time prover
can cause a verifier to accept with ``high'' probability, then a witness for
$\thmm\land\ok$ can be extracted with essentially the same probability. If $\ok$ is true,
then with all but negligible probability $\prg$ will not be true. Soundness of
the proof of knowledge sub-protocol then implies that $\tthm'$ is true. But this
means that $\thmm'$ is true.
$\Pi_\FLS$ is also zero-knowledge (in addition, to simulating for $\tthm$, the simulator
also uses the equivocal commitment property to decommit to an $R$
such that $\prg$ is true.). Furthermore, $\Pi_\FLS$  is an argument of knowledge for $\thmm$.
  
Note that although we are using \nmcom\ we are not making any claim here that uses non-malleability. All claims above simply rely on the hiding of \nmcom. The non-malleability is used by the two-party protocol which uses $\Pi_\FLS$.

Also note that in order to handle a general \nmcom\ of $k$ rounds, simply execute the first $k-3$ rounds before the protocol above begins. The statements are then modified to work with the transcript, rather than the first message of the protocol.



%%%%%%%%%%%%%%%%%%%%%%%%%%%%%%%%%%%%%%%%%%%%%%%%%%%%%%%%%%%%%%%%%%%%%%
\section{The Need for Dual GSW}\label{sec:whyDual}
For the interested reader, we explain below why we need to use the
``dual'' GSW scheme rather than the ``primal'' GSW as in
\cite{C:CleMcg15,MW16}. As we explained, the main differece
between the primal and dual schemes is that that the matrix~$A$ in
``primal'' GSW is $(n-1)$-by-$m$, while in our ``dual'' scheme it
is $(m-1)$-by-$n$ (in both cases we have $m<n\log q$). While it is
certailny possible that a one-round ILWE-hard protocol exists also
for the ``primal'' scheme, we were not able to find one that we can
prove secure under any standard assumption. Below we detail some
specific failed attempts.

%----------------------------------------------------------------
\paragraph{Failed attempt \#1, parties choose different columns.}
Consider a protocol simlar to the one in \secref{ILWE-Prot}, in which
each party $P_i$ is choosing a random $n\times m'$ matrix $A_i$ and
the matrix $A\in \ZZ_q^{n\times m'N}$ is just the column-concatenation
of all the $A_i$'s, $A=(A_1|A_2|\ldots|A_N)$.

An adversary (who controls $P_N$ without loss of generality), can just
set its matrix as $A_N=G$ where $G$ is the GSW ``gadget matrix''.
That gadget matrix has the property that given the vector $sG+e$ for
a small error vector $e$, it is easy to find $e$ and $s$. Now, notice
that the vector $sA+e$ that $P_N$ sends to $\rA$ has the form
$(sA_1+e_1|sA_2+e_2|\ldots|sA_N+e_N)$, so in particular the adversary
can set the portion $sA_N+e_N=sG+e_N$ to recover the secret key~$s$.
(This is exactly where the ``dual'' scheme helps: the adversary still
sees some ``leakage'' $sA_N+e_N$, but it cannot recover $s$ since $s$
still has a lot of min-entropy even given that leakage.)

%----------------------------------------------------------------
\paragraph{Failed attempt \#2, parties choose different rows.}
One way to avoid attacks as above is to ensure that for any fixed
matrix that the adversary may put in ``its entries'', a random matrix
by the honest user will make $sA+e$ pseudorandom.

One way to ensure this is to let each party choose a random $n'\times
m$ matrix $A_i$ and set $A\in\ZZ_q^{Nn'\times m}$ as the
row-concatenation of the $A_i$'s, i.e., $A^T=(A_1^T|\ldots|A_N^T)$. It
is now easy to prove that $sA+e$ is pseudorandom (under LWE), no
matter what the adversary does. But this arrangement opens another
avenue of attack: The adversary (still controlling $P_N$) set
$A_N=A_1$, so the bottom few rows in $A$ are equal to the top few
rows. Hence, also the bottom few rows in $AR$ are equal to the top few
rows, which lets the adversary distinguish $AR$ from a uniform
random~$U$.

%----------------------------------------------------------------
\paragraph{Some other failed attempts.}
At this point one may hope that if we let the parties choose different
diagonals then neither of the attacks above would apply, but this is
not the case. Here too, an adversary controlling all but one party can
force the matrix~$A$ to have many identical rows, which would mean
that so does the matrix~$AR$. More generally, it seems that any
arrangement where each party chooses a subset of the entries in~$A$
will let the adversary force~$A$ to be low rank, and hence also $AR$
will be of low rank. (Here too the ``dual'' scheme works better, since
the attacker sees $AR+E$ rather than $AR$ itself.)

%\anti{Does it worth to mention the variant of the attack where the
%secret $s$ is a matrix (short secret LWE assumption)?}
%\shai{No, why does it matter? This attack completely ignores $s$.}

\iffalse
 %----------------------------------------------------------------
 \paragraph{A simple plausible candidate, parties choose different bits.}
 Letting different parties choose different entries in $A$ does not
 seem to work, but we can instead let each party choose some
 \emph{bits} in each entry. For example, with $N$ parties we can set
 $q=2^{\kappa N}$, then let party~$P_1$ choose bits
 $0,N,2N,\ldots,N(\kappa-1)$ in each entry of~$A$, party $P_2$ choose
 bits $1,N+1,2N+1,\ldots,N(\kappa-1)+1$, etc.
 
 As far as we can see, the two lines of attacks from above do not apply
 to this candidate. On one hand, if the adversary's bits are fixed
 irrespective of the bits of the honest party, then each column of $A$
 would have sufficient entropy to render $sA+e$ pseudorandom. On the
 other hand, the honest party controls enough bits in every row, so it
 seems hard for the adversary to cause $A$ have low rank.
\fi






































%%%%%%%%%%%%%%%%%%%%%%%%%%%%%%%%%%%%%%%%%%
%%%%%%%%%%%%%%%%%%%%%%%%%%%%%%%%%%%%%%%%%%
%%%%%% NONE OF THIS IS BEING USED, TO BE DELETED LATER%%%%%%
%%%%%%%%%%%%%%%%%%%%%%%%%%%%%%%%%%%%%%%%%%
%%%%%%%%%%%%%%%%%%%%%%%%%%%%%%%%%%%%%%%%%%

\iffalse
\section{Preliminaries}

\paragraph{Basic notations.}
We denote the security parameter by $\kappa$. We say that a function $\mu:\NN\rightarrow\NN$ is {\em negligible} if for every positive polynomial $p(\cdot)$ and all sufficiently large $\kappa$'s it holds that $\mu(\kappa)<\frac{1}{p(\kappa)}$. We use the abbreviation \ppt\ to denote probabilistic polynomial-time. We specify next the definition of computationally indistinguishable and statistical distance.

\BD
Let $X=\{X(a,\kappa)\}_{a\in\bit^*,\kappa\in\NN}$ and $Y=\{Y(a,\kappa)\}_{a\in\bit^*,\kappa\in\NN}$ be two distribution ensembles. We say that $X$ and $Y$ are {\em computationally indistinguishable}, denoted $X\indist Y$, if for every \ppt\ machine $D$, every $a\in\bit^*$, every positive polynomial $p(\cdot)$ and all sufficiently large $\kappa$'s,
$$
\big|\prob\left[D(X(a,\kappa),1^\kappa)=1\right]-\prob\left[D(Y(a,\kappa),1^\kappa)=1\right]\big|
<\frac{1}{p(\kappa)}.
$$
\ED

\BD
Let $X_\kappa$ and $Y_\kappa$ be random variables accepting values taken from a finite domain $\Omega\subseteq\bit^\kappa$. The \emph{statistical distance} between $X_\kappa$ and $Y_\kappa$ is
$$
SD(X_\kappa,Y_\kappa)=\frac{1}{2}\sum_{\omega\in\Omega}\big|\Pr[X_\kappa=\omega]-\Pr[Y_\kappa=\omega]\big|.
$$
We say that $X_\kappa$ and $Y_\kappa$ are \emph{$\varepsilon$-close} if their statistical distance is at most $SD(X_\kappa,Y_\kappa) \le  \varepsilon(\kappa)$. We say that $X_\kappa$ and $Y_\kappa$ are \emph{statistically close}, denoted $X_\kappa\approx_s Y_\kappa$, if $\varepsilon(\kappa)$ is negligible in $\kappa$.
\ED


\subsection{Commitment Schemes}\label{sec:com}

Commitment schemes are used to enable a party, known as the {\em sender} $\sen$, to commit itself to a value while keeping it secret from the {\em receiver} $\rec$ (this property is called \emph{hiding}). Furthermore, in a later stage when the commitment is opened, it is guaranteed that the ``opening'' can yield only a single value determined in the committing phase (this property is called \emph{binding}). In this work, we consider commitment schemes that are \emph{statistically binding}, namely while the hiding property only holds against computationally bounded (non-uniform) adversaries, the binding property is required to hold against unbounded adversaries. Formally,

\BD[Commitment schemes]\label{def:com}
A \ppt\ machine $\Com = \langle S, R\rangle$ is said to be a non-interactive commitment scheme if the following two properties hold.
\begin{description}
\item[Computational hiding:] For every (expected) \ppt\ machine $\rec^*$, it holds that the following ensembles are computationally indistinguishable.
\BI
\item $\{\view_{\Com}^{\rec^*}(m_1,z)\}_{\kappa \in N,m_1, m_2 \in\{0,1\}^\kappa,z\in\{0,1\}^*}$

\item $\{\view_{\Com}^{\rec^*}(m_2,z)\}_{\kappa \in N,m_1, m_2 \in\{0,1\}^\kappa,z\in\{0,1\}^*}$
\EI
where $\view_{\Com}^{R^*}(m,z)$ denotes the random variable describing the output of $\rec^*$ after receiving a commitment to $m$ using $\Com$.

\item[Statistical binding:] %Informally, the statistical-binding property asserts that, with overwhelming probability over the coin-tosses of the receiver $\rec$, the transcript of the interaction fully determines the value committed to by the sender.  More formally,
For any (computationally unbounded) malicious sender $\sen^*$ and auxiliary input $z$, it holds that the probability that there exist valid decommitments to two different values for a view $v$, generated with an honest receiver while interacting with $\sen^*(z)$ using $\Com$, is negligible.
\end{description}
\ED
We refer the reader to \cite{Goldreich01} for more details. We recall that non-interactive perfectly binding commitment schemes can be constructed based on one-way permutation, whereas two-round statistically binding commitment schemes can be constructed based on one-way functions~\cite{Naor91}.
To set up some notations, we let $\com_m \leftarrow \Com(m; r_m)$ denote a commitment to a message $m$, where the sender uses uniform random coins $r_m$. The decommitment phase consists of the sender sending the decommitment information $\decom_m = (m, r_m)$ which contains the message $m$ together with the randomness $r_m$. This enables the receiver to verify whether $\decom_m$ is consistent with the transcript $\com_m$. If so, it outputs $m$; otherwise it outputs $\bot$. For simplicity of exposition, in the sequel, we will assume that random coins are an implicit input to the commitment functions, unless specified explicitly.

\anti{insert notation for non-malleable commitments}


\remove{\BD[Trapdoor commitment schemes]\label{def:tcom}
Let $\Com=(\sen,\rec)$ be a statistically binding commitment scheme. We say that $\Com$ is a trapdoor commitment scheme is there exists an expected \ppt\ oracle machine $\cS = (\cS_1,\cS_2)$ such that for any \ppt\ $\rec^*$ and all $m\in\bit^\kappa$, the output $(\tau,w)$ of the following experiments is computationally indistinguishable:
\BDE
\item[-] an honest sender $\sen$ interacts with $\rec^*$ to commit to $m$, and then opens the commitment: $\tau$ is the view of $\rec^*$ in the commit phase, and $w$ is the message $\sen$ sends in the open phase.

\item[-] the simulator $\cS$ generates a simulated view $\tau$ for the commit phase, and then opens the commitment to $m$ in the open phase: formally $(\tau,state)\gets\cS_1^{\rec^*}(1^\kappa)$, $w\gets\cS_2(state,m)$.
\EDE
\ED}



\subsection{Hardcore Predicates}

\BD[Hardcore predicate]\label{def:hardcore}\anti{if we keep this definition I have to modify it}
Let $f : \bit^\kappa \rightarrow \bit^*$ and $\hb: \bit^\kappa \rightarrow \bit$ be a polynomial-time computable functions. We say $\hb$ is a hardcore predicate of $f$, if for every \ppt\ machine $A$, there exists a negligible function $\ngl(\cdot)$ such that
$$
\Pr[x \leftarrow \bit^\kappa; y = f(x) : A(1^\kappa,y) = \hb(x) ] \leq \frac{1}{2} + \ngl(\kappa).
$$
\ED

\remove{An important theorem by Goldreich and Levin~\cite{GoldreichL89} states that if $f$ is a one-way function over $\bit^\kappa$ then the one-way function $f'$ over $\bit^{2\kappa}$, defined by $f'(x,r)=(f(x),r)$, admits the following hardcore predicate $b(x,r)=\langle x,r \rangle =\Sigma x_i r_i \bmod 2$, where $x_i,r_i$ is the $i$th bit of $x,r$ respectively. In the following, we refer to this predicate as the GL bit of $f$. We will use the following theorem that establishes the list-decoding property of the GL bit.

\BT[\cite{GoldreichL89}]\label{thm:gole} There exists a \ppt\ oracle machine $\Inv$ that on input $(\kappa,\varepsilon)$ and oracle access to a predictor \ppt\ $B$, runs in time $poly(\kappa,\frac{1}{\varepsilon})$, makes at most $O(\frac{\kappa^2}{\varepsilon^2})$ queries to $B$ and outputs a list $L$ with $|L| \leq \frac{4\kappa}{\varepsilon^2}$ such that if
$$
\Pr[r \leftarrow \bit^\kappa: B(r) = \langle x,r \rangle] \geq \frac{1}{2}+\frac{\varepsilon}{2}
$$
then
$$
\Pr[L \leftarrow \Inv^B(\kappa,\varepsilon) : x \in L] \geq \frac{1}{2}.
$$
\ET
\anti{update the hardcore predicates}}


\remove{\subsection{Trapdoor Permutations}

\BD[Trapdoor Permutation] Let $\cF = (\Gen, \Eval, \Invert)$ be three \ppt algorithms such
that
\BI
\item $\Gen(1^\kappa)$ outputs a pair $(f,{\sf trap})$ where $f : \{0, 1\}^\kappa \rightarrow \{0, 1\}^\kappa$ is a permutation;
\item $\Eval(f, \cdot) = f(\cdot)$ evaluates $f$; and
\item $\Invert(f,{\sf trap}, \cdot) = f^{-1}(\cdot)$ evaluates $f^{-1}$.
\EI
We say that $\cF$ is a family of trapdoor permutations (TDPs) if for any \ppt algorithm $\sf R$
$$\Pr[{(f,{\sf trap})\leftarrow\Gen(1^\kappa);y\leftarrow \{0,1\}^\kappa}:({\sf R}(f, y) = f^{-1}(x)\big)]= negl(\kappa)$$.
\ED}




\remove{\subsection{Secret-Sharing}\label{def:ss}

A secret-sharing scheme allows distribution of a secret among a group of $n$ players, each of whom in a \emph{sharing phase} receive a share (or piece) of the secret. In its simplest form, the goal of secret-sharing is to allow only subsets of players of size at least $t+1$ to reconstruct the secret. More formally a $t+1$-out-of-$n$ secret sharing scheme comes with a sharing algorithm that on input a secret $s$ outputs $n$ shares $s_1,\ldots,s_n$ and a reconstruction algorithm that takes as input $((s_i)_{i \in S},S)$ where $|S| > t$ and outputs either a secret $s'$ or $\bot$. In this work, we will use the Shamir's secret sharing scheme~\cite{Shamir79} with secrets in $\FF = GF(2^\kappa)$. We present the sharing and reconstruction algorithms below:
\begin{description}
\item[Sharing algorithm:] For any input $s \in \FF$, pick a random polynomial $f(\cdot)$ of degree $t$ in the polynomial-field $\FF[x]$ with the condition that $f(0) = s$ and output $f(1),\ldots,f(n)$.

\item[Reconstruction algorithm:] For any input $(s_i')_{i \in S}$ where none of the $s_i'$ are $\bot$ and $|S| > t$, compute a polynomial $g(x)$ such that $g(i) = s_i'$ for every $i \in S$. This is possible using Lagrange interpolation where $g$ is given by
$$
g(x) = \sum_{i \in S} s_i' \prod_{j \in S/\{i\}} \frac{x - j}{i-j}~.
$$
Finally the reconstruction algorithm outputs $g(0)$.
\end{description}
%
We will additionally rely on the following property of secret-sharing schemes. To this end, we view the Shamir secret-sharing scheme as a linear code generated by the following $n\times (t+1)$ Vandermonde matrix
$$
A=\left(
\begin{array}{ccccc}
1 & 1^2& \cdots& 1^{t}\\
1 & 2^2 &\cdots& 2^{t}\\
\vdots&\vdots&\vdots&\vdots \\
1& n^2&\cdots&n^{t}
\end{array}
\right)
$$
More formally, the shares of a secret $s$ that are obtained via a polynomial $f$ in the Shamir scheme, can be obtained by computing $A\textbf{c}$ where $\textbf{c}$ is the vector containing the coefficients of $f$. Next, we recall that for any linear code $A$, there exists a parity check matrix $H$ of dimension $(n-t-1)\times n$ which satisfies the equation $HA=\textbf{0}_{(n-t-1)\times (t+1)}$, i.e. the all $0$'s matrix. We thus define the linear operator $\phi(v) = Hv$ for any vector $v$. Then it holds that any set of shares $\textbf{s}$ is valid if and only if it satisfies the equation $\phi(\textbf{s}) = \textbf{0}_{n-t-1}$.

%The authors in \cite{DZ13} were the first to propose an algorithm for verifying membership in (binary) codes, i.e., verifying the product of Boolean matrices in quadratic time with exponentially small error probability, while previous methods only achieved constant error.
}

\section{Garbled Circuits}





\paragraph{Yao Garbling.} \anti{move this to the appendix}We briefly describe the garbling technique of Yao~\cite{Yao86} as described by Lindell and Pinkas in~\cite{LindellP09}. In this construction, the desired function $f$ is represented by a boolean circuit $C$ that is computed gate by gate from the input wires to the output wires. In the following, we distinguish four different types of wires used in a given boolean circuit: ({\bf a}) circuit-input wires; ({\bf b}) circuit-output wires; ({\bf c}) gate-input wires (that enter some gate $g$); and ({\bf d}) gate-output wires (that leave some gate $g$). The underlying idea is to associate every wire $w$ with two random values, say $\lbl^0_w,\lbl^1_w$, such that $\lbl^0_w$ represents the bit $0$ and $\lbl^1_w$ represents the bit $1$. The algorithm $\GCircuit$ on input the security parameter $\sec$, the circuit $C$, and the set of labels $\lbl^{w}_b$ for all the wires $w \in \inp(C)$ and $b \in \bit$ generates the garbled table for each gate which maps random input values  to random output values, with the property that given two input values it is only possible to learn the output value that corresponds to the output bit. This is accomplished by viewing the  four potential inputs to a gate $\lbl^0_w,\lbl^1_w$ (values associated with the first input wire)  and $\lbl^0_{w+1},\lbl^1_{w+1}$ (values associated with the second input wire), as encryption keys. So that the output key values $\lbl^0_{w+2},\lbl^1_{w+2}$ are encrypted under the appropriate input keys. For instance, let $\gate$ be a NAND gate. Then, the output key $\lbl_{w+2}^1$ (that corresponds to bit $1$) is encrypted under the pair of keys associated with the values $(0,0),\;(0,1),\;(1,0)$. Whereas, the output key $\lbl^0_{w+2}$ is encrypted under the pair of keys associated with $(1,1)$ which yields the following four ciphertexts
$$
\enc_{\lbl^0_w}(\enc_{\lbl^0_{w+1}}(\lbl^1_{w+2}))$$$$\enc_{\lbl^0_w}(\enc_{\lbl^1_{w+1}}(\lbl^1_{w+2}))$$$$\enc_{\lbl^1_w}(\enc_{\lbl^0_{w+1}}(\lbl^1_{w+2}))$$$$ \enc_{\lbl^1_w}(\enc_{\lbl^1_{w+1}}(\lbl^0_{w+2})),
$$
where $(\gen,\enc,\dec)$ is a symmetric key encryption scheme that has {\em chosen double encryption security} and an {\em elusive efficiently verifiable range}; see~\cite{LindellP09} for the formal definitions. These ciphertexts are randomly permuted in order to obtain the garbled table for gate $\gate$. The evaluation algorithm $\Eval$ performing the same operation per gate of $C$ proceeds as follows. Given the input wire keys $\lbl^{\alpha}_w,\lbl^{\beta}_{w+1}$ for a $\gate$ in $C$ that correspond to the bits $\alpha$ and $\beta$ and the garbled table containing the four ciphertexts, it is possible to obtain the output wire key $\lbl^{\gate(\alpha,\beta)}_{w+2}$. The description of the garbled circuit is concluded with the {\em output decryption tables}, mapping the random values on the circuit output wires back to their corresponding boolean values.
\fi
