%
% Paper latex template
%
% Date: 2015-01-05
%
% Changes:
% - 2015-01-05: 
%     major revision
%     use of etoolbox toggle, cleveref, xspace todo, ...
%

%
% Parameters
%

\RequirePackage[l2tabu, orthodox]{nag}
\RequirePackage{etoolbox}

\newtoggle{full}          % false = conference version; true = full version
\newtoggle{showoverflow}  % true = show overflows
\newtoggle{allowtodo}     % false = remove todo command
\newtoggle{showtodo}      % true = show todo and notes; false = hide todo
\newtoggle{anonymous}     % true = anonymous
\newtoggle{submission}    % true = submission (force llncs style for crypto)
\newtoggle{llncs}

\providetoggle{forcefull} % true = force full version from outside (see paper-full.tex)
\providetoggle{forceconf} % true = force conf version from outside (see paper-conf.tex)
\iftoggle{forcefull}{
  \toggletrue{full}
}{
  \iftoggle{forceconf}{
    \togglefalse{full}
  }{
    \toggletrue{full}     % default value for full
  }
}


\toggletrue{submission}   % default value for submission
\toggletrue{showoverflow} % default value for showoverflow
\toggletrue{allowtodo}    % default value for allowtodo
\toggletrue{showtodo}     % default value for showtodo
\toggletrue{anonymous}    % default value for anonymous

\ifboolexpr{togl{full} and (not togl{submission})}{
  \togglefalse{llncs}
}{
  \toggletrue{llncs}
}

\newcommand{\remove}[1]{}
%
% Document class 
%
\iftoggle{llncs}{
  \documentclass[envcountsame,envcountsect,runningheads]{llncs}
}{
  % \documentclass[11pt,envcountsame,envcountsect,runningheads]{llncs}
  % \documentclass[11pt,envcountsame,envcountsect,runningheads]{llncs+}
  \documentclass[11pt]{article}

  % Palatino is obsolete but the alternative does not give exactly the same result...
  \usepackage{palatino}

  \usepackage[hmargin=1in,vmargin=1in]{geometry}
  \usepackage{amsthm,amsmath}
  \usepackage{fullpage}
}

%
% Custom header
%
% !TEX root =../main.tex

%
% Packages
%

% Standard packages
\usepackage[utf8]{inputenc}
\usepackage[T1]{fontenc}
\usepackage{lmodern}
\usepackage{microtype}

\usepackage{amsmath}
\usepackage{amssymb}
\usepackage{mathtools}
\usepackage{mathrsfs}

\usepackage{booktabs}
\usepackage{array}
\newcolumntype{L}[1]{>{\raggedright\arraybackslash}m{#1}}
\newcolumntype{C}[1]{>{\centering\let\newline\\\arraybackslash\hspace{0pt}}m{#1}}
\newcolumntype{R}[1]{>{\raggedleft\arraybackslash}m{#1}}
\usepackage{threeparttable}
\usepackage{blkarray}
\usepackage{multirow}

\usepackage[noadjust]{cite}
\usepackage{url}
\usepackage{xcolor}
\usepackage{xspace}
\usepackage{needspace}
\usepackage[inline]{enumitem}

\usepackage{pifont}
\newcommand{\cmark}{\ding{51}}
\newcommand{\xmark}{\ding{55}}

\usepackage{soul}

\usepackage{tikz}

  % float has to be loaded before hyperref and algorithm after hyperref
\usepackage{float}
\usepackage{graphicx}
\usepackage{caption,subcaption}
\usepackage{framed}
\usepackage{mdframed}
\mdfsetup{skipabove=0pt,skipbelow=0pt}

\iftoggle{allowtodo}{
  \iftoggle{showtodo}{
    \usepackage[textsize=small]{todonotes}
  }{
    \usepackage[disable]{todonotes}
  }
  % add xspace to todo command (http://tex.stackexchange.com/a/68741)
  \makeatletter
  \expandafter\apptocmd\csname\string\todo\endcsname{\xspace}{}{}
  \makeatother
}{
}

% Personal packages
\usepackage{games}

% Hyperref
\usepackage[pdfpagelabels=true,\iftoggle{llncs}{}{pagebackref}]{hyperref}
%\usepackage[colorlinks=true,linkcolor=blue,citecolor=blue]{hyperref}

\hypersetup{
  linktoc = page,
  pdfpagemode = UseNone,
  colorlinks,
  linkcolor={red!50!black},
  citecolor={blue!50!black},
  urlcolor={blue!80!black}
}

\iftoggle{llncs}{}{
  \renewcommand*{\backref}[1]{}
  \renewcommand*{\backrefalt}[4]{
    \ifcase #1
    (Not cited.)
    \or
    (Page~#2.)
    \else
    (Pages~#2.)
    \fi
  }
  \renewcommand*{\backrefsep}{, }
  \renewcommand*{\backreftwosep}{ and~}
  \renewcommand*{\backreflastsep}{, and~}
}


% Packages to be loaded AFTER hyperref

\usepackage[capitalize]{cleveref}
% https://tex.stackexchange.com/a/161340/34384
\newcommand{\creflastconjunction}{, and\nobreakspace}

\usepackage{algorithm}
\usepackage[noend]{algpseudocode}
  % decrease algorithm indent http://tex.stackexchange.com/a/40289/34384
\expandafter\patchcmd\csname\string\algorithmic\endcsname%
  {\labelwidth 0.5em}{\labelwidth0pt\labelsep0pt}{}{} 
  % fix bugs with labels on lines of algorithmicx http://tex.stackexchange.com/a/177033/34384
\makeatletter
\newcounter{algorithmicH}% New algorithmic-like hyperref counter
\let\oldalgorithmic\algorithmic
\renewcommand{\algorithmic}{%
  \stepcounter{algorithmicH}% Step counter
  \oldalgorithmic}% Do what was always done with algorithmic environment
\renewcommand{\theHALG@line}{ALG@line.\thealgorithmicH.\arabic{ALG@line}}
\makeatother

\usepackage{soul}

% Fix a bug when llncs is used with both envcountsame and envcountsect
% https://tex.stackexchange.com/a/372108/34384

\def\theHcase{\theHtheorem}
\def\theHconjecture{\theHtheorem}
\def\theHcorollary{\theHtheorem}
\def\theHdefinition{\theHtheorem}
\def\theHexample{\theHtheorem}
\def\theHexercise{\theHtheorem}
\def\theHlemma{\theHtheorem}
\def\theHnote{\theHtheorem}
\def\theHproblem{\theHtheorem}
\def\theHproperty{\theHtheorem}
\def\theHproposition{\theHtheorem}
\def\theHquestion{\theHtheorem}
\def\theHsolution{\theHtheorem}
\def\theHremark{\theHtheorem}

%
% Misc
%

\iftoggle{llncs}{
  %% change proof to accept the optional argument as article
  \let\llncsproof\proof
  \renewcommand{\proof}[1][]{%
    \ifx!#1!\else\renewcommand{\proofname}{#1}\fi
    \llncsproof
  }
  % add automatically qed to proof
  \AtEndEnvironment
  {proof}
  {\qed}
}{}

\setcounter{tocdepth}{2}

\newsavebox{\fboxenvbox}
\newenvironment{fboxenv}
    {\begin{lrbox}{\fboxenvbox}}
    {\end{lrbox}\fbox{\usebox{\fboxenvbox}}}

\iftoggle{showoverflow}{
  \overfullrule=10pt
}{}

\newcommand{\textwidthfbox}{\dimexpr\textwidth-2\fboxsep-2\fboxrule\relax} % text width inside a fbox/fboxenv

%
% References
%

\iftoggle{full}{
  \newcommand*{\appref}[1]{Appendix~\ref{#1}}
}{
  \newcommand*{\appref}[1]{the full version}
}

%
% General and Math commands
%

% Abbreviations
\newcommand{\etal}{et al.\xspace}

% Math sets and groups
\newcommand{\F}{\mathbb{F}}
\newcommand{\Z}{\mathbb{Z}}
\newcommand{\N}{\mathbb{N}}
\newcommand{\G}{\mathbb{G}}
\newcommand{\Gt}{\mathbb{G}_T}

% Probability
\newcommand*{\prob}[1]{{\Pr}\left[\,{#1}\,\right]}
\newcommand*{\probb}[2]{{\Pr}_{#1}\left[\,{#2}\,\right]}

% Math misc
% \makeatletter
% \newcommand\suchthat{%
%  \@ifstar
%   {\mathrel{}\middle|\mathrel{}}
%   {\mid}%
% }
% \makeatother
\newcommand\suchthat{:}

\newcommand{\defeq}{\coloneqq}
\newcommand{\eqdef}{\eqqcolon}
\newcommand{\eeq}{\overset{{}_{?}}{=}}

\newcommand{\nexts}{\:;\:}

\newcommand{\xor}{\mathbin{\mathsf{xor}}}

\newcommand{\numberthis}{\addtocounter{equation}{1}\tag{\theequation}}

%
% General Crypto commands
%

% Algorithms
% \newcommand{\getsr}{\stackrel{{}_R}{\leftarrow}}
\newcommand{\getsr}{\gets}

% Parameters
\newcommand{\secpar}{\lambda}

% Experiments / Game
\newcommand{\advA}{A}
\newcommand{\state}{\mathsf{st}}

\newcommand{\Exp}{\mathsf{Exp}}
\newcommand{\Succ}{\mathsf{Succ}}  % success
\newcommand{\Adv}{\mathsf{Adv}}    % advantage

\newcommand{\secu}{\web{sec}}
\newcommand{\SECU}{\web{SEC}}

\algdef{SnE}[Oracle]{Oracle}{EndOracle}
  [1]{\ul{#1}}
  {}

\algdef{SnE}[AlgoExp]{AlgoExp}{EndAlgoExp}
  [1]{#1}
  {}

% Protocol flows
\newcommand*{\Rflow}[2][5em]{\ensuremath{\xrightarrow{\smash{\makebox[#1][c]{$#2$}}}}}
\newcommand*{\Lflow}[2][5em]{\ensuremath{\xleftarrow{\smash{\makebox[#1][c]{$#2$}}}}}
\newcommand*{\LRflow}[3][5em]{\ensuremath{\xleftrightarrow[\smash{\makebox{$#2$}}]{\smash{\makebox[#1][c]{$#3$}}}}}

% Crypto misc
\newcommand{\ve}{\varepsilon}
\newcommand{\eps}{\ve}

\DeclareMathOperator{\negl}{negl}
% \DeclareMathOperator{\poly}{poly}
\newcommand{\poly}{\text{poly}}
\DeclareMathOperator{\lcm}{lcm}

% Hard problems (uppercase) and associated experiment names (lowercase)
\newcommand{\hardprobfont}[1]{\texorpdfstring{\ensuremath{\mathsf{#1}}}{#1}}

\newcommand{\DL}{\hardprobfont{DL}\xspace}
\newcommand{\dl}{\hardprobfont{dl}\xspace}
\newcommand{\DHtup}{\hardprobfont{DH}\xspace}    % DH tuple

\newcommand{\DDH}{\hardprobfont{DDH}\xspace}
\newcommand{\ddh}{\hardprobfont{ddh}\xspace}

\newcommand{\MDDH}{\hardprobfont{MDDH}\xspace}
\newcommand{\mddh}{\hardprobfont{mddh}\xspace}

\newcommand{\DLin}{\hardprobfont{DLin}\xspace}
\newcommand{\dlin}{\hardprobfont{dlin}\xspace}

\newcommand{\XDH}{\hardprobfont{XDH}\xspace}
\newcommand{\xdh}{\hardprobfont{xdh}\xspace}

\newcommand{\CDH}{\hardprobfont{CDH}\xspace}
\newcommand{\cdh}{\hardprobfont{cdh}\xspace}

\newcommand{\SXDH}{\hardprobfont{SXDH}\xspace}
\newcommand{\sxdh}{\hardprobfont{sxdh}\xspace}

\newcommand{\BDDH}{\hardprobfont{BDDH}\xspace}
\newcommand{\bddh}{\hardprobfont{bddh}\xspace}

\newcommand{\BDDHI}{\hardprobfont{BDDHI}\xspace}
\newcommand{\bddhi}{\hardprobfont{bddhi}\xspace}

\newcommand{\klin}{\texorpdfstring{\kappa}{k}}
\newcommand*{\Lin}[1]{\texorpdfstring{\ensuremath{{#1\text{-}\mathsf{Lin}}}}{#1-Lin}\xspace}
\newcommand*{\lin}[1]{\texorpdfstring{\ensuremath{{#1\text{-}\mathsf{lin}}}}{#1-lin}\xspace}

%
% Commands specific to the paper 
%

% Wide bar https://tex.stackexchange.com/a/60253/34384

\makeatletter
\let\save@mathaccent\mathaccent
\newcommand*\if@single[3]{%
  \setbox0\hbox{${\mathaccent"0362{#1}}^H$}%
  \setbox2\hbox{${\mathaccent"0362{\kern0pt#1}}^H$}%
  \ifdim\ht0=\ht2 #3\else #2\fi
  }
%The bar will be moved to the right by a half of \macc@kerna, which is computed by amsmath:
\newcommand*\rel@kern[1]{\kern#1\dimexpr\macc@kerna}
%If there's a superscript following the bar, then no negative kern may follow the bar;
%an additional {} makes sure that the superscript is high enough in this case:
\newcommand*\widebar[1]{\@ifnextchar^{{\wide@bar{#1}{0}}}{\wide@bar{#1}{1}}}
%Use a separate algorithm for single symbols:
\newcommand*\wide@bar[2]{\if@single{#1}{\wide@bar@{#1}{#2}{1}}{\wide@bar@{#1}{#2}{2}}}
\newcommand*\wide@bar@[3]{%
  \begingroup
  \def\mathaccent##1##2{%
%Enable nesting of accents:
    \let\mathaccent\save@mathaccent
%If there's more than a single symbol, use the first character instead (see below):
    \if#32 \let\macc@nucleus\first@char \fi
%Determine the italic correction:
    \setbox\z@\hbox{$\macc@style{\macc@nucleus}_{}$}%
    \setbox\tw@\hbox{$\macc@style{\macc@nucleus}{}_{}$}%
    \dimen@\wd\tw@
    \advance\dimen@-\wd\z@
%Now \dimen@ is the italic correction of the symbol.
    \divide\dimen@ 3
    \@tempdima\wd\tw@
    \advance\@tempdima-\scriptspace
%Now \@tempdima is the width of the symbol.
    \divide\@tempdima 10
    \advance\dimen@-\@tempdima
%Now \dimen@ = (italic correction / 3) - (Breite / 10)
    \ifdim\dimen@>\z@ \dimen@0pt\fi
%The bar will be shortened in the case \dimen@<0 !
    \rel@kern{0.6}\kern-\dimen@
    \if#31
      \overline{\rel@kern{-0.6}\kern\dimen@\macc@nucleus\rel@kern{0.4}\kern\dimen@}%
      \advance\dimen@0.4\dimexpr\macc@kerna
%Place the combined final kern (-\dimen@) if it is >0 or if a superscript follows:
      \let\final@kern#2%
      \ifdim\dimen@<\z@ \let\final@kern1\fi
      \if\final@kern1 \kern-\dimen@\fi
    \else
      \overline{\rel@kern{-0.6}\kern\dimen@#1}%
    \fi
  }%
  \macc@depth\@ne
  \let\math@bgroup\@empty \let\math@egroup\macc@set@skewchar
  \mathsurround\z@ \frozen@everymath{\mathgroup\macc@group\relax}%
  \macc@set@skewchar\relax
  \let\mathaccentV\macc@nested@a
%The following initialises \macc@kerna and calls \mathaccent:
  \if#31
    \macc@nested@a\relax111{#1}%
  \else
%If the argument consists of more than one symbol, and if the first token is
%a letter, use that letter for the computations:
    \def\gobble@till@marker##1\endmarker{}%
    \futurelet\first@char\gobble@till@marker#1\endmarker
    \ifcat\noexpand\first@char A\else
      \def\first@char{}%
    \fi
    \macc@nested@a\relax111{\first@char}%
  \fi
  \endgroup
}
\makeatother


\newcommand{\bits}{{\{0,1\}}}

% distinguisher

\newcommand{\dist}{{\cal D}}	

% circuit classes

\newcommand{\circlass}{\mathcal{C}}
\newcommand{\univcirc}{U}
\newcommand{\univcir}{\univcirc}
\newcommand{\cirin}{x}
\newcommand{\cirinlen}{n}
\newcommand{\ciroutlen}{l}
\newcommand{\cir}{C}
\newcommand{\cirout}{y}
\newcommand{\cirsize}{S}

% com

\newcommand{\com}{\mathsf{com}}
\newcommand{\Com}{\mathsf{Com}}
\newcommand{\comcom}{\mathsf{COM.Com}}
\newcommand{\comver}{\mathsf{COM.Ver}}
\newcommand{\comc}{c}
\newcommand{\comr}{\rho}
\newcommand{\comrlen}{\tau}
\newcommand{\commsg}{x}

% hom com

\newcommand{\hc}{\mathsf{FC}}
\newcommand{\hccom}{\mathsf{FC.Com}}
\newcommand{\hcopen}{\mathsf{FC.FOpen}}
\newcommand{\hcver}{\mathsf{FC.FVer}}
\newcommand{\hcsim}{\mathsf{FC.SimC}}
\newcommand{\csim}{\mathsf{FC.Sim}}
\newcommand{\encsim}{\mathsf{FC.SimE}}
%\newcommand{\hcsetup{\mathsf{FC.Setup^{bind}}}
%\newcommand{\ehcsetup{\mathsf{FC.Setup^{equiv}}}

\newcommand{\ehcsetup}{\mathsf{FC.Setup_{equiv}}}
\newcommand{\setup}{\mathsf{FC.Setup}}
\newcommand{\hcsetup}{\mathsf{FC.Setup_{bind}}}

\newcommand{\tex}{\mathsf{trap_e}}
\newcommand{\teq}{\mathsf{trap_q}}
\newcommand{\cteq}{\mathsf{ctrap_q}}

\newcommand{\cc}{\mathsf{c}}
\newcommand{\ee}{\mathsf{e}}
\newcommand{\ff}{\mathsf{f}}



\newcommand{\hcc}{c}
\newcommand{\hcd}{d}
\newcommand{\hcr}{\rho}
\newcommand{\renc}{\zeta}
\newcommand{\hcrlen}{\tau}
\newcommand{\hcmsg}{v}
\newcommand{\hcout}{y}
\newcommand{\hcsize}{S}

\newcommand{\hccirclass}{\mathcal{G}}
\newcommand{\hccirinlen}{n}
\newcommand{\hcciroutlen}{l}
\newcommand{\hccir}{G}
\newcommand{\hccirout}{y}

\newcommand{\ehc}{\mathsf{eFC}}
\newcommand{\ehccom}{\mathsf{eFC.Com}}
\newcommand{\ehcopen}{\mathsf{eFC.FOpen}}
\newcommand{\ehcver}{\mathsf{eFC.FVer}}
\newcommand{\ehcsimc}{\mathsf{eFC.SimC}}
\newcommand{\Equiv}{\mathsf{FC.Equiv}}
\newcommand{\ehcsimd}{\mathsf{FC.EquivC}}
\newcommand{\encequiv}{\mathsf{FC.EquivE}}
\newcommand{\fequiv}{\mathsf{FC.EquivF}}
\newcommand{\ehctrap}{\mathsf{trap_q}}
\newcommand{\eehctrap}{\mathsf{trap_e}}

\newcommand{\ihc}{\mathsf{iFC}}
\newcommand{\ihcS}{\mathsf{iFC.S}}
\newcommand{\ihcR}{\mathsf{iFC.R}}
\newcommand{\ihcopen}{\mathsf{iFC.FOpen}}
\newcommand{\ihcver}{\mathsf{iFC.FVer}}
\newcommand{\ihcsim}{\mathsf{iFC.Sim}}
\newcommand{\ihcRr}{{\rho'}}
\newcommand{\ihcRrlen}{{\tau'}}

\newcommand{\eihc}{\mathsf{eiFC}}
\newcommand{\eihcS}{\mathsf{eiFC.S}}
\newcommand{\eihcR}{\mathsf{eiFC.R}}
\newcommand{\eihcopen}{\mathsf{eiFC.FOpen}}
\newcommand{\eihcver}{\mathsf{eiFC.FVer}}
\newcommand{\eihcsimc}{\mathsf{eiFC.SimC}}
\newcommand{\eihcsimd}{\mathsf{eiFC.SimD}}
\newcommand{\eihcRr}{{\rho'}}
\newcommand{\eihcRrlen}{{\tau'}}

\newcommand{\interaction}[2]{\langle {#1},\allowbreak {#2} \rangle}

% WE

\newcommand{\we}{\mathsf{WE}}
\newcommand{\weenc}{\mathsf{WE.Enc}}
\newcommand{\wedec}{\mathsf{WE.Dec}}
\newcommand{\welang}{\mathcal{L}}
\newcommand{\wewitr}{\mathcal{R}}
\newcommand{\wewit}{\mathsf{w}}
\newcommand{\weword}{\mathsf{x}}
\newcommand{\wect}{\mathsf{ct}}
\newcommand{\wemsg}{\mathsf{M}}

% nondetalg

\newcommand{\nda}{{\mathcal{O}}}
\newcommand{\ndax}{q}
\newcommand{\ndaxlen}{n}
\newcommand{\ndaw}{w}
\newcommand{\ndawlen}{m}
\newcommand{\nday}{a}
\newcommand{\ndaylen}{l}
\newcommand{\ndadistr}{\mathrm{w}\mathcal{D}}
\newcommand{\ndaaux}{\mathsf{aux}}
\newcommand{\ndadistrind}{\mathsf{id}}

% WS

\newcommand{\ws}{\mathsf{WS}}
\newcommand{\wsenc}{\mathsf{\hc.Enc}}
\newcommand{\wsdec}{\mathsf{\hc.Dec}}
\newcommand{\wsct}{\mathsf{ct}}
\newcommand{\wsmsg}{\mathsf{X}}
\newcommand{\wsdistr}{\ndadistr}%\mathrm{w}\mathcal{D}}

\newcommand{\denc}{\mathsf{denc}}
\newcommand{\vecdenc}{\widebar{\denc}}

% GC
\newcommand{\ckey}{\mathsf{cKeys}}
\newcommand{\vkey}{\mathsf{vKeys}}
\newcommand{\fkey}{\mathsf{fKeys}}




\newcommand{\gc}{\mathsf{GC}}
\newcommand{\gckey}{\mathsf{dataKeys}}
\newcommand{\Gckey}{\mathsf{stateKeys}}
\newcommand{\gcgen}{\mathsf{GC.Gen}}
\newcommand{\gcequiv}{\mathsf{GC.Equiv}}
\newcommand{\gcgarble}{\mathsf{GC.Garble}}
\newcommand{\gceval}{\mathsf{GC.Eval}}
\newcommand{\gcsim}{\mathsf{GC.Sim}}
\newcommand{\gccir}{\widehat{C}}
\newcommand{\gcsimcir}{\widetilde{C}}
\newcommand{\gcinlen}{\alpha}

\newcommand{\tgc}{{\mathsf{oGC}}}
\newcommand{\tgckey}{{\mathsf{okey}}}
\newcommand{\tgcgen}{{\mathsf{oGC.Gen}}}
\newcommand{\tgcgarble}{{\mathsf{oGC.Garble}}}
\newcommand{\tgceval}{{\mathsf{oGC.Eval}}}
\newcommand{\tgcsim}{{\mathsf{oGC.Sim}}}
\newcommand{\tgccir}{\widehat{\mathsf{o}C}} 
\newcommand{\tgcsimcir}{\widetilde{\mathsf{o}C}}

\newcommand{\tcir}{\mathsf{o}C}
\newcommand{\tcirsize}{{\mathsf{o}S}}

% MPC / functionality
\newcommand{\mpcfunc}{f}
\newcommand{\mpcin}{x}
\newcommand{\mpcvecin}{\bar{x}}
\newcommand{\mpcout}{y}
\newcommand{\mpcinlen}{\kappa}
\newcommand{\mpcnext}{\mathsf{Nextmsg}}
\newcommand{\mpcoutput}{\mathsf{Output}}
\newcommand{\round}{\ell}
\newcommand{\nbrounds}{L}
\newcommand{\mpcrand}{r}
\newcommand{\mpcvecrand}{\bar{\mpcrand}}
\newcommand{\mpcrandlen}{R}
\newcommand{\msg}{m}
\newcommand{\Msg}{M}
\newcommand{\vecmsg}{\vec m}
\newcommand{\mpc}{\pi}
\newcommand{\MPC}{\Pi_\mathsf{MPC}}
\newcommand{\sMPC}{\Pi_\mathsf{sMPC}}
\newcommand{\impc}{\Phi}
\newcommand{\mpcview}{\mathsf{View}}
\newcommand{\mpcsim}{\mathsf{\cS}}

\newcommand{\mpcideal}{\mathsf{Ideal}}
\newcommand{\mpcreal}{\mathsf{Real}}
\newcommand{\mpcrealsm}{\mathsf{Real}^{\mathsf{sm}}}
\newcommand{\mpcrealdef}{\mathsf{Real}^{\mathsf{def}}}

% Tools for constructing malicious MPC
\newcommand{\NMZK}{\mathsf{NMZK}}
\newcommand{\WI}{\mathsf{WI}^{\mathrm{or}}}
\newcommand{\WIo}{\mathsf{WI}^{\mathrm{or,1}}}
\newcommand{\WIt}{\mathsf{WI}^{\mathrm{or,2}}}
\newcommand{\OTz}{\mathsf{OT}^{\mathrm{0}}}
\newcommand{\OTo}{\mathsf{OT}^{\mathrm{1}}}


\newcommand{\qOTsetup}{\mathsf{qOT.Setup_{equiv}}}


\newcommand{\OTsetup}{\mathsf{OT.Setup_{bind}}}
\newcommand{\OTSetup}{\mathsf{OT.Setup}}
\newcommand{\OTSetupq}{\mathsf{OT.Setup_{equiv}}}
\newcommand{\ROTequiv}{\mathsf{OT.EquivR}}
\newcommand{\SOTequiv}{\mathsf{OT.EquivS}}


\newcommand{\OT}{\mathsf{OT}}
\newcommand{\ijl}{{\itoj,\ell}}
\newcommand{\Trap}{\mathsf{Trap}}
\newcommand{\NMCom}{\mathsf{NMCom}}
\newcommand{\wimim}{\mathsf{wiMIM}}
\newcommand{\err}{\mathsf{err}}
\newcommand{\bad}{\mathsf{bad}}
\newcommand{\overtime}{\mathsf{overtime}}
\newcommand{\vk}{\mathrm{vk}}
\newcommand{\sk}{\mathrm{sk}}
\newcommand{\id}{\mathrm{id}}
\newcommand{\pk}{\mathrm{pk}}
\newcommand{\ck}{\mathrm{ck}}
\newcommand{\crs}{\mathrm{crs}}
\newcommand{\mpk}{\mathrm{mpk}}
\newcommand{\ct}{\mathrm{ct}}
\newcommand{\td}{\mathrm{td}}
\newcommand{\NP}{\mathrm{NP}}
\newcommand{\CR}{{\langle C,R \rangle}}
\newcommand{\view}{{\mathrm{view}}}
\newcommand{\itoj}{{i \rightarrow j}}
\newcommand{\jtoi}{{j \rightarrow i}}
\newcommand{\jtoistar}{{j \rightarrow i^\star}}
\newcommand{\lang}{{\mathcal{L}}}
\newcommand{\rel}{{\mathcal{R}}}
\newcommand{\Time}{{\mathcal{T}}}
\newcommand*{\expt}[1]{{\mathrm E}\left[\,{#1}\,\right]}
\newcommand{\set}[1]{\{#1\}}



% IC

\newcommand{\iC}{{\mathrm C}}
\newcommand{\iF}{{\mathrm F}}
\newcommand{\iV}{{\mathrm V}}
\newcommand{\gval}{g}
\newcommand{\aval}{\mathsf{\alpha}}
\newcommand{\icnext}{{\mathrm C}}
\newcommand{\ifnext}{{\mathrm F}}
%\newcommand{\icnext}{{\mathrm{i}C.\mathsf{Next}}}
\newcommand{\ic}{\mathit{IC}}
\newcommand{\icclass}{\mathcal{C}}
\newcommand{\icc}{C}
\newcommand{\icout}{o}
\newcommand{\icl}{\ell}
\newcommand{\iclen}{L}
\newcommand{\icq}{\mathsf{q}}
\newcommand{\ica}{\mathsf{a}}
\newcommand{\icst}{\mathsf{st}}
%\newcommand{\icdistr}{\mathit{IC}\mathcal{D}}
\newcommand{\icdistr}{\mathrm{i}\mathcal{D}}
\newcommand{\icind}{\mathrm{id}}
\newcommand{\icaux}{\mathsf{aux}}
\newcommand{\iceval}{\mathsf{IC.Eval}}
\newcommand{\qwa}{\mathit{QWA}}
\newcommand{\trans}{\mathsf{trans}}

% GIC

\newcommand{\gic}{\mathsf{GiC}}
\newcommand{\gicgarble}{\mathsf{GiC.Garble}}
\newcommand{\giceval}{\mathsf{GiC.Eval}}
\newcommand{\gicsim}{\mathsf{GiC.Sim}}
\newcommand{\giccir}{\widehat{\iC}}

\newcommand{\gifcir}{\widehat{\iF}}
\newcommand{\givcir}{\widehat{\iV}}

\newcommand{\gicsimcir}{\widetilde{\iC}}
\newcommand{\gicst}{\hat{\icst}}

\newcommand{\icaugnext}{{\mathrm{i}C.\mathsf{AugNext}}}
\newcommand{\gcicaugnext}{\widehat{\mathrm{i}C.\mathsf{AugNext}}{}}


% OT



\newcommand{\eotsetup}{\mathsf{eOT.Setup}}
\newcommand{\eotsendone}{\mathsf{eOT.Msg}_1}
\newcommand{\eotsendtwo}{\mathsf{eOT.Msg}_2}

\newcommand{\eootsendone}{\mathsf{eOT.MsgO}_1}
\newcommand{\eootsendtwo}{\mathsf{eOT.MsgO}_2}
\newcommand{\obothsend}{\mathsf{qOT.MsgO}}
\newcommand{\iobothsend}{\mathsf{qOT.MsgO^{-1}}}


\newcommand{\qootsendtwo}{\mathsf{qOT.MsgO}_2}
\newcommand{\qot}{\mathsf{qOT}}
\newcommand{\eot}{\mathsf{eOT}}
\newcommand{\eotsetupb}{\mathsf{eOT.Setup_{extr}}}

\newcommand{\qotsetup}{\mathsf{qOT.Setup}}
\newcommand{\qotsetupb}{\mathsf{qOT.Setup_{ext}}}
\newcommand{\qotsetupe}{\mathsf{qOT.Setup_{equiv}}}
\newcommand{\qotsendone}{\mathsf{qOT.Msg}_1}
\newcommand{\qotsendtwo}{\mathsf{qOT.Msg}_2}
\newcommand{\qotequivS}{\mathsf{qOT.EquivS}}
\newcommand{\qootsendone}{\mathsf{qOT.MsgO}_1}
\newcommand{\qiotsendone}{\mathsf{qOT.MsgO}_1^{-1}}
\newcommand{\qotextr}{\mathsf{qOT.Extr}}




\newcommand{\eotextr}{\mathsf{eOT.Extr}}
\newcommand{\qcom}{\mathsf{C}}
\newcommand{\qcomcom}{\mathsf{C.Com}}
\newcommand{\qcomopen}{\mathsf{C.Open}}
\newcommand{\qcomequiv}{\mathsf{C.Equiv}}
\newcommand{\qcomsetup}{\mathsf{C.Setup}}
\newcommand{\qcomsim}{\mathsf{C.Sim}}
\newcommand{\qqcomsetup}{\mathsf{C.Setup_{equiv}}}

\newcommand{\ot}{\mathsf{OT}}
\newcommand{\otsendone}{\mathsf{OT.Rec}}
\newcommand{\otsendtwo}{\mathsf{OT.Send}}
\newcommand{\ootsendone}{\mathsf{OT.MsgO}_1}
\newcommand{\ootsendtwo}{\mathsf{OT.MsgO}_2}
\newcommand{\iotsendone}{\mathsf{OT.MsgO}_1^{-1}}
\newcommand{\iotsendtwo}{\mathsf{OT.MsgO}_2^{-1}}
\newcommand{\otextr}{\mathsf{OT.Extr}}
\newcommand{\otoutput}{\mathsf{OT.Output}}
\newcommand{\qotoutput}{\mathsf{qOT.Output}}
\newcommand{\len}{\lambda}
\newcommand{\eotoutput}{\mathsf{eOT.Output}}
\newcommand{\otflowone}{\mu^1}
\newcommand{\otflowtwo}{\mu^2}

\newcommand{\otflowonet}{\mu}
\newcommand{\otflowtwot}{\nu}




\newcommand{\otmsg}{x}
\newcommand{\otmsglen}{k}
\newcommand{\otrandone}{\rho}
\newcommand{\otrandonelen}{\tau}
\newcommand{\otsel}{\sigma}

\newcommand{\NC}{\mathsf{NC}}
\newcommand{\NCEnc}{\mathsf{NC.Enc}}
\newcommand{\NCDec}{\mathsf{NC.Dec}}
\newcommand{\NCSim}{\mathsf{NC.Sim}}
\newcommand{\NCGen}{\mathsf{NC.Gen}}


\newcommand{\keys}{\mathsf{keys}}

% interactive

\newcommand{\otsender}{\mathsf{OT.S}}
\newcommand{\otreceiver}{\mathsf{OT.R}}

% hybrids
\newcommand{\distr}{\mathcal{D}}
\newcommand{\hybrid}{\mathcal{H}}
\newcommand{\hyb}[1]{\mathcal{H}_{#1}}

%%%% Rachel's and Fabrice's note %%%%%
\iftoggle{showtodo}{
  \definecolor{darkgreen}{rgb}{0, 0.5, 0.0}
\newcommand{\Rnote}[1]{{\color{blue}[\textbf{Rachel's Note}: #1]}}
\newcommand{\Fnote}[1]{{\color{red}[\textbf{Fabrice's Note}: #1]}}
\newcommand{\anti}[1]{{\color{magenta}[\textbf{Antigoni's Note}: #1]}}
\newcommand{\Mnote}[1]{{\color{darkgreen}[\textbf{Muthu's Note}: #1]}}
}{
\newcommand{\Rnote}[1]{}
\newcommand{\Fnote}[1]{}
\newcommand{\Anote}[1]{}
\newcommand{\Mnote}[1]{}
}

\newcommand{\smallset}[1]{{\{#1\}}}
\newcommand{\bigset}[1]{{\left\{#1\right\}}}
\newcommand{\iO}{{\mathrm{i}O}}
\renewcommand{\subparagraph}[1]{\smallskip\noindent\ul{\textsc{#1}}}

\newcommand{\pprotocol}[5]{
%  \vspace{-.2cm}
  {\begin{figure}[#4]
      \centering
        \fbox{ \hbox{
            \begin{minipage}{.92\linewidth}
              \begin{center}
                \textbf{#1}
              \end{center}
              #5
            \end{minipage}
          } }
        \caption{\label{#3} #2}
      \vspace{-3ex}
    \end{figure}
  }  
%  \vspace{-.2cm}
 }


\newcommand{\tmpcnext}{{\widetilde{\mathsf{Next}}}}
\newcommand{\augnext}{{{\mathsf{AugNext}}}}
\newcommand{\aux}{{{\mathsf{aux}}}}
\newcommand{\out}{{{\mathsf{out}}}}
\newcommand{\CRS}{{{\mathsf{CRS}}}}
\renewcommand{\st}{{{{st}}}}
\newcommand{\haugnext}{{\widehat{\mathsf{AugNext}}}}
\newcommand{\haugnextsup}[1]{{\widehat{\mathsf{AugNext}}}{}^{#1}}
\newcommand{\tmpcview}{{\widetilde{\mathsf{View}}}}
\newcommand{\tmpcoutput}{{\widetilde{\mathsf{Output}}}}
\newcommand{\tmpcrand}{{\tilde r}}
\newcommand{\tmsg}{\widetilde{m}}
\newcommand{\tmpc}{{\widetilde\Pi}}
\newcommand{\tvecmsg}{\widetilde{\vec m}}
\newcommand{\tmpcvecrand}{\widetilde{\vec r}}

\newcommand{\istar}{{i^{\star}}}
\newcommand{\iii}{i} % replace \istar almost everywhere as the star is not really useful
\newcommand{\msgsize}{{\gamma}}
\renewcommand{\vec}[1]{{\bar{#1}}}
\newcommand{\nC}{{\text{n}C}}
\newcommand{\UnderlineText}[2][red]{\setulcolor{#1}\ul{#2}}
\newcommand{\cul}[1]{\UnderlineText[red]{#1}}
\newcommand{\hP}{\widehat{P}{}}
\newcommand{\hU}{\widehat{U}{}}
\newcommand{\hV}{\widehat{V}{}}
\newcommand{\bc}{c'}
\newcommand{\pot}{\mathrm{ot}}
\newcommand{\pOT}{\mathrm{p}\mathsf{OT}}
\newcommand{\decom}{\mathrm{decom}}
% \newcommand{\bkey}{\mathsf{label}}
\newcommand{\bkey}{\tgckey}



% Add some space around enumerate nested inside description (default: almost no spacing)
\setlist[enumerate]{topsep=.5\baselineskip}

\newcommand{\roundstar}{{\round^{\star}}}
\newcommand{\jstar}{{j^{\star}}}
\newcommand{\kstar}{{k^{\star}}}



\newtheorem{thm}{Theorem}[section]      % A counter for Theorems etc
\newcommand{\BT}{\begin{thm}}   \newcommand{\ET}{\end{thm}}
\newtheorem{dfn}[thm]{Definition}      %
\newcommand{\BD}{\begin{dfn}}   \newcommand{\ED}{\end{dfn}}
\newtheorem{corr}[thm]{Corollary}      %
\newcommand{\BCR}{\begin{corr}} \newcommand{\ECR}{\end{corr}}
%---
\newtheorem{Ithm}{Theorem}[section]     % A counter for Theorems in Intro
\newcommand{\BIT}{\begin{Ithm}}   \newcommand{\EIT}{\end{Ithm}}
%---
\newtheorem{lem}{Lemma}[section]  % A counter for Lemmas etc
\newcommand{\BL}{\begin{lem}}   \newcommand{\EL}{\end{lem}}
\newtheorem{prop}[lem]{Proposition}
\newcommand{\BP}{\begin{prop}}   \newcommand{\EP}{\end{prop}}
\newtheorem{clm}[lem]{Claim}            %
\newcommand{\BCM}{\begin{clm}}   \newcommand{\ECM}{\end{clm}}
\newtheorem{fact}[lem]{Fact}            %
\newcommand{\BF}{\begin{fact}}   \newcommand{\EF}{\end{fact}}
\newtheorem{prot}{Protocol}      % A counter for Protocols
\newcommand{\BPR}{\begin{prot}}   \newcommand{\EPR}{\end{prot}}
\newenvironment{cproof}{\noindent{\bf Proof:~~}}{\hfill $\Box$}
\newcommand{\BCPF}{\begin{cproof}} \newcommand {\ECPF}{\end{cproof}}
\newcommand{\Qed}{\hfill $\Box$}

%\newtheorem{remark}[lem]{Remark}            %
\newcommand{\BR}{\begin{remark}}   \newcommand{\ER}{\end{remark}}

\newcommand{\BDE}{\begin{description}}
\newcommand{\EDE}{\end{description}}
\newcommand{\BE}{\begin{enumerate}}
\newcommand{\EE}{\end{enumerate}}
\newcommand{\BI}{\begin{itemize}}
\newcommand{\EI}{\end{itemize}}
\newcommand{\BEQ}{\begin{eqnarray*}}
\newcommand{\EEQ}{\end{eqnarray*}}
\def\blackslug
{\hbox{\hskip 1pt\vrule width 8pt height 8pt depth 1.5pt\hskip 1pt}}
\def\qed{\quad\blackslug\lower 8.5pt\null\par}
\def\qqed{$\Box$}

\newcommand{\ccP}{{\cal P}}
\newcommand{\cA}{{\cal A}}
\newcommand{\cS}{{\cal S}}
\newcommand{\func}{\mathcal{F}}
\newcommand{\R}{\mathbb{R}}
\newcommand{\idind}{\equiv}
\newcommand{\compind}{\approx_{\mathrm{c}}}
\newcommand{\statind}{\approx_{\mathrm{s}}}

\newcommand{\nizk}{\ensuremath{\mathsf{NIZK}}}
\newcommand{\nizkSetup}{\mathsf{NIZK.Setup}}
% \newcommand{\nizkSetupSound}{\mathsf{NIZK.Setup_{sound}}}
% \newcommand{\nizkSetupEquiv}{\mathsf{NIZK.Setup_{equiv}}}
\newcommand{\nizkProve}{\mathsf{NIZK.Prove}}
\newcommand{\nizkVer}{\mathsf{NIZK.Ver}}
\newcommand{\nizkSim}{\mathsf{NIZK.Sim}}
\newcommand{\nizkEquiv}{\mathsf{NIZK.Equiv}}
\newcommand{\nizkcrs}{\mathrm{crs}}
\newcommand{\nizktrap}{\mathrm{trap}}
\newcommand{\nizkst}{\st^{\mathsf{e}}}
\newcommand{\nizkproof}{\pi}
\newcommand{\nizkrand}{\rho}

\newcommand{\word}{\mathsf{x}}
\newcommand{\wit}{\mathsf{w}}

\newcommand{\hbnizk}{\ensuremath{\mathsf{HB}}}
\newcommand{\hbnizkProve}{\mathsf{HB.Prove}}
\newcommand{\hbnizkVer}{\mathsf{HB.Ver}}
\newcommand{\hbnizkEquiv}{\mathsf{HB.Equiv}}
\newcommand{\hbnizkst}{\st^{\mathsf{e}}}

\newcommand{\ltdpdom}{\mathcal{D}}
\newcommand{\ltdprange}{\mathcal{R}}
\newcommand{\ltdpinjgen}{\mathsf{LTDP.InjGen}}
\newcommand{\ltdplossygen}{\mathsf{LTDP.LossyGen}}
\newcommand{\ltdprangesample}{\mathsf{LTDP.S}_{\ltdprange}}
\newcommand{\ltdprangesampleinv}{\mathsf{LTDP.S}_{\ltdprange}^{-1}}


\newcommand{\keygen}{\mathsf{KG}}
\newcommand{\enc}{\mathsf{Enc}}
\newcommand{\enco}{\mathsf{EncO}}
\newcommand{\encoinv}{\mathsf{EncO}^{-1}}
\newcommand{\dec}{\mathsf{Dec}}

\newcommand{\hk}{\mathsf{hk}}
\newcommand{\hp}{\mathsf{hp}}


\newcommand{\HashKG}{\mathsf{HashKG}}
\newcommand{\ProjKGO}{\mathsf{ProjKGO}}
\newcommand{\ProjKGOinv}{\mathsf{ProjKGO}^{-1}}
\newcommand{\Hash}{\mathsf{Hash}}
\newcommand{\ProjHash}{\mathsf{ProjHash}}

\newcommand{\HH}{\mathsf{H}}
\newcommand{\projH}{\mathsf{pH}}

\newcommand{\Jacob}{\mathbb{J}}
\newcommand{\QR}{\mathbb{QR}}
\newcommand{\SQR}{\mathbb{SQR}}

%%% Local Variables: 
%%% mode: latex
%%% TeX-master: "../main"
%%% End: 

%\addtolength{\parskip}{0.75ex}

% Theorem definition should be done after loading cleverref

\nottoggle{llncs}{
  \newtheorem{theorem}{Theorem}[section]
  \newtheorem{lemma}[theorem]{Lemma}
  \newtheorem{claim}[theorem]{Claim}
  \newtheorem{corollary}[theorem]{Corollary}
  \newtheorem{assumption}[theorem]{Assumption}
  \theoremstyle{definition}
  \newtheorem{mydefinition}[theorem]{Definition}
  \theoremstyle{remark}
  \newtheorem{remark}[theorem]{Remark}
}{
    % Force roman in definition
  \makeatletter
  \spnewtheorem{mydefinition}[theorem]{Definition}{\bfseries}{\rmfamily}
  \def\theHmydefinition{\theHtheorem}
  \makeatother
  \Crefname{mydefinition}{Definition}{Definitions}
  \crefname{mydefinition}{Definition}{Definitions}
}

\usepackage{multirow}

%
% Title and other informations
%

\title{Adaptive 2-Round Secure Multiparty Computation}

\date{}

\nottoggle{anonymous}{
  \iftoggle{llncs}{  
    \author{Fabrice Benhamouda\inst{1} \and Huijia Lin\inst{2} \and Antigoni Polychroniadou\inst{3} \and Muthuramakrishnan Venkitasubramaniam\inst{4}} 
    \institute{
      IBM Research, Yorktown Heights, US
      \and
      University of California, Santa Barbara, US
      \and
      Cornell Tech and University of Rochester, US
      \and
      University of Rochester, US
    }
  }{
    % \author{Fabrice Benhamouda\thanks{ \texttt{fabrice.benhamouda@normalesup.org},
    %     IBM Research, Yorktown Heights, US}
    %   \and Huijia Lin\thanks{ \texttt{rachel.lin@cs.ucsb.edu},
    %     University of California, Santa Barbara, US}
    % } 
  }
  \hypersetup{
    pdfauthor = {Fabrice Benhamouda, Huijia Lin, Antigoni Polychroniadou, and Muthuramakrishnan Venkitasubramaniam}
  }
  
  \iftoggle{full}{
    % \publication{}
    % \copyright{}
  }{
  }
}{
  \author{}
  % \authorrunning{}
  \institute{}
}

\makeatletter
\AtBeginDocument{
  \hypersetup{
    pdftitle = {\@title},
  }
}
\makeatother

\begin{document}
\maketitle

\nottoggle{llncs}{
  \patchcmd{\abstract}{\small}{}{}{}
  % use normal size abstract
  % https://tex.stackexchange.com/a/133165/34384
}{
}

\begin{abstract}
The round complexity of secure multiparty computation is a fundamental question in modern cryptography. In this work, we identify the round complexity of multiparty computation against adaptive corruptions. 

 assuming an ideal oblivious transfer (OT) functionality assuming adaptive corruptions. More precisely, we design a two-round adaptively secure MPC protocol in the OT-hybrid. We build upon the recent work of Benhamouda and Lin and Garg and Srinivasan who design a two-round secure MPC protocol in the OT-hybrid against static corruptions. 
%\addvspace{\baselineskip}
%\noindent
%\textbf{Keywords.} 
\end{abstract}

\Fnote{We use CryptoBib (\url{https://cryptobib.di.ens.fr}) for the bibliography. Most references are already inside CryptoBib. See \url{https://cryptobib.di.ens.fr/manual\#labeling-conventions}. If you need to add a reference, please add it in the file add.bib.}

\Fnote{We use cleveref. To make a reference to the introduction for example, just write \texttt{\textbackslash{}ref\{sec:intro\}} and you get \cref{sec:intro}}

\Fnote{We can write notes using Anote, Fnote, Mnote, Rnote for Antigoni, Fabrice, Muthu, and Rachel respectively}

\Rnote{}
\anti{}
\Mnote{}

\Fnote{We use the environment ``mydefinition'' instead of ``definition'' to get roman fonts in llncs. I tried to patch smartly the environment of llncs instead of creating a new one but I did not manage without modifying llncs.cls (and so I think it would have been removed from the camera-ready version).}

\nottoggle{llncs}{
  \clearpage
  \tableofcontents
  \clearpage
}


\section{Introduction}
\label{sec:intro}

Some thoughts on the intro. 

MPC. 

static security vs adaptive security. we focus on the setting where every one can be corrupted. 

Advances in adaptive security has been slow. O(1)-round static security in the plain model was known in the 80's, but O(1)-round adaptive security starts from 2009 in the CRS model and 2XXX (Garg-Sahai, Muthu) in the plain model. Recent advances has drove down the exact complexity of static secure MPC [....], leading to an array of 2-round MPC protocols in the CRS model from various assumptions.... [....] , even from the necessary assumption of malicious OT in the CRS model[GargSrin]. In contrast, 2-round adaptive security is only known under subexponentially secure IO. Thus we ask: 

Can we have 2-round adaptively secure MPC from standard assumptions? 

In fact, the feasibility of 2-round adaptive security remains open even in the restricted 2PC setting, and even for the specific 2PC functionality of OT. . what is known.... 

Can we have 2-round adaptively secure OT and 2PC from standard assumptions? 

\anti{
 (1st question:) it was open on how to do optimal round adaptive MPC from standard assumptions (2nd question) even in the 2 party case we do not know optimal round adaptive from standard assumptions ( STOC adaptive garbling paper did only semi-honest 2 round) (3rd question) so even optimal round adaptive OT is not known from standard assumptions?}
 
\subsection{Our Results}
 
mention the theorems only in the semi-malicious case; both theorems for MPC and OT, 
 
add a line at the end saying that we can get malicious applying UC adaptive NIZK. 
 
\subsection{Our Techniques}

In~\cite{STOC:CLOS02}, Canetti, Lindell, Ostrovsky, and Sahai show how to construct a 2-round adaptive semi-honest OT, from augmented non-committing encryption (augmented meaning that it is possible to obliviously sample a public key).
The construction is simple and elegant: the receiver generates two non-committing encryption public keys. The one corresponding to its input $\otsel$ is generated correctly with the associated secret key, while the other one is generated obliviously.
The sender then encrypts its first input $\otmsg_0$ under the first key and its second input $\otmsg_1$ under the second key.
The receiver can then decrypts $\otmsg_\otsel$.
Unfortunately, this construction does not achieve security against a semi-malicious adversary, as such an adversary can use the same strategy as a simulator simulating an honest receiver, namely: it can generate two valid public keys and ``claim'' that one is obliviously generated.


Our construction of 2-round adaptive semi-malicious OT uses different methods to avoid this issue.
Let us now give a high-level overview of our construction.
%We consider a sender with two input bits $\otmsg_0,\otmsg_1$ and a receiver with an input selection bit $\otsel$.
%
Compared to 2-round static semi-malicious OT, adaptive OT needs to provide a way to equivocate the sender and the receiver flows at the end. There are three kinds of equivocation:
\begin{enumerate}
\item equivocation of the receiver at the end;
\item equivocation of the sender at the end, while the receiver was malicious at the beginning;
\item equivocation of the sender at the end, while the receiver was honest at the beginning.
\end{enumerate}

We first remark that the third kind of equivocation is easy to achieve by making the receiver sends a public key for a non-committing encryption scheme in the first flow, and the sender encrypting its flow under this key.
This way, if the sender is corrupted at the end, we can just equivocate its flow to contain an honestly-generated sender-flow for the learned bits $\otmsg_0,\otmsg_1$.

To build a scheme satisfying the first two equivocation properties, we follow these steps.

\subparagraph{Step 1: 2-round static OT with oblivious samplability and extraction.}
We start from a 2-round (static) semi-malicious OT with the following additional properties:
\begin{enumerate}
\item the sender flow can be sampled obliviously;
\item the receiver flow can be sampled obliviously;
\item the input $\otsel$ can be extracted from a malicious receiver.
\end{enumerate}
\Fnote{This is not exactly true.}
Oblivious sampling is a much weaker property than equivocation.
A variant of the oblivious transfer from Naor and Pinkas~\cite{SODA:NaoPin01} (later generalized by Halevi and Kalai~\cite{JC:HalKal12}) based on the Decisional Diffie-Hellman (DDH) assumption, where some elements of the first flow are generated by a trusted party and written in the CRS, instead of being generated by the sender (in order to allow extraction of the selection bit of the receiver) can be used.
More generally, we show how to construct such OT from encryption schemes with ciphertext oblivious sampling and smooth projective function (or hash proof systems, a primitive introduced by Cramer and Shoup in~\cite{EC:CraSho02}) with some obliviousness property.
In particular, this generic construction can also be instantiated from the Quadratic Residuosity (QR) assumption.

\subparagraph{Step 2: adding equivocation of the sender.}
We show a generic transformation from any 2-round static OT with oblivious samplability and extraction to a similar OT with in addition equivocation of the sender (against a malicious sender).

The first idea to achieve this property is to run the original protocol twice in parallel.
The receiver would use the same input $\otsel$ for both executions, while the sender would generates one of its flow normally with its inputs $(\otmsg_0,\otmsg_1)$ and one obliviously.
To allow the receiver to know which flow was obliviously generated, we can suppose that our new protocol works with sender bit inputs, but the original protocol works with sender string inputs, so that an obliviously sampled sender flow corresponds to a random string which is unlikely to be equal to a bit. Thus the receiver can know which flow to use.
The simulator of an honest sender against a maliciours receiver would first extract $\otsel$ from the flows of the receiver, get $\otmsg_\otsel$ and then generate both sender flows non-obliviously using $\otmsg_\otsel$ and $\otmsg_{1-\otsel}=0,1$.


Unfortunately, this idea does not work for at least three reasons:
\begin{enumerate}
\item The simulated sender is not indistinguishable from a real sender, as even an honest sender would get from both these flows, the same output $\otmsg_\otsel$, while for a real sender, one of these outputs is uniformly random.
\item A semi-malicious sender can generate the first flow for $(0,0)$ and the send flow for $(1,1)$, which would break correctness (an honest receiver would not know which value to output).
\item The receiver can use a different input $\otsel$ for each of its flows, in which case extraction will fail.
\end{enumerate}

To solve the first issue, the OT is repeated four times, still with the same receiver input $\otsel$. The two first times are used for $\otmsg_0$: the sender generates one of the two corresponding flows obliviously, and generate the other for $(\otmsg_0,s_0)$, with $s_0$ a random string.
Similarly, the two second times are used for $\otmsg_1$.

To solve the second issue, we make the sender commit to $\otmsg_0$ and $\otmsg_1$ with an equivocal commitment scheme: $\com_0 = \Com(\otmsg_0;\rho_0)$ and $\com_1 = \Com(\otmsg_1;\rho_1)$. Then, it uses the following inputs $\otmsg_0 \| \rho_0$ and $\otmsg_1 \| \rho_1$ to generate its flows, instead of $\otmsg_0$ and $\otmsg_1$.
On the one hand, a malicious sender cannot open an equivocal commitment to both $0$ and $1$, so that it cannot misbehave as in the second issue.
On the other hand, a simulator can open the equivocal commitment both to $0$ and $1$ and its simulation strategy still works.


To solve the third issue, we change the protocol so that the receiver only sends a single flow instead of four, which is then re-used four times in parallel by the sender.

\subparagraph{Step 3: adding equivocation of the receiver.}
The final step is to add equivocation of the receiver.
For that, we combine the OT with garbled circuits using techniques reminiscent of~\cite{C:CDGGMP17,C:DotGar17,EC:BenLin18,EC:GarSri18}.

The first idea is similar to the idea to construct non-committing encryption in~\cite{AC:CDMW09}.
We run in parallel $4 \secpar$ OT.
The sender generates $4 \secpar$ first flows (with $\secpar$ the security parameter): a random $\secpar$-size subset of them is generated correctly for its input $\otsel$, while the $3\secpar$ others are obliviously sampled.
The receiver similarly selects $\secpar$ executions and generate its flow correctly for them (for its inputs $(\otmsg_0,\otmsg_1)$).
It samples obliviously the $3\secpar$ other sender flows.
This allows a simulator of an honest receiver to generate $\secpar$ flows for $\otsel=0$, $\secpar$ flows for $\otsel=1$, and $2\secpar$ flows obliviously; and so later to equivocate its randomness to match either $\otsel = 0$ or $\otsel = 1$.
\Fnote{$2$ flows from receiver should be sufficient instead of this.}

%%% Local Variables:
%%% mode: latex
%%% TeX-master: "../main"
%%% End:

\section{Overview}
\label{sec:overview}

%%% Local Variables:
%%% mode: latex
%%% TeX-master: "../main"
%%% End:

% !TEX root =../main.tex
\section{Preliminaries}
\label{sec:preliminaries}



\subsection{Notation}
Throuhgout the paper $\secpar \in \N$ will denote the security parameter. We say that a function $f : \N \rightarrow \R$ is negligible if $\forall c~ \exists~  n_c$ such that if $n>n_c$ then $f(n)< n^{-c}$. We will use $\negl(\cdot)$ to denote an unspecified negligible function. We often use $[n]$ to denote the set $\{1,...,n\}$. The concatenation of $a$ with $b$ is denoted by $a||b$. Moreover, we use $d \leftarrow \dist$ to denote the process
of sampling $d$ from the distribution $\dist$ or, if $\dist$ is a set, a uniform choice from it.  If $\dist_1$ and $\dist_2$ are two distributions, then we denote that they are statistically
close by $\dist_1 \statind \dist_2$; we denote that they are computationally indistinguishable
by $\dist_1 \compind  \dist_2$; and we denote that they are identical by $ \dist_1 \idind \dist_2$.


We recall the notion of polynomial-size circuit classes and families, together with the notion of statistical and computational indistinguishability in \cref{A:preliminaries}.

For the sake of simplicity, we suppose that all circuits in a circuit
class have the same input and output lengths. This can be achieved
without loss of generality using appropriate paddings.  We recall that
for any $\cirsize$-size circuit class $\circlass =
{\{\circlass_\secpar\}}_{\secpar \in \N}$, there exists a universal
$\poly(\cirsize)$-size circuit family
${\{\univcirc_\secpar\}}_{\secpar \in \N}$ such that for any $\secpar
\in \N$, any circuit $\cir \in \circlass_\secpar$ with input and
output lengths $\cirinlen, \ciroutlen$, and any input $\cirin \in
\bits^\cirinlen$, $\univcirc_\secpar(\cir, \cirin) =
\cir(\cirin)$.


\begin{mydefinition}[Equivocal Garbled Circuit]
  Let $\circlass = {\{\circlass_\secpar\}}_{\secpar \in \N}$ be a
  $\poly$-size circuit class with input and output lengths $\cirinlen$
  and $\ciroutlen$.  A \emph{garbled circuit} scheme $\gc$ for
  $\circlass$ % with
%   polynomial universal circuits 
  is a tuple of four
  polynomial-time algorithms $\gc=(\gcgen,\gcgarble,\gceval,\allowbreak\gcsim)$: 
  \begin{description}
  \item[Input Labels Generation:] $\gckey \getsr \gcgen(1^\secpar)$
%     1^\cirinlen)$
    generates input labels $\{\gckey^i\}_{i \in
      [\cirinlen] }$ (with $\gckey^i[b] \in \bits^\kappa$ being
    the input label corresponding to the value $b$ of the $i$-th input
    wire) for the security parameter $\secpar$, input length
    $\cirinlen$, and input label length $\kappa$; % = \cirinlen(\secpar)$;
  \item[Circuit Garbling:] $\gccir \getsr \gcgarble(\gckey, \cir;\sigma)$
     garbles the circuit $\cir \in \circlass_\secpar$
    into $\gccir$;
  \item[Evaluation:] $\cirout = \gceval(\gccir, \gckey[x])$  evaluates the
    garbled circuit $\gcgarble$ using input labels
    ${\gckey[{x_i}]}$ for input some input $x=(x_1,\ldots,x_n)$
    and returns the output $\cirout \in \bits^\ciroutlen$; %   with
%      $\ciroutlen = \ciroutlen(\secpar)$
    
  \item[Simulation:] $(\gckey, \gcsimcir,\state) \getsr \gcsim(1^\secpar,
    \cirout)$
    % 1^\cirsize, 1^\cirinlen, \cirout)$
     simulates  input labels $\gckey$, a garbled circuit $\gcsimcir$ and state $\state$ for the security parameter
    $\secpar$ on the output $\cirout \in \bits^\ciroutlen$;
  \end{description}
  satisfying the following security properties:
  \begin{description}
  
  
  
   \item[Equivocation:] $( \gckey,\sigma) \getsr \gcequiv( 
    \gccir,\cirin, \state)$ such that $\gccir = \gcgarble(\gckey, \cir;\sigma)$, given
$\cirin$, the simulator generates (inactive) labels and fake randomness $\sigma$ of the garbling that makes $\gccir,\gckey$ look like a real garbling
of $\cir, \cirin$.\anti{update the rest according to this syntax}
   
  \end{description}
  satisfying the following security properties:
  \begin{description}
  
  
  
  
  \item[Correctness:] For any security parameter $\secpar \in \N$, for any circuit $\cir \in \circlass_\secpar$, for any input $\cirin \in \bits^{\cirinlen}$, for any $\gckey$ in the image of $\gcgen(1^\secpar)$ and any $\gccir$ in the image of $\gcgarble(\gckey, \cir)$:
    \[ \gceval(\gccir, {\{\gckey[\cirin_i]\}}_{i \in [\cirinlen]}) = \cir(\cirin)\enspace. \]
  \item[Simulatability:] The following two distributions are computationally indistinguishable:\anti{add equivocation property}
    \begin{align*}
      \bigg\{
      ({\{\gckey[{\cirin}]\}}, \gccir)
      \ &: \
          \begin{array}{l}
            \gckey \getsr \gcgen(1^\secpar); \\
            \gccir \getsr \gcgarble(\gckey, \cir)
          \end{array}
      {\bigg\}}_{\secpar, \cir \in \circlass_\secpar, \cirin \in \bits^{\cirinlen}} \enspace, \\
      \big\{
      (\gckey, \gccir)
      \ &: \
          \begin{array}{l}
            (\gckey,\gccir) \getsr \gcsim(1^\secpar, \cir(\cirin))
          \end{array}
          {\big\}}_{\secpar, \cir \in \circlass_\secpar, \cirin \in
          \bits^{\cirinlen}} \enspace.
    \end{align*}
  \end{description}
\end{mydefinition}

We recall that garbled circuit schemes can be constructed from one-way
functions.



\anti{remove this paragraph}
For the sake of simplicity, if $\cirin \in \bits^\cirinlen$ and $\gckey = {\{\gckey[i, b]\}}_{i \in
      [\cirinlen], b \in \bits}$, we define
$\gckey[\cirin] = \smallset{\gckey[i,\cirin_i]}_{i \in
  [\cirinlen]}$.
  
  \remove{
We extend this notation when the input is a tuple: for example, if $\cirin=(u,v) \in \bits^{\cirinlen_1} \times \bits^{\cirinlen_2}$, we define $\gckey[u] = \smallset{\gckey[i,u_i]}_{i \in [\cirinlen_1]}$ and $\gckey[v] = \smallset{\gckey[\cirinlen_1 + i,v_i]}_{i \in [\cirinlen_2]}$.
We also abuse notation and define $\gckey[[u]]$ (resp., $\gckey[[v]]$) to be the $2\cirinlen_1$ (resp., $2\cirinlen_2$) input labels corresponding to the input wires for $u$ and $v$: $\gckey[[u]] = \smallset{\gckey[i,b]}_{i \in [\cirinlen_1],b \in \bits}$ and $\gckey[[v]] = \smallset{\gckey[\cirinlen_1 + i,b]}_{i \in [\cirinlen_2], b \in \bits}$.
This notation is also used for $\gckey' = {\{\gckey[i]\}}_{i \in [\cirinlen]}$: $\gckey'[[u]] = \smallset{\gckey'[i]}_{i \in [\cirinlen_1]}$ and $\gckey'[[v]] = \smallset{\gckey'[\cirinlen_1 + i]}_{i \in [\cirinlen_2]}$.}


We make use of garbled circuit schemes.
A \emph{garbled circuit} scheme $\gc$ for a $\poly$-size circuit class $\circlass = {\{\circlass_\secpar\}}_{\secpar \in
  \N}$ is defined by four polynomial-time algorithms $\gc=(\gcgen,\gcgarble,\gceval,\allowbreak\gcsim)$:
\textit{i)} $\gckey \getsr \gcgen(1^\secpar)$
% 1^\cirinlen)$
generates input labels $\gckey = {\{\gckey[i, b]\}}_{i \in
  [\cirinlen], b \in \bits}$;
\textit{ii)}
$\gccir \getsr \gcgarble(\gckey, \cir)$
garbles the circuit $\cir \in \circlass_\secpar$
into $\gccir$;
\textit{iii)} $\cirout = \gceval(\gccir, \gckey')$  evaluates the
garbled circuit $\gcgarble$ using input labels
$\gckey' = {\{\gckey'[i]\}}_{i \in [n]}$ (where $\gckey'[i] \in \bits^\kappa$)
and returns the output $\cirout \in \bits^\ciroutlen$;
\textit{iv)} $(\gckey', \gcsimcir) \getsr \gcsim(1^\secpar,
    \cirout)$
     simulates  input labels $\gckey' = {\{\gckey'[i]\}}_{i \in [\cirinlen]}$ and
    a garbled circuit $\gcsimcir$ corresponding to the output $\cirout \in \bits^\ciroutlen$.
The formal definition can be found in \appref{}.
We recall that garbled circuit schemes can be constructed from one-way
functions.


%%% Local Variables:
%%% mode: latex
%%% TeX-master: "../main"
%%% End:

% !TEX root =../main.tex

\section{2-Round Semi-Malicious MPC}


\begin{mydefinition}[(Semi-Malicious) Equivocal Functional Commitment]
  \label{def:hc}
  Let $\hccirclass = {\{\hccirclass_\secpar\}}_{\secpar \in \N}$ be a
  $\poly$-size circuit class. A \emph{(semi-malicious) equivocable functional commitment} scheme $\hc$ for $\hccirclass$ % with
  % polynomial universal circuits 
  is a tuple of eight polynomial-time algorithms $\hc=(\setup, \hccom,\hcopen,\wsenc,\wsdec,\csim,\Equiv)$:
  \begin{description}
  
   \item[Setup:] $\setup$ can be viewed as a pair of PPT algorithms $(\hcsetup, \ehcsetup)$ such that the following holds:
   \BI
   \item $\ck \getsr  \hcsetup(1^\secpar)$ expects as input the unary representation of the security parameter $\secpar$ and
outputs a public parameter $\ck$.
    \item $(\ck,\teq) \getsr \ehcsetup(1^\secpar)$ outputs a public parameter $\ck$ together with a trapdoor $\teq$ (used for equivocation).
  \EI
  
  \item[Commitment:] $\hcc = \hccom(\ck, \hcmsg;\hcr)$  generates a commitment $\hcc$ of $\hcmsg \in \bits^{\poly(\secpar)}$ using random tape $\hcr \in \bits^{\poly(\secpar)}$;
  \item[Functional Opening:] $\hcd = \hcopen(\ck, \hcc, \hccir, \hcmsg, \hcr)$  derives from the commitment~$\hcc$ and the random tape $\hcr$ used to generate it, a functional decommitment $\hcd$ of~$\hcc$ to $\hcout = \hccir(\hcmsg)$ for $\hccir \in \hccirclass_\secpar$;
%  \item[Functional Verification:] $b = \hcver(\hcc, \hccir, \hcout, \hcd)$  outputs $b=1$ if $\hcd$ is a valid functional decommitment of $\hcc$ to $\hcout$ for $\hccir \in \hccirclass_\secpar$; and outputs $b=0$ otherwise;
  
  \item[Encryption:] $\wsct = \wsenc(\ck, \hcc, \hccir, \hcout,
    \wsmsg;\renc)$  encrypts messages $\wsmsg
    =\smallset{\wsmsg[i,b]}_{i \in [\ndaylen],b\in \bits}$ for $(\hccir, \hcout)$ using random tape $\renc$, into a ciphertext $\wsct$, where each
    message has the same length $|\wsmsg[i,b]| =
    \poly(\secpar)$;
  \item[Decryption:] $\wsmsg' = \wsdec(\ck, \hcc, \wsct,\hcd)$ 
    decrypts a ciphertext $\wsct$ into messages $\wsmsg' =
    \smallset{\wsmsg'[i]}_{i \in [\ndaylen]}$ using decommitment $\hcd$;
    \item[Commitment Simulation:] $\csim$ can be viewed as a pair of PPT algorithms $(\hcsim,\encsim)$ such that the following holds:
    
    \BI
    \item $(\hcc,\state^\cc)\getsr \hcsim({\ck},{\teq})$ simulates an (equivocal) commitment $\hcc$ together with $\state^\cc$;
    \item $(\wsct,\state^\ee) \getsr \encsim(\ck, \hcc, {\teq}, \rho)$ simulates an (equivocal) ciphertext $\wsct$ together with $\state^\ee$;
    \EI
    
  \item[Commitment Equivocation:] $\Equiv$ can be viewed as a triple of PPT algorithms $(\ehcsimd,\encequiv, \fequiv)$ such that the following holds. Let $({\ck},\teq)\getsr \ehcsetup(1^\secpar)$: 
 \BI
 \item $\hcd \getsr \fequiv(\ck, \hcc,\ehctrap,\hccir,\hcout)$ equivocates the commitment $\hcc$ and output a functional decommitment $\hcd$ of $\hcc$ to $\hcout$ for $\hccir \in \hccirclass_\secpar$;
 \item ${\hcr}' \getsr\ehcsimd({\ck},{\teq},\hcc,\state^\cc,\hcmsg)$ such that $\hccom({\ck},\hcmsg;\hcr')={\hcc}$;
\item ${\renc}' \getsr\encequiv({\ck},{\teq},\wsct,\state^\ee, \wsmsg,\hccir,\hcout, \hcd)$ such that $ \wsenc(\ck, \hcc, \hccir, \hcout,
    \wsmsg;\renc')=\wsct $; 


  \EI
  %  \item[Commitment Equivocation:] $\rho \getsr \ehcsimd(\hcc,\ehctrap, \hcmsg)$ equivocates the commitment $\hcc$ and outputs a string $\rho'$;
%  \item[Encryption Equivocation:] $\renc'\getsr \encequiv(\wsct,\ehctrap,\hccir,\hcout, \hcd)$ equivocates the ciphertext $\wsct$ and outputs
%a string $\renc'$. 

  
  \end{description}
  satisfying the following properties:
  \begin{description}
  \item[Correctness:] For any security parameter $\secpar \in
    \N$, for any messages $\wsmsg =\smallset{\wsmsg[i,b]}_{i,b}$, for any $\hcmsg \in \bits^{\poly(\secpar)}$, for any circuit $\hccir \in
    \hccirclass_{\secpar}$, for any $\hcr \in \bits^{\poly(\secpar)}$, it holds that if $\hcd= \hcopen(\hcc, \hccir, \hcmsg, \hcr)$ is a valid functional decommitment of $\hcc$ to $ \hccir (\hcmsg)$ then:
   \[ \Pr\Big[ \wsdec(\ck, \wsenc(\ck, \hcc,\hccir, \hcout,
    \wsmsg),\; \hcd) =  \smallset{\wsmsg[i,\hcout_i]}_{i \in [\ndaylen]} \Big] = 1 \enspace; \]

 % \item[Semi-Malicious Functional Binding:] For any polynomial-time
 %    circuit family $\advA=\smallset{\advA_\secpar}_{\secpar \in \N}$,
 %    there exists a negligible function $\negl$, such that for any $\secpar \in \N$, for any $\hcmsg \in
 %    \bits^{\hccirinlen}$, for any circuit $\hccir \in
 %    \hccirclass_{\secpar}$, for any random tape $\hcr \in \bits^\comrlen$: 
 %    \begin{multline*}
 %      \Pr\Big[
 %      \hcc = \hccom(\ck, \hcmsg;\hcr) \text{ and } \hcout \neq
 %      \hccir(\hcmsg)
 %      \ :\\
 %      \hcc = 
 %      \hccom(\ck, \hcmsg; \hcr); \ (\hcout, \hcd) \getsr
 %      \advA_\secpar(1^\secpar,\hcmsg,\hcr)
 %      \Big] \le \negl(\secpar)\enspace;      
 %    \end{multline*}
  \item[Commitment Simulatability:]
    For any polynomial-time circuit family $\advA =\smallset{\advA_\secpar}_{\secpar \in \N}$,
    there exists a negligible function $\negl$, such that for any $\secpar \in N$ and for any $\hcmsg \in
    \bits^{\hccirinlen}$:
    \begin{multline*}
      \Bigg| \Pr\bigg[
      \advA_\secpar(\state,\hcc,\hcd) = 1
      \ : \
      \begin{array}{l}
        \hcr \getsr \bits^{\poly(\secpar)}; \ \hcc=
        \hccom(1^\secpar,\hcmsg;\hcr); \\
        (\state,\hccir) \getsr \advA_\secpar(\hcc); \ \hcd = \hcopen(\hcc, \hccir, \hcmsg,
        \hcr)
      \end{array}
      \bigg] - \\
      \Pr\bigg[
      \advA_\secpar(\state,\hcc,\hcd) = 1
      \ : \
      \begin{array}{l}
        (\hcc, \ehctrap) \getsr
        \ehcsimc(1^\secpar); \\
        (\state,\hccir) \getsr \advA_\secpar(\hcc); \ 
        \hcd \getsr \fequiv(\hcc,\ehctrap,\hccir,\hccir(\hcmsg))
      \end{array}
      \bigg]
      \Bigg| \le \negl(\secpar) \enspace.
    \end{multline*}

   
    The same applies to the algorithms $(\ehcsimd,\hcsim)$ and $(\encsim,\encequiv)$. We require that for $\hcmsg, \wsmsg \in \{0,1\}$ the distribution of $\{({\ck},{\hcc},{\hcr}_\hcmsg)\}$ and $\{({\ck},{\wsct},\renc_ \wsmsg)\}$ generated via $(\ehcsimd,\hcsim)$ and $(\encsim,\encequiv)$ is computationally indistinguishable from the distribution $\{({\ck},\hcc,{\hcr}_\hcmsg)\}$ and $\{({\ck},{\wsct},\renc_ \wsmsg)\}$ where $\hcc = \hccom({\ck}, \hcmsg;\hcr_\hcmsg)$ and $ \wsenc(\ck, \hcc, \hccir, \hcout,
    \wsmsg;\renc_ \wsmsg)$, respectively. \anti{write formal definitions in the appendix}
    
    \remove{
      \item[Encryption Simulatability:]
    For any polynomial-time circuit family $\advA =\smallset{\advA_\secpar}_{\secpar \in \N}$,
    there exists a negligible function $\negl$, such that for any $\secpar \in N$ and for any $\hcmsg \in
    \bits^{\hccirinlen}$:
    \begin{multline*}
      \Bigg| \Pr\bigg[
      \advA_\secpar(\state,\wsct,\hcc,\hcd) = 1
      \ : \
      \begin{array}{l}
       \hcr \getsr \bits^\hcrlen; \ \hcc=
        \hccom(1^\secpar,\hcmsg;\hcr); \\
        \renc\getsr \bits^\hcrlen; \ \wsct = \wsenc(\hcc, \hccir, \hcout,
    \wsmsg;\renc); \\
        (\state,\hccir) \getsr \advA_\secpar(\hcc,\wsct); \ \hcd = \ehcopen(\hcc, \hccir, \hcmsg,
        \hcr)
      \end{array}
      \bigg] - \\
      \Pr\bigg[
      \advA_\secpar(\state,\wsct,\hcc,\hcd,\renc') = 1
      \ : \
      \begin{array}{l}
       (\wsct,\eehctrap) \getsr \encsim(1^\secpar,\hcc,\rho); \\
        (\state,\hccir) \getsr \advA_\secpar(\wsct,\hcc); \ 
      \renc'\getsr \encequiv(\wsct,\ehctrap,\hccir,\hccir(\hcmsg), \hcd)
      \end{array}
      \bigg]
      \Bigg| \le \negl(\secpar) \enspace.
    \end{multline*}}
    
    
  \item[Soundness Security:] For any polynomial-time
    circuit family $\advA=\smallset{\advA_\secpar}_{\secpar \in \N}$,
    there exists a negligible function $\negl$, such that for any $\secpar \in \N$, for any circuit $\hccir \in
    \hccirclass_{\secpar}$, for any $\hcmsg,\hcr \in
    \bits^{{\poly(\secpar)}}$, $\hcc=\hccom(1^\secpar, \hcmsg;\hcr) \text{ and } \hcout \neq  \hccir(\hcmsg)
    $ and equal length
    messages $ \wsmsg$ and  $\wsmsg'$: 
    
    \[
      \Pr[\advA_\secpar(\wsenc(\ck, \hccir, \hcout, \wsmsg))=1]
      -
      Pr[\advA_\secpar(\wsenc(\ck, \hccir, \hcout, \wsmsg'))=1]
      \leq
      \negl(\secpar)
    \]
    
    
  
    \item[ Indistinguishability of Public Parameters:] We require that:
\begin{equation}\nonumber
\begin{split}
\Big|\Pr\Big{[}(\ck,\mu)\leftarrow& \hcsetup(1^\secpar): \mathcal{A}(\ck)=1\Big{]}-\\ &
\Pr\Big{[}({\ck},{\mu,\nu})\leftarrow \ehcsetup(1^\secpar) :\mathcal{A}({\ck})=1\Big{]}\Big| < \negl(\secpar).
\end{split}
\end{equation}
   
  \end{description}
\end{mydefinition}

\smallskip

Note that the simulatability property implies the standard hiding
property of commitments, if each circuit class $\hccirclass_\secpar$ contains a constant circuit: Consider indeed any constant circuit $C(x) = \alpha$, the fact
that $(\hcc, \hcd)$ can be simulated from $C$ and $\alpha$ implies
that $\hcc$ hides the message committed inside. 

\smallskip

\remove{\begin{mydefinition}
  \label{def:nda-for-hc}
  Let $\hccirclass = {\{\hccirclass_\secpar\}}_{\secpar \in \N}$ be a
  $\poly$-size circuit class. Let $\hc=(\hccom,\allowbreak\hcopen,\hcver,\allowbreak\hcsim)
  $ be a \emph{functional commitment} scheme for~$\hccirclass$.
  We define the following \emph{associated non-deterministic oracle family} $\nda^\hc = \smallset{\nda^\hc_{\secpar}}_{\secpar \in \N}$:
  \[
    \nda^\hc_\secpar((\hcc, \hccir), (\hcout, \hcd)) =
    \begin{cases}
      \hcout &\text{if } \hcver(\hcc,\hccir,\hcout,\hcd) = 1 \\
      \perp &\text{otherwise;}
    \end{cases}
  \]
%   and the following \emph{associated unique-answer distribution} $\ndadistr^\hc =
%   \smallset{\ndadistr^\hc_{\secpar,\hccir,\hcmsg}}$ where $\secpar \in \N$, $\hccir \in \hccirclass_\secpar$, and $\hcmsg \in \bits^{\hccirinlen}$:
%   \begin{align*}
%     \ndadistr^\hc_{\secpar,\hccir,\hcmsg} = \bigset{
%     \begin{array}{l}
%       \hcr \getsr \bits^{\hcrlen}; \ \hcc = \hccom(1^\secpar,\hcmsg;\hcr); \\
%       \hcout = \hccir(\hcmsg); \ \hcd = \hcopen(\hcc,\hccir,\hcmsg,\hcr)
%     \end{array}
%     \ : \ ((\hcc,\hccir),\;(\hcout,\hcd),\;\ndaaux=\hcr) }\enspace.
%   \end{align*}
%   Finally, a \emph{witness selector} associated to $\hc$ is a witness selector for $(\nda^\hc,\ndadistr^\hc)$.
 \end{mydefinition}}

 \smallskip

% The unique-answer property of $\ndadistr^\hc$ for $\nda^\hc$ follows from the semi-honest functional binding property of the functional commitment $\hc$.



\subsection{Construction of 2-Round Semi-Honest MPC}
\label{sec:cons-sh-mpc}


{\small\BPR [Adaptive malicious protocol $\Pi_f$]\label{prot:semimpc} Let $\mpcfunc$ be an arbitrary $N$-party functionality.\footnote{Formal definitions of MPC protocol and $N$-party functionality are provided in \cref{A:def-MPC}.}
Protocol $\Pi_f$ relies on the following components: 
\BI
\item 


An adaptive malicious $L$-round $N$-party protocol
  $\mpc=(\mpcnext, \mpcoutput)$ for $\mpcfunc$. Without
loss of generality, we will assume that in each round $\round$ of $\mpc$, each party $P_i$
broadcasts a single message that depends on its input $\mpcin_i$, randomness $\mpcrand_i$ and on the messages $\Msg^{< \round} = \smallset{\msg_j^{\round'}}_{j \in [N], \round' < \round}$ that it
received from all parties in all previous rounds, i.e. $\msg_{j}^\round = \mpcnext_j(\mpcin_j,\mpcrand_j,\Msg^{< \round})$, where $\mpcnext$ is the next message function. In the last round $L$ of $\mpc$ each party $P_i$ locally computes the output $\mpcout_i =
  \mpcoutput_i(\mpcin_i,\mpcrand_i,\Msg)$ after receiving all the messages $\Msg = \smallset{\msg_j^\round}_{j \in [N],\round \in [\nbrounds]}$.

 \item A malicious adaptive oblivious transfer scheme $\ot = (\OTSetup,\otsendone,\otsendtwo,\\\otoutput,\ROTequiv,\SOTequiv,\otextr)$. 

 % The security parameter $\secpar$ is an implicit parameter $1^\secpar$ of $\mpcnext$ and $\mpcoutput$.
%\item A semi-malicious equivocal functional commitment scheme
%  $\hc=(\setup, \hccom,\\\hcopen,\wsenc,\wsdec,\csim,\Equiv)$ for the class of all
%  $\cirsize$-size circuits with a sufficiently large polynomial bound
%  $\cirsize$.  %We denote by $\nda^\hc$ the associated non-deterministic oracle family defined in \cref{def:nda-for-hc}.
%   We denote by $\nda^\hc$ and $\ndadistr^\hc$ the associated
%   non-deterministic oracle family and unique-answer distribution
%   defined in \cref{def:nda-for-hc}.
\item A garbled circuit scheme $\gc=(\gcgen,\gcgarble,\gceval,\gcsim)$ for the class of all
  $\cirsize$-size circuits with a sufficiently large polynomial bound
  $\cirsize$.
  
  \EI
\medskip\noindent\textsc{Input:} Parties $P_1,\ldots, P_N$ are given input $(x_1,\ldots,x_N)$, respectively. 
\BI
\item \textsc{Round 1:} 

For $\round$ from  $\nbrounds$ to $1$ each party $P_\istar$ proceeds as follows: 

%Each party $P_\istar$ generates and broadcasts $\hcc_{\istar}^\round = \hccom(\ck,
 %   (\mpcin_{\istar},\mpcrand_{\istar}); \hcr_\istar^\round)$ for $\round \in [\nbrounds]$ where $\mpcrand_{\istar}$ is the random tape for running $\mpc$. 


  \begin{enumerate}
  \item Generate input labels $\ckey_\istar^\round \getsr \gcgen(1^\secpar)$ (using a random tape derived from $\hcr$).
  \item Garble $\cir_\istar^\round = \univcirc_\secpar(\star,(\mpcin_{\istar},\mpcrand_{\istar}))$, which is the universal circuit partially evaluated on~$(\mpcin_{\istar},\mpcrand_{\istar})$: $\gccir_\istar^\round \getsr \gcgarble(\ckey_\istar^\round,\cir_\istar^\round)$ where $\mpcrand_{\istar}$ is the random tape for running $\mpc$. 

  \item For each $k \in [\cirsize]$, for each bit $b \in \bits$, broadcast OT receiver messages $\otflowonet_{\istar,k,b}^\round= \otsendone(1^\secpar,\ckey_{\istar,k}^\round[b]; \otrandone^\round_{\istar,k,b})$.  % and where the random tape $\otrandone_{k,b,j,\istar}$ is derived from $\hcr$.
  
  \end{enumerate}
%  Output $\hcc_\istar^\round = (\gccir_\istar^\round, \smallset{\otflowonet_{\istar,k,b}^\round})$.




\item \textsc{Round 2:} For $\round$ from  $\nbrounds$ to $1$ each party $P_\istar$ garbles $\iF_\istar  = \smallset{\ifnext_\istar ^\round}_{\round \in [\nbrounds]}$, defined in \cref{fig:ic-mpc}, as follows: 
\BE

\item Generate input labels $\Gckey_{\istar} ^\round,\smallset{\gckey_{\istar,j} ^\round}_{j\in[N]} \getsr \gcgen(1^\secpar)$ for $j\in[N]$. \Fnote{Maybe in the Figure of $\ifnext$, we should say to what correspond $\Gckey$ and $\gckey$.}
\item Garble the circuit $\ifnext^\round_\istar$  and broadcast $\gifcir_\istar=\smallset{\gifcir_\istar ^\round}_{\round \in [\nbrounds]}$:
  \[ \gifcir_\istar^\round \getsr \gcgarble(\Gckey_{\istar}^\round, \smallset{\gckey_{\istar,j}^\round}_{j}, \ifnext_{\istar}^\round) \enspace. \]

\EE
\item \textsc{Output phase:} Each party evaluates the garbled circuits. In particular $P_\istar$ proceeds as follows in $\nbrounds$ iterations ($\round \in [\nbrounds]$): 

\BE
\item If $\ell=1$, for $i \in [N]$, execute $(\Gckey^{2}_{i}, \smallset{\givcir_{\iii,j}^1,\otflowtwot_{\iii,j,k}^1,\hcd_{i,k}^1}_{j,k},\; \msg_i^1)= \gceval(\gifcir_i^1,\emptyset)$ for all $ j \in [N]$ (we recall that $\Gckey^{1}_{i}=\emptyset$ and $\smallset{\gckey^1_{i,j}}_j = \emptyset$).\Fnote{In $\smallset{\givcir_{\iii,j}^1,\otflowtwot_{\iii,j,k}^1,\hcd_{i,k}^1}_{j,k}$, the last element does not depend on $j$ and the first one does not depend on $k$, so this notation is slightly abusive. But we can keep it for conciseness if you prefer. }

 
\item For every $1<\round \le L$, for $i \in [N]$, for each $j \in [N]$, proceed as follows for the circuit $G_j^{{\round-1}}(\star,\star) = \mpcnext_j(\star,\star, \Msg^{< \round-1})$ where $\gval_{j,k}^{\round-1}={[G_j^{\round-1}]}_k$: \Fnote{maybe sligthly better to add braces around square brackets so that the indice was ok: ${[G_j^{\round-1}]}_k$ is better than $[G_j^{\round-1}]_k$}
    
  
  \begin{align*}
    \forall k \in [S], \; \smallset{ \vkey_{i,j,k'}^{\round-1}[\gval_{j,k}^{\round-1}]}_{k'}
    &= \otoutput(\otflowtwot_{\iii,j,k}^{\round-1},\ckey_{j,k}^{\round-1}[\gval_{j,k}^{\round-1}], \hcd_{j,k}^{\round-1}) \enspace, \\
    \gckey^{\round}_{i,j}[\msg^{\round-1}_j]
    &=  \gceval(\givcir_{i,j}^{\round-1},\smallset{ \vkey_{i,j,k'}^{\round-1}[\gval_{j,k}^{\round-1}]}_{k,k'} ) \enspace,
  \end{align*}


  where $(k-1) \cdot \len<k'\leq k\cdot\len$ \Fnote{$\lambda$ is also used for the security parameter. And $\len$ needs to be defined.}
  \Fnote{Important: $\smallset{ \vkey_{i,j,k'}^{\round-1}[\gval_{j,k}^{\round-1}]}_{k'}$ should be $\smallset{ \vkey_{i,j,k'}^{\round-1}[{[\ckey_{j,k}^{\round-1}[\gval_{j,k}^{\round-1}]]}_{k' - (k-1)\len}]}_{k'}$. We need a better solution...}
  
  

  
  
  
  and compute,
  \begin{multline*} 
    (\Gckey^{\round+1}_{i}[\Msg^{<\round}], \smallset{\givcir_{\iii,j}^\round,\otflowtwot_{\iii,j,k}^\round,\hcd_{i,k}^{\round}}_{j},\; \msg_j^{\round})= \\
    \gceval(\gifcir_i^{\round}, \Gckey^{\round}_{i}[\Msg^{<\round-1}], \smallset{\gckey^{\round}_{i,j}[\msg^{\round-1}_j]}_j)
  \end{multline*}
  
  for the case where $\ell=L$, $\Gckey^{\round+1}_{i}[\Msg^{<\round}],\smallset{\givcir_{\iii,j}^\round,\otflowtwot_{\iii,j,k}^\round}_{j}=\emptyset$.
  
  % \item if $\ell=L$, for each $k \in [|\vec \wsct^\round|]$ decrypt    \[ m^{\round+1} = \wsdec(\wsct^\round_k,(\msg^\round_k,\hcd^\round_k))\enspace\]
\item After all $\nbrounds$ iterations, $P_\istar$ obtains the set of all
  messages $\Msg$, and computes the output $
  y_\istar =  \mpcoutput_\istar\left(\mpcin_\istar,\mpcrand_\istar,\Msg\right)
  $.
\EE


\EI
\EPR
}


\pprotocol{Circuit $\ifnext_\iii^\round$
}{Pseudocode of circuit $\ifnext_\iii^\ell$}{fig:ic-mpc}{htbp}{
  
  \textbf{Hardwired Values: } $1^\secpar$, $\round$, $\iii$, $\mpcin_\iii$,
  $\mpcrand_\iii$, $\smallset{\hcr_{\iii,k,b}^\round}_{k\in [\cirsize], b\in \bits}$, $\smallset{\otflowonet_{j,k,b}^\round}_{j\in[N],k\in[\cirsize],b\in\bits}$, $\smallset{\gccir_j^\round}_{j \in [N]},\Gckey^{\round+1}_{i}$, $\smallset{\gckey^{\round+1}_{i,j}}_{j\in[N]}$. \\
  \smallskip
    \textbf{Inputs: } $(\Msg^{< \round-1}, \vec \msg^{\round-1})$ where $\vec \msg^{\round-1}:= \smallset{\msg_j^{\round-1}}_{j\in[N]}$.
    
% \[ \msg^{\round-1} = \wsdec(\wsct^{\round-1}_k,(\msg^{\round-1}_k,\hcd^{\round-1}_k))\enspace. \]
  


 % These inputs define $\vec \msg^{< \round}$.

  \smallskip

  \textbf{Procedure: }
  \begin{enumerate}
  \item Define the circuit $G_j^\round$ as $G_j^{\round}(\star,\star) = \mpcnext_j(\star,\star,\Msg^{< \round-1}, \vec \msg^{\round-1})$, for $j \in [N]$ and set $\gval_{j,k}^\round=[G_j^\round]_k$ as the $k$-th bit of $G_j^\round$.
  \item Compute the $\round$-th message of $P_\iii$ in the inner MPC:\\ $\msg_{\iii}^\round =
      \mpcnext_\iii\left(\mpcin_\iii,\mpcrand_\iii, (\Msg^{< \round-1}, \vec \msg^{\round-1})\right)$.

  \item For all $k\in[\cirsize]$, set $\hcd_{\iii,k}^\round=\hcr_{\iii,k,\gval_{\iii,k}^\round}^\round$.
 
 \item Generate input labels $\vkey_{\iii,j}^\round \getsr \gcgen(1^\secpar)$ for $j\in[N]$. 
 \item For all $j\in[N],k\in[\cirsize]$, generate the circuit: %set $\aval_{j,k}^{\round-1}=\ckey_{j,k}^{\round-1}[\gval_{j,k}^{\round-1}]$ and
 
   \[ \iV_{\iii,j}^\round(\aval_{j,k}^{\round}) =
      \begin{cases}
      \msg_j^{\round}= 
    \gceval(\giccir_j^{\round}, \aval_{j,k}^{\round}) \\
       \text{output}~ \gckey^{\round+1}_{i,j}[\msg^{\round}_j]
      \end{cases}
    \]
    \anti{This should be $(\giccir_j^{\round}$. Do you agree or am I confused?}
    \Fnote{I agree. I've changed. Same for $\aval_{j,k}^\round$.}
 
\item Garble the circuit $\givcir_{\iii,j}^\round \getsr \gcgarble(\smallset{\vkey_{\iii,j}^\round}_{j\in[N]},\iV_{\iii,j}^\round) \enspace.$

 \item Generate $\otflowtwot_{\iii,j,k}^\round= \otsendtwo(\otflowonet_{j,k,\gval_{j,k}^\round}^\round,\smallset{\vkey_{\iii,j,k'}^\round}_{(k-1) \cdot \len<k'\leq k\cdot\len} ) \enspace.$
 
  \item Select the input labels $\Gckey^{\round+1}_{i}[\Msg^{< \round-1}, \vec \msg^{\round-1}]$ for the next step for all $j\in[N]$, corresponding to the messages $\Msg^{< \round-1}, \vec \msg^{\round-1}$. 
 %     By convention, $\st^\round$ and $\gckey^{\round+1}[\st^\round]$ are empty if $\round = \nbrounds$.

  \end{enumerate}

  \textbf{Output: } $
    (\Gckey^{\round+1}_{i}[\Msg^{< \round-1}, \vec \msg^{\round-1}], \smallset{\givcir_{\iii,j}^\round,\otflowtwot_{\iii,j,k}^\round,\hcd_{\iii,k}^\round}_{j,k}, \msg_\iii^\round)
  $.

  \smallskip
}




\anti{old stuff}

{\small\BPR [Adaptive semi-malicious protocol $\sMPC$]\label{prot:semimpc} Let $\mpcfunc$ be an arbitrary $N$-party functionality.\footnote{Formal definitions of MPC protocol and $N$-party functionality are provided in \cref{A:def-MPC}.}
Protocol $\Pi_f$ relies on the following components: 
\BI
\item 


An adaptive semi-honest $L$-round $N$-party protocol
  $\mpc=(\mpcnext, \mpcoutput)$ for $\mpcfunc$. Without
loss of generality, we will assume that in each round $\round$ of $\mpc$, each party $P_i$
broadcasts a single message that depends on its input $\mpcin_i$, randomness $\mpcrand_i$ and on the messages $\Msg^{< \round} = \smallset{\msg_j^{\round'}}_{j \in [N], \round' < \round}$ that it
received from all parties in all previous rounds, i.e. $\msg_{j}^\round = \mpcnext_j(\mpcin_j,\mpcrand_j,\Msg^{< \round})$, where $\mpcnext$ is the next message function. In the last round $L$ of $\mpc$ each party $P_i$ locally computes the output $\mpcout_i =
  \mpcoutput_i(\mpcin_i,\mpcrand_i,\Msg)$ after receiving all the messages $\Msg = \smallset{\msg_j^\round}_{j \in [N],\round \in [\nbrounds]}$.


 % The security parameter $\secpar$ is an implicit parameter $1^\secpar$ of $\mpcnext$ and $\mpcoutput$.
\item A semi-malicious equivocal functional commitment scheme
  $\hc=(\setup, \hccom,\\\hcopen,\wsenc,\wsdec,\csim,\Equiv)$ for the class of all
  $\cirsize$-size circuits with a sufficiently large polynomial bound
  $\cirsize$.  %We denote by $\nda^\hc$ the associated non-deterministic oracle family defined in \cref{def:nda-for-hc}.
%   We denote by $\nda^\hc$ and $\ndadistr^\hc$ the associated
%   non-deterministic oracle family and unique-answer distribution
%   defined in \cref{def:nda-for-hc}.
\item A garbled circuit scheme $\gc=(\gcgen,\gcgarble,\gceval,\gcsim)$ for the class of all
  $\cirsize$-size circuits with a sufficiently large polynomial bound
  $\cirsize$.
  
  \EI
\medskip\noindent\textsc{Input:} Parties $P_1,\ldots, P_N$ are given input $(x_1,\ldots,x_N)$ each of length $\kappa$, respectively. 
\BI
\item \textsc{Round 1:} Each party $P_\istar$ generates and broadcasts $\hcc_{\istar}^\round = \hccom(\ck,
    (\mpcin_{\istar},\mpcrand_{\istar}); \hcr_\istar^\round)$ for $\round \in [\nbrounds]$ where $\mpcrand_{\istar}$ is the random tape for running $\mpc$. 


\item \textsc{Round 2:} For $\round$ from  $\nbrounds$ to $1$ each party $P_\istar$ garbles $\iC_\istar  = \smallset{\icnext_\istar ^\round}_{\round \in [\nbrounds]}$, defined in \cref{fig:ic-mpc}, as follows: 
\BE

\item Generate input labels $\Gckey_{\istar} ^\round,\smallset{\gckey_{\istar,j} ^\round}_{j\in[N]} \getsr \gcgen(1^\secpar)$ for $j\in[N]$.
\item Garble the circuit $\icnext^\round_\istar$  and broadcast $\giccir_\istar=\smallset{\giccir_\istar ^\round}_{\round \in [\nbrounds]}$:
  \[ \giccir_\istar^\round \getsr \gcgarble(\Gckey_{\istar}^\round, \smallset{\gckey_{\istar,j}^\round}_{j}, \icnext_{\istar}^\round) \enspace. \]

\EE
\item \textsc{Output phase:} Each party evaluates the garbled circuits. In particular $P_\istar$ proceeds as follows in $\nbrounds$ iterations ($\round \in [\nbrounds]$): 

\BE
\item If $\ell=1$, for $i \in [N]$, execute $(\Gckey^{2}_{i}, \smallset{\wsct^1_{i,j}}_{j},\; \msg_j^1,\hcd_j^1)= \gceval(\giccir_i^1,\emptyset)$ for all $ j \in [N]$ (we recall that $\Gckey^{1}_{i}=\emptyset$ and $\smallset{\gckey^1_{i,j}}_j = \emptyset$).

 
\item For every $1<\round \le L$, for $i \in [N]$, for each $j \in [N]$, decrypt $\wsct^{\round-1}_{i,j}$ using as witness $(\msg^{\round-1}_j,\hcd^{\round-1}_j)$ for the circuit $G_j^{{\round-1}}(\star,\star) = \mpcnext_j(\star,\star, \Msg^{< \round-1})$:
  \[ \gckey^{\round}_{i,j}[\msg^{\round-1}_j] = \wsdec(\ck, \wsct^{\round-1}_{i,j},(\msg^{\round-1}_j,\hcd^{\round-1}_j))\enspace\]
  
  and compute,
  \begin{multline*} 
    (\Gckey^{\round+1}_{i}[\Msg^{<\round}], \smallset{\wsct^{\round}_{i,j}}_{j},\; \msg_j^{\round},\hcd_j^{\round})= \\
    \gceval(\giccir_i^{\round}, \Gckey^{\round}_{i}[\Msg^{<\round-1}], \smallset{\gckey^{\round}_{i,j}[\msg^{\round-1}_j]}_j)
  \end{multline*}
  
  for the case where $\ell=L$, $\Gckey^{\round+1}_{i,j}[\Msg^{<\round}],\smallset{\wsct^{\round}_{i,j}}_{j}=\emptyset$.
  
  % \item if $\ell=L$, for each $k \in [|\vec \wsct^\round|]$ decrypt    \[ m^{\round+1} = \wsdec(\wsct^\round_k,(\msg^\round_k,\hcd^\round_k))\enspace\]
\item After all $\nbrounds$ iterations, $P_\istar$ obtains the set of all
  messages $\Msg$, and computes the output $
  y_\istar =  \mpcoutput_\istar\left(\mpcin_\istar,\mpcrand_\istar,\Msg\right)
  $.
\EE


\EI
\EPR
}








%{We will show that using the constructions in \cref{sec:cons-gic,sec:cons-hc-ws}, we can construct the two last tools from 2-round (semi-honest) OT.}
\remove{
\subparagraph{The First Round:} Each party $P_{\iii}$ computes its
first message 
% $\tmsg_{\iii}^1$ 
  $\tmsg_{\iii}^1 = \tmpcnext_{\iii}(\mpcin_{\iii},\allowbreak\tmpcrand_{\iii},\allowbreak \emptyset)$, 
using security parameter $\secpar$, input $\mpcin_{\iii}$,
random tape $\tmpcrand_{\iii}$, and no messages, as follows.
\begin{enumerate}
\item Take a sufficient long substring $\mpcrand_{\iii}$ of
  $\tmpcrand_{\iii}$ as the random tape for running the inner
  MPC protocol $\mpc$.

\item Commit $\nbrounds$ times to $(\mpcin_{\iii},\mpcrand_{\iii})$ using the functional
  commitment scheme $\hc$: for $\round \in [\nbrounds]$,
  %\begin{align*}
    $\hcc_{\iii}^\round = \hccom(1^\secpar,
    (\mpcin_{\iii},\mpcrand_{\iii}); \hcr_\iii^\round)$, %\enspace,
  %\end{align*}
    where all the $\hcr_\iii^\round$'s (and $\mpcrand_i$) are non-overlapping substrings of $\tmpcrand_\iii$. %is a fresh substring of the random tape $\tmpcrand_\iii$.
\item  Broadcast the first message $\tmsg_{\iii}^1 =
  \smallset{\hcc^\round_{\iii}}_{\round \in [\nbrounds]}$, and keep $\smallset{\hcr^\round_\iii}_{\round \in [\nbrounds]}$ secret. 
\end{enumerate}}

\remove{\subparagraph{The Second Round:} Each party $P_{\iii}$ computes
its second message 
  $\tmsg_\iii^2 = \tmpcnext_{\iii}(\mpcin_{\iii},\allowbreak\tmpcrand_{\iii},\allowbreak \smallset{\tmsg_j^1}_{j \in N})$,
using all first messages $\smallset{\tmsg_j^1}_{j \in N}$ as follows: 


  \begin{enumerate}
  \item[]$\giccir_\iii \getsr \gicgarble(1^\secpar, \iC_\iii)$ garbles the interactive circuit $\iC_\iii  = \smallset{\icnext_\iii ^\round}_{\round \in [\nbrounds]}$ into $\giccir_\iii $ as follows: For $\round$ from  $\nbrounds$ to $1$, 
  \item Generate input labels $\gckey_\iii ^\round \getsr \gcgen(1^\secpar)$.
\item Garble the circuit $\icnext^\round_\iii$ defined in \cref{fig:ic-mpc}:
  \[ \giccir_\iii^\round \getsr \gcgarble(\gckey_\iii^\round, \icnext_\iii^\round) \enspace. \]
\item Broadcast the second message $\tmsg_{\iii}^2 = \giccir_\iii$. 
\end{enumerate}}

\pprotocol{Circuit $\icnext_\iii^\round$
}{Pseudocode of circuit $\icnext_\iii^\ell$}{fig:ic-mpc}{htbp}{
  
  \textbf{Hardwired Values: } $1^\secpar$, $\round$, $\iii$, $\mpcin_\iii$,
  $\mpcrand_\iii$, $\hcr_\iii^\round$, $\Gckey^{\round+1}_{i}$, $\smallset{\gckey^{\round+1}_{i,j},\hcc_j^\round}_j$, $\mpcnext_\iii$. \\
  \smallskip
    \textbf{Inputs: } $(\Msg^{< \round-1}, \vec \msg^{\round-1})$ 
    
% \[ \msg^{\round-1} = \wsdec(\wsct^{\round-1}_k,(\msg^{\round-1}_k,\hcd^{\round-1}_k))\enspace. \]
  


 % These inputs define $\vec \msg^{< \round}$.

  \smallskip

  \textbf{Procedure: }
  \begin{enumerate}
  \item Define the circuit $G_j^\round$ as $G_j^{\round}(\star,\star) = \mpcnext_j(\star,\star,\Msg^{< \round})$, for $j \in [N]$.
  \item Compute the $\round$-th message of $P_\iii$ in the inner MPC:\\ $\msg_{\iii}^\round =
      \mpcnext_\iii\left(\mpcin_\iii,\mpcrand_\iii, (\Msg^{< \round-1}, \vec \msg^{\round-1})\right)$.

  \item Compute the associated functional decommitment of $\hcc_\iii^\round$:\\
    $ \hcd_\iii^\round = \hcopen(\ck, \hcc^\round_\iii, G_\iii^\round, (\mpcin_\iii,\mpcrand_\iii), \hcr^\round_\iii)$.

 % \item Compute the next queries: for every $j \in [N]$, $q^\round_j = (\hcc_j^{\round},G_j^\round)$.

%  \item Define the output by $\icout^\round_\iii = (\msg_\iii^\round,\hcd_\iii^\round)$.
 \item For every $j \in [N]$, generate using a hardcoded random tape:
      \[ \wsct^\round_{i,j} = \wsenc(\ck,\hcc_j^{\round},G_j^\round,\gckey^{\round+1}_{i,j})\enspace, \]
  %    where $\gckey^{\round+1}[[\msg_k^\round,\hcd_k^\round]]$ is the tuple of input labels $\gckey^{\round+1}[i,b]$ for all $b \in \bits$ and for the input wires 
% set $\vec \wsct^\round = \smallset{\wsct^\round_j}_j$.
 
 
  \item Select the input labels $\Gckey^{\round+1}_{i}[\Msg^{<\round}]$ for the next step for all $j\in[N]$, corresponding to the messages $\Msg^{< \round} $. 
 %     By convention, $\st^\round$ and $\gckey^{\round+1}[\st^\round]$ are empty if $\round = \nbrounds$.

  \end{enumerate}

  \textbf{Output: } $
    (\smallset{\Gckey^{\round+1}_{i}[\Msg^{<\round}]}_j, \smallset{\wsct^\round_{i,j}}_j,\; \msg_\iii^\round,\hcd_\iii^\round)
  $.

  \smallskip
}



% We have the following lemma proven in \cref{A:lem-mpc-ic-nda-cons}.

% \begin{lemma}
%   \label{lem:mpc-ic-nda-cons}
%   The distribution $\icdistr$ defined above is consistent with $\ndadistr^\hc$.
% \end{lemma}


\begin{theorem}
  \label{th:sec-sh-mpc}
 Assume the existence of an equivocal semi-malicious functional commitment scheme $\hc$, a garbled circuit scheme $\gic$ and an $N$-party adaptive semi-honest
protocol $\mpc$. Then the two-round protocol $\sMPC$, presented in Protocol \ref{prot:semimpc}, securely realizes
any multi-party functionality against an adaptive semi-malicious adversary in the CRS model.


\end{theorem}


\subsubsection{Description of the Simulator.}\label{sec:simu}
Let $\ccP = \{P_1,\ldots, P_N\}$ be the set of parties, let $\cA$ be a malicious, adaptive adversary, and let $\ccP^* \subseteq \ccP$ be the set of parties corrupted by~$\cA$. We construct a simulator $\cS$ (the ideal world adversary) with access to the ideal functionality $\mathbf{\func}$, such that the ideal world experiment with $\cS$ and $\mathbf{\func}$ is indistinguishable from a real execution of $\sMPC$ with $\cA$. %The simulator $\cS$ only generates messages on behalf of parties $\ccP\backslash \ccP^*$, as follows:


\paragraph{Simulated CRS:} $\cS$ runs the algorithm $\ehcsetup(1^\secpar)$ of the semi-malicious equivocal functional commitment scheme $\hc$ and obtains $(\ck,\teq)$. $\cS$ sets the common reference string equal to $\ck$ and locally stores $\teq$.



\paragraph{Round 1 Messages $\cS \rightarrow \cA$:} In the first round, $\cS$ generates messages on behalf of each honest party $P_i \notin \ccP^*$, as follows:

\BI
\item Generate a fake commitment $(\hcc_i^\round,\state^\cc)\getsr \hcsim({\ck},{\teq})$ for $\round \in [\nbrounds]$ and for the circuit $G_\iii^\round$. 
 \item Send $\hcc_i$ to $\Adv$ on behalf of party $P_i$ and internally save $\state_i^\cc$.
\EI





\paragraph{Round 1 Messages $\cA \rightarrow \cS$:} Also in the first round the adversary $\cA$ generates the messages $\hcc_{j}^\round = \hccom(\ck,
    (\mpcin_{j},\mpcrand_{j}); \hcr_j^\round)$ on behalf of corrupted parties $P_j\in\ccP^*$ to honest parties $P_i\notin\ccP^*$. 

\paragraph{Completion of Round 1:} $\cS$ receives the random tapes $\smallset{\hcr_i^\round}_{j \in \ccP^*, \round \in [\nbrounds]}$ and the inputs and randomness $\smallset{\mpcin_j,\mpcrand_j}_{P_j \in \ccP^*}$ of the inner protocol $\pi$. $\cS$ runs the simulator of the inner protocol $\pi$ to receive $\Msg$. 

\paragraph{Simulating corruption of parties in Round 1:}  When $\cA$ corrupts a real world party $P_i$, then $\cS$ corrupts the corresponding ideal world party $P_i$ and prepares the internal state on behalf of $P_i$ such that it will be consistent with the commitment value $\hcc_i$ that it had provided to $\cA$ earlier. In particular, $\cS$ computes ${\hcr}_i^{\round} \getsr\ehcsimd({\ck},{\teq},\hcc_i^\round,\state^\cc_i,\hcmsg)$ for $\round \in [\nbrounds]$ such that ${\hcc}_i^\round=\hccom({\ck},\hcmsg;\hcr_i^\round)$. $\cS$ provides $\{{\hcr}_i^\round\}_{\round \in [\nbrounds]}$ as the randomness of party $P_i$ to $\cA$. 


\paragraph{Round 2 Messages $\cS \rightarrow \cA$:} In the first round, $\cS$ generates messages on behalf of each honest party $P_i \notin \ccP^*$, as follows:

\BI
\item Equivocate the commitment $\hcd_\iii^\round \getsr \fequiv(\ck,\hcc^\round_\iii,\ehctrap^\round_\iii,G_\iii,\msg_\iii^\round) \enspace$ for $\round \in [\nbrounds]$.
\item For $\round$ from $\nbrounds$ to $1$, the simulator does the following:
  \BE
%  \item Define $\gckey^{\round}, \Gckey^{\round}$ to be such that $\gckey^{\round}[i,b] = \gckey'^{\round}[i]$ for all input wire $i$ and all bits $b \in \bits$. $\gckey'^{\nbrounds+1}$ and $\gckey^{\nbrounds+1}$ are empty.
\item Choose input labels $\Gckey_{\iii} ^\round,\smallset{\gckey_{\iii,j} ^\round}_{j\in[N]}$ for $j\in[N]$ at random.
 \item Simulate the garbling of $\icnext^\round_\iii$:

    \[ \giccir^\round_i \getsr \gcsim(1^\secpar,(\Gckey^{\round}_{i}[\Msg^{<\round+1}],\smallset{\wsct^\round_{i,j}}_j,\msg^\round_i,\hcd^\round_i)) \enspace. \] 
  \item Encrypt the labels generated for the round $\round+1$ corresponding to $\msg_j^\round,\hcd_j^\round$ for each $j$:
  
    \[ \wsct^\round_{i,j}  \getsr \wsenc(\ck,\hcc_j^{\round},G_j^\round,\gckey^{\round+1}_{i,j}[\msg_j^\round])\enspace, \]

  
%    \[ \wsct^\round_k \getsr \wsenc(1^\secpar, \vec q^\round, \gckey^{\round+1}[[a^\round_k]])\enspace. \]
    (For $\round = L$, $\{\wsct_{i,j}^\round\}$ and $\gckey^{\round+1}$ are empty.)
 
  \EE

\EI





\paragraph{Round 2 Messages $\cA \rightarrow \cS$:} Also in the second round the adversary $\cA$ generates the garble circuits $\giccir_j=\smallset{\giccir_j^\round}_{\round \in [\nbrounds]}$ on behalf of corrupted parties $j\in\ccP^*$ to honest parties $h\notin\ccP^*$. 

\paragraph{Simulating corruption of parties during/at the end of Round 2:}
When $\cA$ corrupts a party $P_i$ in the real word, then $\cS$ corrupts the corresponding party $P_i$ in the ideal world and prepares the internal state on behalf of $P_i$ such that it will be consistent with messages it had sent on behalf of $P_i$.
As explained before, $\cS$ generates randomness ${\hcr}_i^{\round} \getsr\ehcsimd({\ck},{\teq},\hcc_i^\round,\state^\cc_i,\hcmsg)$ for $\round \in [\nbrounds]$ such that ${\hcc}_i^\round=\hccom({\ck},\hcmsg;\hcr_i^\round)$ and $\hcd_\iii^\round \getsr \fequiv(\hcc^\round_\iii,\ehctrap^\round_\iii,G_\iii,\msg_\iii^\round) \enspace$.
Next, $\cS$ needs to explain ciphertexts $\wsct^\round_{i,j}$ and garble circuits $\giccir^\round_i$.
To this end, $\cS$ generates ${\renc}'_{i,j} \getsr\encequiv({\ck},{\teq},\wsct_{i,j}^\round,\state^\ee_{i,j}, \wsmsg,\hccir_j^\round,M, \hcd_j^\round)$ such that  $\wsct^\round_{i,j}  = \wsenc(\ck,\hcc_j^{\round},G_j^\round,\wsmsg;{\renc}'_{i,j})$ where $\wsmsg=\{\gckey^{\round+1}_{i,j}[\msg_j^\round]\}_j$.
Then, $\cS$ generates randomness $\sigma \getsr \gcequiv( 
    \giccir^\round_i,\cirin_i, \Gckey^{\round}_{i}[\Msg^{<\round-1}], \smallset{\gckey^{\round}_{i,j}[\msg^{\round-1}_j]}_j)$. \\






% \item ${\hcr}' \getsr\ehcsimd({\ck},{\teq},\hcc,state^\cc,\hcmsg)$ such that $\hccom({\ck},\hcmsg;\hcr')={\hcc}$;
%\item ${\renc}' \getsr\encequiv({\ck},{\teq},\wsct,\state^\ee, \wsmsg,\hccir,\hcout, \hcd)$ such that $ \wsenc(\ck, \hcc, \hccir, \hcout,
 %   \wsmsg;\renc')=\wsct $

This completes the description of the simulator.


\subsubsection{Proof of Indistinguishability.}
We need to prove that for any semi-malicious adaptive adversary $\cA$, the view generated by the simulator $\cS$ above is indistinguishable from the real view, namely:


  \[ \smallset{\mpcideal_{\ccP^*,\mpcsim}(1^\secpar,\mpcvecin)}_{\secpar,\ccP^*,\mpcvecin} \approx \smallset{\mpcrealsm_{\ccP^*,\advA}(1^\secpar,\mpcvecin)}_{\secpar,\ccP^*,\mpcvecin} \enspace. \]

%$$
%\left\{\ideall_{\mathbf{\func},\cS}(\kappa,\cdot)\right\}_{\kappa} \indist
%\left\{\reall_{\Pi,\cA}(\kappa,\cdot)\right\}_{\kappa}
%$$
To prove indistinguishability, we consider a sequence of hybrid experiments. Let $H_0$ be the hybrid describing the real-world execution of the protocol, and we modify it in steps:


%---------------------------------------------------------------------
\paragraph{$\hyb{0}$:}
This hybrid is the real execution. In particular, $H_0$ starts the execution of $\cA$ providing it fresh randomness and input $\{x_j\}_{P_j \in\ccP^* }$, and interacts with it honestly by performing all actions of the honest parties with uniform randomness and input. The output consists of $\cA$'s view.


\paragraph{$\hyb{1}$:} In this hybrid we change the way the public parameters of the commitment scheme are generated. In particular, $\cS$ runs the setup algorithm $(\ck,\teq) \getsr \ehcsetup(1^\secpar)$ and stores trapdoor $\teq$.

\begin{lemma}
$\hyb{0}\compind \hyb{1}$.
\label{lemma:hyb1}
\end{lemma}
\begin{proof}
The indistinguishability of $\hyb{1}$ from $\hyb{0}$ follows from the indistinguishability of public parameters of the commitment scheme $\hc$. 
\end{proof}


\paragraph{$\hyb{2}$:} This hybrid is similar to hybrid $\hyb{1}$, except that the ciphertexts $\{\wsct_i^\round\}_i$ are simulated: $(\wsct_i^\round,\state_i^\ee) \getsr \encsim(\ck, \hcc_i^\round, {\teq}, \rho_i^\round)$.


\begin{lemma}
$\hyb{1}\compind \hyb{2}$.
\label{lemma:hyb1}
\end{lemma}
\begin{proof}
The indistinguishability of $\hyb{1}$ from $\hyb{2}$ follows from the soundness security of the commitment scheme $\hc$. 
\end{proof}


\paragraph{$\hyb{3}$:}
In this hybrid, we change how the internal randomness of the corrupted party is explained to $\cA$ on being adaptively corrupted. Specifically we change the randomness that is used to explain the ciphertext $\cS$ generates on behalf of parties in round 2 of protocol $\sMPC$: 
${\renc}'_i \getsr\encequiv({\ck},{\teq},\wsct_i^\round,\state^\ee_i, \wsmsg,\hccir_i^\round,M, \hcd_i^\round)$.





\begin{lemma}
$\hyb{2}\compind \hyb{3}$.
\label{lemma:hyb1}
\end{lemma}
\begin{proof}
The indistinguishability of $\hyb{1}$ from $\hyb{2}$ follows from the equivocation property of the commitment scheme $\hc$. 
\end{proof}


\paragraph{$\hyb{4}$:} This hybrid is similar to $\hyb{3}$, except that we encrypt twice the input labels that correspond to the actual inputs instead of encrypting both possible input labels: 
  \[ \wsct^\round_j = \wsenc(\ck, \hcc_j^{\round},G_j^\round,\smallset{\gckey^{\round}_{i,j}[{\msg_j^\round}]}_j)\enspace\]

\begin{lemma}
$\hyb{3}\compind \hyb{4}$.
\label{lemma:hyb1}
\end{lemma}
\begin{proof}
The indistinguishability of $\hyb{2}$ from $\hyb{3}$ follows from the soundness security of the commitment scheme $\hc$. 
\end{proof}


\paragraph{$\hyb{5}$:} This hybrid is similar to hybrid $\hyb{4}$, except that the garble circuits are simulated:  
\BE
\item $\cS$ chooses input labels $\{\Gckey_{\iii,j} ^\round,\gckey_{\iii,j} ^\round\}_{j\in[N]}$ for $j\in[N]$ at random.
 \item $\cS$ simulates the garbling of $\icnext^\round_\iii$:

    \[ \giccir^\round_i \getsr \gcsim(1^\secpar,(\Gckey^{\round+1}_{i,j}[\Msg^{<\round+1}],\smallset{\wsct^\round_j}_j,\msg^\round_i,\hcd^\round_i)) \enspace. \]
    
  \item $\cS$ equivocates $\giccir^\round_i$ by generating randomness:

    \[\sigma \getsr \gcequiv( 
    \giccir^\round_i,\cirin_i, (\Gckey^{\round}_{i}[\Msg^{<\round-1}],\smallset{\gckey^{\round}_{i,j}[\msg^{\round-1}_j]}_j))\enspace. \]
\EE

\begin{lemma}
$\hyb{4}\compind \hyb{5}$.
\label{lemma:hyb1}
\end{lemma}
\begin{proof}
The indistinguishability of $\hyb{3}$ from $\hyb{4}$ follows from the simulation and equivocation property of the garble circuit scheme $\gc$. 
\end{proof}
\paragraph{$\hyb{6}$:} In this hybrid we change the way $\cS$ generates the message $\smallset{\hcc_j^\round,\hcd_j^\round}_{j \notin I, \round \in [\nbrounds]}$ on behalf of the honest parties. Instead of computing

       \[ \hcc_{j}^\round = \hccom(\ck,
      (\mpcin_{j},\mpcrand_{j}); \hcr_j^\round); \
      \hcd_j^\round = \hcopen(\ck,\hcc_j^\round, G_j^\round,\mpcrand_{j}, \hcr_j^\round)
      \enspace, \]
   $\cS$ simulates:
   
   
   \[ (\hcc_i^\round,\state^\cc)\getsr \hcsim({\ck},\ehctrap_j^\round), \hcd \getsr \fequiv(\ck,\hcc_j^\round, \ehctrap_j^\round,G_j^\round,\msg_{j}^\round) \enspace. \]
     



\begin{lemma}
$\hyb{5}\compind \hyb{6}$.
\label{lemma:hyb1}
\end{lemma}
\begin{proof}
The indistinguishability of $\hyb{4}$ from $\hyb{5}$ follows from the simulation and equivocation property of the commitment scheme $\hc$. 
\end{proof}
   
   
      \paragraph{$\hyb{7}$:} In this hybrid we change the way $\cS$ generates the commitments on behalf of the honest parties. In particular we will remove the inputs and make these commitments equivocal. In particular, $\cS$ generates randomness ${\hcr}_i^{\round} \getsr\ehcsimd({\ck},{\teq},\hcc_i^\round,\state^\cc,\msg_{j}^\round)$ for $\round \in [\nbrounds]$ such that ${\hcc}_i^\round=\hccom({\ck},\msg_{j}^\round;\hcr_i^\round)$. 
      
      
\begin{lemma}
$\hyb{6}\compind \hyb{7}$.
\label{lemma:hyb1}
\end{lemma}
\begin{proof}
The indistinguishability of $\hyb{5}$ from $\hyb{6}$ follows from the equivocation property of the commitment scheme $\hc$. 
\end{proof}
   
   \paragraph{$\hyb{8}$:} In this hybrid $\cS$ essentially simulates the inner mpc protocol messages $\Msg$ after seeing the random tapes and inputs of the semi-malicious adversary for $\pi$.
    \begin{lemma}
$\hyb{7}\compind \hyb{8}$.
    \label{lemma:hyb1}
\end{lemma}
\begin{proof}
We base the indistinguishability between hybrids $\hyb{6}$ and  $\hyb{7}$ on the simulation property of the the inner protocol $\pi$. 
\end{proof}

    



\subsection{Recall: 2-Round Adaptive Semi-Malicious Oblivious Transfer}
We recall the definition of 2-round OT.
\begin{mydefinition}
  \label{def:2ot}
  A \emph{2-round adaptive semi-malicious oblivious transfer (OT)} is a tuple of three polynomial-time algorithms $\ot = (\OTSetup,\otsendone,\otsendtwo,\\\otoutput,\ROTequiv,\SOTequiv)$:
  \begin{description}
   \item[Setup:] $\OTSetup$ can be viewed as a pair of PPT algorithms $(\qOTsetup,\OTsetup)$ such that the following holds:
   \BI
   \item $\crs \getsr \OTsetup(1^\secpar)$ expects as input the unary representation of the security parameter $\secpar$ and
outputs a public parameter $\crs$.
    \item $(\crs,\teq) \getsr \OTSetupq(1^\secpar)$ outputs a public parameter $\crs$ together with a trapdoor $\teq$ (used for equivocation).
  \EI

  \item[First Round:] $\otflowone = \otsendone(\crs,\otsel;\otrandone)$ generates the first flow $\otflowone$ (from the receiver to the sender) for the selection bit $\otsel \in \bits$, the security parameter~$\secpar$, and the random tape $\otrandone \in \bits^{\otrandonelen}$, where $\otrandonelen$ is polynomial in $\secpar$;
  \item[Second Round:] $\otflowtwo \getsr \otsendtwo(\crs,\otflowone,\otmsg_0,\otmsg_1;\zeta)$  generates the second flow (from the sender to the receiver) for the messages $(\otmsg_0,\otmsg_1) \in {(\bits^\otmsglen)}^2$, where the message length $\otmsglen$ is polynomial in $\secpar$;
  \item[Output:] $\otmsg = \otoutput(\otflowtwo,\otsel,\otrandone)$ computes the output $\otmsg \in \bits^\otmsglen$ of the receiver;
  
  
  \item[Sender Equivocation:] ${\zeta} \getsr\SOTequiv({\crs},{\teq},\otflowtwo,\otflowone,\otmsg_b)$
 such that  $\otflowtwo \getsr \otsendtwo(\otflowone,\otmsg_b,\otmsg_{1-b};\zeta)$. 
 
   \item[Receiver Equivocation:] $\otrandone \getsr\ROTequiv({\crs},{\teq},\otflowone,\otsel)$ such that $\otflowone = \otsendone(1^\secpar,\otsel;\otrandone)$.\anti{update both syntax according to the last section}
   

  
  
  
  
  
  \end{description}
  satisfying the following properties:
  \begin{description}
  \item[Correctness:] For any security parameter $\secpar \in \N$, for any selection bit $\otsel \in \bits$, for any messages $(\otmsg_0,\otmsg_1) \in {(\bits^\otmsglen)}^2$, for any $\otrandone \in \bits^{\otrandonelen}$, it holds that:
        \begin{multline*}
      \Pr\Big[\otflowone = \otsendone(1^\secpar,\otsel;\otrandone); \ \otflowtwo \getsr \otsendtwo(\otflowone,\otmsg_0,\otmsg_1) \ : \\
       \otmsg_\otsel = \otoutput(\otflowtwo,\otsel,\otrandone) \Big] = 1 \enspace;
    \end{multline*}
  \item[Receiver Privacy:] The following two distributions are computationally indistinguishable:
    \begin{align*}
      \big\{ \otsendone(1^\secpar,0;\otrandone) \ &: \ \otrandone \getsr \bits^\otrandonelen {\big\}}_{\secpar}\enspace, \\
      \big\{ \otsendone(1^\secpar,1;\otrandone) \ &: \ \otrandone \getsr \bits^\otrandonelen {\big\}}_{\secpar}
        \enspace;
    \end{align*}
  \item[Semi-Malicious Sender Privacy:] The following two distributions are computationally indistinguishable:
    \begin{align*}
      \big\{ (\otrandone, \otsendtwo(\otflowone,\otmsg_0,\otmsg_1))\ &: \otflowone = \otsendone(1^\secpar,\otsel;\otrandone)  {\big\}}_{\secpar, \otsel, \otmsg_0, \otmsg_1} \enspace, \\
      \big\{ (\otrandone, \otsendtwo(\otflowone,\otmsg_\otsel,\otmsg_\otsel)) \ &: \otflowone = \otsendone(1^\secpar,\otsel;\otrandone) {\big\}}_{\secpar, \otsel, \otmsg_0, \otmsg_1} \enspace.
    \end{align*}
    \Fnote{I think $\otrandone$ should also be an index of the distributions.}
        \item[ Indistinguishability of Public Parameters:] We require that:   
         \begin{equation}\nonumber
\begin{split}
\Big|\Pr\Big{[}(\crs)\leftarrow&  \OTsetup(1^\secpar): \mathcal{A}(\ck)=1\Big{]}-\\ &
\Pr\Big{[}({\crs},{\teq})\leftarrow \OTSetupq(1^\secpar) :\mathcal{A}({\crs})=1\Big{]}\Big| < \negl(\secpar).
\end{split}
\end{equation}
    
    
  \end{description}
\end{mydefinition}
\anti{receiver and sender equivocation properties are defined in the OT transformation, sections, bring them here}
\subsection{Functional Commitment with WS from 2-Round OT}

Let $\hccirclass = {\{\hccirclass_\secpar\}}_{\secpar \in \N}$ be a $S$-size circuit class (where $S$ is polynomial in $\secpar$).
To construct a functional commitment scheme $\hc=(\hccom,\hcopen,\hcver,\allowbreak\hcsim)$ with an associated witness selector $\ws = (\wsenc,\wsdec)$, we rely on the following tools:
\begin{itemize}
\item A $\poly$-size universal circuit family $\smallset{\univcirc_\secpar}_{\secpar \in \N}$ for $\hccirclass$; we recall that $\univcirc_\secpar(\hcmsg,\hccir) = \hccir(\hcmsg)$, for $\hccir \in \hccirclass_\secpar$ and $\hcmsg \in \bits^{\hccirinlen}$.
\item A garbled circuit scheme $\gc=(\gcgen,\gcgarble,\gceval,\gcsim)$ for the circuit class $\smallset{\smallset{\univcirc_\secpar(\star,\hcmsg)}_{\hcmsg \in \bits^{\hccirinlen}}}_{\secpar \in \N}$ of partially evaluated universal circuits on any possible input $\hcmsg$; we recall that the input of the circuit $\univcirc_\secpar(\star,\hcmsg)$ is a circuit $\hccir \in \hccirclass_\secpar$ represented by a $S$-bit string $(\hccir[1],\dots,\hccir[S]) \in \bits^S$.
\item A garbled circuit scheme $\tgc = (\tgcgen, \tgcgarble, \tgceval, \tgcsim)$ for the class of
  $\tcirsize$-sized circuits with a sufficiently large polynomial bound
  $\tcirsize$. The prefix ``$\mathsf{o}$'' stands for ``outer'' as this garbled circuit scheme will be used in the WS encryption procedure to garble a circuit containing the $\gceval$.
\item A 2-round OT $\ot = (\otsendone,\otsendtwo,\otoutput)$ with sufficiently large message size $k = |\otmsg_0| = |\otmsg_1|$.\footnote{This is without loss of generality, as we can always repeat in parallel a 1-bit-message 2-round OT to get a $\poly(\secpar)$-bit-message 2-round OT.}
\end{itemize}

The construction proceeds as follows:
\begin{description}
\item[Commitment:] $\hcc = \hccom(1^\secpar, \hcmsg;\hcr)$ commits to $\hcmsg \in \bits^\hccirinlen$ as follows:
  \begin{enumerate}
  \item Generate input labels $\gckey \getsr \gcgen(1^\secpar)$ (using a random tape derived from $\hcr$).
  \item Garble $\cir = \univcirc_\secpar(\star,\hcmsg)$, which is the universal circuit partially evaluated on~$\hcmsg$: $\gccir \getsr \gcgarble(\gckey,\cir)$.
  \item For each $i \in [S]$, for each bit $b \in \bits$, for each $j \in [|\gckey[i,b]|]$, generate a first flow $\otflowone_{i,b,j} = \otsendone(1^\secpar,\gckey[i,b]_j; \otrandone_{i,b,j})$, where $\gckey[i,b]_j$ is the $j$-th bit of the input label $\gckey[i,b]$ and where the random tape $\otrandone_{i,b,j}$ is derived from $\hcr$.
  \end{enumerate}
  And returns $\hcc = (\gccir, \smallset{\otflowone_{i,b,j}})$.
\item[Functional Opening:] $\hcd = \hcopen(\hcc, \hccir, \hcmsg, \hcr)$ derives the functional decommitment $\hcd$ of $\hcc$ to $\hcout = \hccir(\hcmsg) = \univcirc_\secpar(\hccir, \hcmsg)$ as follows: $\hcd = \smallset{\gckey'[i], \smallset{\otrandone'_{i,j}}_j}_{i \in [S]}$, 
  where $
    \gckey'[i] = \gckey[i,\hccir[i]]$ and $\otrandone'_{i,j} = \otrandone_{i,\hccir[i],j}$.
\item[Functional Verification:] $\hcver(\hcc,\hccir,\hcout,\hcd)$ returns $1$, if and only if for all $i \in [S]$ and $j \in [|\gckey'[i]|]$:
  \begin{align*}
    \otflowone_{i,\hccir[i],j} &= \otsendone(1^\secpar,\gckey'[i]_j; \otrandone'_{i,j})
    &&\text{ and }
    &\hcout &= \gceval(\gccir, \gckey')\enspace.
  \end{align*}        
\item[Simulation:] $(\hcc,\hcd) \getsr \hcsim(1^\secpar, \hccir, \hcout)$ generates the commitment $\hcc$ and its functional decommitment $\hcd$ as follows:
  \begin{enumerate}
  \item Simulate the garble circuit and its partial key: $(\gckey',\gcsimcir) \getsr \gcsim(1^\secpar,\hcout)$.
  \item Define $\gckey$ as follows: $\gckey[i,\hccir[i]] = \gckey'[i]$ and $\gckey[i,1-\hccir[i]] = 0^{|\gckey'[i]|}$.
  \item For each $i \in [S]$, for each bit $b \in \bits$, for each $j \in [|\gckey[i,b]|]$, generate a first flow $\otflowone_{i,b,j} = \otsendone(1^\secpar,\gckey[i,b]_j; \otrandone_{i,b,j})$, where $\gckey[i,b]_j$ is the $j$-th bit of the input label $\gckey[i,b]$ and where the random tape $\otrandone_{i,b,j}$ is derived from $\hcr$.
  \end{enumerate}
  And sets
  $\hcc = (\gcsimcir, \smallset{\otflowone_{i,b,j}})$
  and
  $\hcd = \smallset{\gckey'[i], \smallset{\otrandone'_{i,j}}_j}_{i \in [S]}$,
  where $\gckey'[i] = \gckey[i,\hccir[i]]$ and $\hcr'_{i,j} = \hcr_{i,\hccir[i],j}$.
\item[Encryption:] $\wsct \getsr \wsenc(1^\secpar, (\hcc,\hccir), \wsmsg)$ encrypts the messages $\wsmsg
  =\smallset{\wsmsg[I,B]}_{I,B}$ for $\ndax=(\hcc,\hccir)$ into $\wsct$ as follows: 
  \begin{enumerate}
  \item For every $I \in [\hcciroutlen]$ and $B \in \bits$, create the circuit:
    \[ \tcir_{I,B}(\gckey') =
      \begin{cases}
        \wsmsg[I,B] &\text{if } \hcout_I = B \text{ where } \hcout = \gceval(\gccir,\gckey'), \\
        \perp &\text{otherwise.}
      \end{cases}
    \]
    
    
  \item For every $I \in [\hcciroutlen]$ and $B \in \bits$, garble this circuit: $\tgckey_{I,B} \getsr \tgcgen(1^\secpar)$ and $\tgccir_{I,B} \getsr \tgcgarble(\tgckey, \tcir_{I,B})$; we write $\tgckey_{I,B}[i,j,b]$ the key corresponding to the $j$'bit of the input $\gckey'[i]$ of $\tcir_{I,B}$ being~$b$ (i.e., $\gckey'[i]_j = b$, where $\gckey' = \smallset{\gckey'[i]}$ is the input of the circuit $\tcir_{I,B}$).
  \item Define the OT messages: $\otmsg_{i,j,b} = \smallset{\tgckey_{I,B}[i,j,b]}_{I,B}$.
  \item Compute the second flows of the OT corresponding to the first flows $\otflowone_{i,\hccir[i],j}$: $\otflowtwo_{i,j} \getsr \otsendtwo(\otflowone_{i,\hccir[i],j},\otmsg_{i,j,0},\otmsg_{i,j,1})$;
  \end{enumerate}
  and return
  $ \wsct = (\smallset{\tgccir_{I,B}}_{I \in [\hcciroutlen],B \in \bits}, \smallset{\otflowtwo_{i,j}}_{i \in [S], j \in [|\gckey[i,0]|]})$.
\item[Decryption:] $\wsmsg = \wsdec(\wsct, (\hcout,\hcd))$ decrypts $\wsct$ as follows:
  \begin{enumerate}
  \item For every $i \in [S]$ and $j \in [|\gckey'[i]|]$, compute:
    \[ \smallset{\tgckey'_{I,B}[i,j]}_{I,B} = \otmsg_{i,j,\gckey'[i]_j} = \otoutput(\otflowtwo_{i,j}, \gckey'[i]_j, \otrandone'_{i,j}) \enspace; \]
  \item For every $I \in [\hcciroutlen]$ and $B = \hcout_I$, evaluate the garble circuit $\tgccir_{I,B}$:
    \[ \wsmsg[I,B] = \tgceval(\tgccir_{I,B},\smallset{\tgckey'_{I,B}[i,j]}_{i,j}) \]
  \end{enumerate}
  and return $\wsmsg = \smallset{\wsmsg[I,\hcout_I]}_{I \in [\hcciroutlen]}$.
\end{description}

\vspace{\baselineskip}

Correctness of the functional commitment scheme is straightforward.
Correctness for the decryption of the witness selector comes from the fact that:
\[ \smallset{\tgckey'_{I,B}[i,j]}_{I,B} = \smallset{\tgckey_{I,B}[i,j,\gckey[i,\hccir[i]]_j]}_{I,B} \]
and therefore $\tgceval(\tgccir_{I,B},\smallset{\tgckey'_{I,B}[i,j]}_{i,j})$ is a correct evaluation of the garbled circuit $\tgccir_{I,B}$ on the input $\gckey' = \smallset{\gckey[i,\hccir[i]]}_{i \in [S]}$, satisfying $\gceval(\gccir,\gckey') = \cir(\hccir) = \hccir(\hcmsg) = \hcout$.

\medskip
\subparagraph{Security:} 
Correctness for the decryption of the witness selector comes from the fact that:
\iftoggle{full}{We have the following theorem.}{We prove the following security theorem in \appref{}.}
\begin{theorem}
  \label{th:sec-hc-ws}
  If $\ot$ is correct, receiver-private, and (semi-honest) sender-private, then the functional commitment scheme $\hc$ defined above is correct, semi-honest functionally binding, and simulatable.
  Furthermore, the associated witness selector $\ws$ is correct and semantically secure.
\end{theorem}

\Fnote{was 2-Round Semi-Malicious MPC}
\label{sec:sm-mpc}


Our construction of semi-malicious 2-round MPC is very similar to our
construction of semi-honest 2-round MPC in \cref{sec:sh-mpc} with the
following two main differences: the functional commitment $\hc$ is
replaced by a stronger (semi-malicious) equivocable functional
commitment $\ehc$ and the inner MPC is supposed to be secure against
semi-malicious adversaries instead of just semi-honest adversaries.



\subsection{Equivocable FC with WS from 2-Round Semi-Malicious OT}
\label{sec:cons-ehc-ws}

To conclude the construction of semi-malicious 2-round MPC, we need to construct an equivocable functional commitment with witness selector from semi-malicious 2-round OT.

\subparagraph{Semi-Malicious 2-Round OT:} Let us first define the notion of semi-malicious 2-round OT.

\begin{mydefinition}
  A \emph{semi-malicious 2-round oblivious transfer (OT)} is a 2-round oblivious OT (see \cref{def:2ot}) satisfying the following additional property:
  \begin{description}
      \item[Semi-Malicious Sender Privacy:] The following two distributions are computationally indistinguishable:
    \begin{align*}
      \big\{ \otsendtwo(\otflowone,\otmsg_0,\otmsg_1) \ &: \ \otflowone = \otsendone(1^\secpar,\otsel;\otrandone) {\big\}}_{\secpar, \otsel, \otmsg_0, \otmsg_1,\otrandone} \enspace, \\
      \big\{ \otsendtwo(\otflowone,\otmsg_\otsel,\otmsg_\otsel) \ &: \ \otflowone = \otsendone(1^\secpar,\otsel;\otrandone) {\big\}}_{\secpar, \otsel, \otmsg_0, \otmsg_1,\otrandone} \enspace.
    \end{align*}
  \end{description}
\end{mydefinition}

We remark that semi-honest sender privacy is clearly implied by semi-malicious sender privacy. The only difference between the two notions is that the former notion just needs to hold when the first flow is honestly generated using a uniform random tape $\otrandone \getsr \bits^\otrandonelen$, while the latter one needs to hold for any random tape $\otrandone \in \bits^\otrandonelen$.

\subparagraph{Construction of Equivocable Functional Commitment:}
Let $\hccirclass = {\{\hccirclass_\secpar\}}_{\secpar \in \N}$ be a $S$-size circuit class (where $S$ is polynomial in $\secpar$).
To construct an equivocable functional commitment scheme $\ehc=(\ehccom,\ehcopen,\ehcver,\allowbreak\ehcsimc,\allowbreak\ehcsimd)$ with an associated witness selector $\ws = (\wsenc,\wsdec)$, we rely on the same tools as in \cref{sec:cons-hc-ws}, except that we suppose the 2-round OT to also be semi-malicious sender-private.

The construction is very similar to the one of \cref{sec:cons-hc-ws}.
The semi-malicious binding property easily follows from the semi-malicious sender privacy property of the OT.
The main difficulty is to add the equivocation property.
In the construction of \cref{sec:cons-hc-ws}, first flows of the OT protocol are used to commit to the input labels of a garbled circuit of $\univcirc(\star,\hcmsg)$.
The issue is that since the garbled circuit is in the clear and the input labels are committed (in a possibly statistically binding way), there is no way to do any equivocation.
The idea is to commit both the input labels and the garbled circuit of $\univcirc(\star,\hcmsg)$ in an equivocable way (and compatible with a witness selector): for each bit $\beta$ of the input labels and of the garbled circuit, we generate two first OT flows both for the selector bit~$\beta$.
The associated ``decommitment'' is the random tape used to generate the $(\beta+1)$-th (first) OT flow (i.e., the first one if $\beta=0$ and the second one if $\beta=1$).
In a simulated commitment generated by $\ehcsimc$, for each bit, the first (first) OT flow is generated for the selector bit~$0$, while the second (first) OT flow is generated is generated for the selector bit~$1$.
But a commitment generated by a semi-malicious adversary remains binding, as even a semi-malicious adversary has to use the same selector bit for both OTs.

More precisely, the construction is as follows:
\begin{description}
\item[Commitment:] $\hcc = \ehccom(1^\secpar, \hcmsg;\hcr)$ commits to $\hcmsg \in \bits^\hccirinlen$ as follows:
  \begin{enumerate}
  \item Generate input labels $\gckey \getsr \gcgen(1^\secpar)$ (using a random tape derived from $\hcr$).
  \item Garble $\cir = \univcirc_\secpar(\star,\hcmsg)$, which is the universal circuit partially evaluated on~$\hcmsg$: $\gccir \getsr \gcgarble(\gckey,\cir)$.
  \item For each $k \in [|\gccir|]$, for each bit $b' \in \bits$, generate a first flow $\otflowone_{k,b'} = \otsendone(1^\secpar,\gccir[k]; \allowbreak \otrandone_{k,b'})$, where $\gccir[k]$ is the $k$-th bit of the garbled circuit $\gccir$ (seen as a bitstring) and where the random tape $\otrandone_{k,b'}$ is derived from $\hcr$.
  \item For each $i \in [S]$, for each bit $b \in \bits$, for each $j \in [|\gckey[i,b]|]$, for each bit $b' \in \bits$, generate a first flow $\otflowone_{i,b,j,b'} = \otsendone(1^\secpar,\gckey[i,b]_j; \otrandone_{i,b,j,b'})$, where $\gckey[i,b]_j$ is the $j$-th bit of the input label $\gckey[i,b]$ and where the random tape $\otrandone_{i,b,j,b'}$ is derived from $\hcr$.
  \end{enumerate}
  And returns:
  \[ \hcc = (\smallset{\otflowone_{k,b'}}_{k,b'}, \smallset{\otflowone_{i,b,j,b'}}_{i,b,j,b'}) \enspace. \]
\item[Functional Opening:] $\hcd = \ehcopen(\hcc, \hccir, \hcmsg, \hcr)$ derives the functional decommitment $\hcd$ of $\hcc$ to $\hcout = \hccir(\hcmsg) = \univcirc_\secpar(\hccir, \hcmsg)$ as follows:
  \[ \hcd = (\gccir, \smallset{\otrandone'_{k}}_{k}, \smallset{\gckey'[i], \smallset{\otrandone'_{i,j}}_j}_{i \in [S]}) \enspace, \]
  where
    $\otrandone'_{k} = \otrandone_{k,\gccir[k]}$,
    $\gckey'[i] = \gckey[i,\hccir[i]]$, and 
    $\otrandone'_{i,j} = \otrandone_{i,\hccir[i],j,\gckey'[i]}$.
\item[Functional Verification:] $\ehcver(\hcc,\hccir,\hcout,\hcd)$ returns $1$, if and only if for all $i \in [S]$ and $j \in [|\gckey'[i]|]$ and $k \in [|\gccir|]$:
  \begin{align*}
    \otflowone_{k,\gccir[k]} &= \otsendone(1^\secpar,\gccir[k]; \otrandone'_{k})\enspace,
    \\
    \otflowone_{i,\hccir[i],j,\gckey'[i]_j} &= \otsendone(1^\secpar,\gckey'[i]_j; \otrandone'_{i,j})\enspace,\\
    \hcout &= \gceval(\gccir, \gckey')\enspace.
  \end{align*}
\item[Simulation:] $(\hcc,\ehctrap) \getsr \ehcsimc(1^\secpar)$ generates the simulated commitment $\hcc$ as $\ehccom$, except that:
  \begin{align*}
    \otflowone_{k,b'} &= \otsendone(1^\secpar,b'; \otrandone_{k,b'})
    && \text{ and }
    & \otflowone_{i,b,j,b'} &= \otsendone(1^\secpar,b'; \otrandone_{i,b,j,b'}) \enspace,
  \end{align*}
  i.e., the first flows $\otflowone_{k,b'}$ and $\otflowone_{i,b,j,b'}$ are for the selector bit~$b'$ instead of $\gccir[k]$ and $\gckey[i,b]_j$. The trapdoor $\ehctrap$ is the random tape.
\item[Equivocation:] $\hcd \getsr \ehcsimd(\hcc,\ehctrap,\hccir,\hcout)$ equivocate the commitment $\hcc$ by simulating $(\gckey', \gcsimcir) \getsr \gcsim(1^\secpar,\hcout)$ and then generating the functional decommitment $\hcd$ similarly to $\hcopen$ as follows:
    \[ \hcd = (\gcsimcir,\; \smallset{\otrandone'_{k}}_k,\; \smallset{\gckey'[i], \smallset{\otrandone'_{i,j}}_j}_{i \in [S]}) \enspace, \]
  where
    $\otrandone'_{k} = \otrandone_{k,\gcsimcir[k]}$,
    $\gckey'[i] = \gckey[i,\hccir[i]]$,
    $\otrandone'_{i,j} = \otrandone_{i,\hccir[i],j,\gckey'[i]}$.
\item[Encryption:] $\wsct \getsr \wsenc(1^\secpar, (\hcc,\hccir), \wsmsg)$ encrypts the messages $\wsmsg
  =\smallset{\wsmsg[I,B]}_{I,B}$ for $\ndax=(\hcc,\hccir)$ into $\wsct$ as follows: 
  \begin{enumerate}
  \item For every $I \in [\hcciroutlen]$ and $B \in \bits$, create the circuit:
    \[ \tcir_{I,B}(\gccir,\gckey') =
      \begin{cases}
        \wsmsg[I,B] &\text{if } \hcout_I = B \text{ where } \hcout = \gceval(\gccir,\gckey'), \\
        \perp &\text{otherwise.}
      \end{cases}
    \]
  \item For every $I \in [\hcciroutlen]$ and $B \in \bits$, garble this circuit: $\tgckey_{I,B} \getsr \tgcgen(1^\secpar)$ and $\tgccir_{I,B} \getsr \tgcgarble(\tgckey, \tcir_{I,B})$; we write $\tgckey_{I,B}[k,b]$ (resp., $\tgckey_{I,B}[i,j,b]$) the key corresponding to the $k$-th bit of the input $\gccir$ of $\tcir_{I,B}$ (resp., the $j$-th bit of the input $\gckey'[i]$ of $\tcir_{I,B}$) being~$b$.
  \item Define the OT messages: $\otmsg_{k,b} = \smallset{\tgckey_{I,B}[k,b]}_{I,B}$ and $\otmsg_{i,j,b} = \smallset{\tgckey_{I,B}[i,j,b]}_{I,B}$.
  \item Compute the second flows of the OT corresponding to the first flows $\otflowone_{k,b'}$ and $\otflowone_{i,\hccir[i],j,b'}$:
    \begin{align*}
      \otflowtwo_{k,0} &\getsr \otsendtwo(\otflowone_{k,0},\otmsg_{i,j,0},\perp)\enspace, \\  
      \otflowtwo_{k,1} &\getsr \otsendtwo(\otflowone_{k,1},\perp,\otmsg_{i,j,1})\enspace, \\
      \otflowtwo_{i,j,0} &\getsr \otsendtwo(\otflowone_{i,\hccir[i],j,0},\otmsg_{i,j,0},\perp)\enspace, \\  
      \otflowtwo_{i,j,1} &\getsr \otsendtwo(\otflowone_{i,\hccir[i],j,1},\perp,\otmsg_{i,j,1})\enspace,
    \end{align*}
    where $\perp$ is an arbitrary message.
  \end{enumerate}
  And return
  \[ \wsct = (\smallset{\tgccir_{I,B}}_{I \in [\hcciroutlen],B \in \bits}, \smallset{\otflowtwo_{k,b'}}_{k,b'},  \smallset{\otflowtwo_{i,j,b'}}_{i,j,b'})\enspace. \]
\item[Decryption:] $\wsmsg = \wsdec(\wsct, (\hcout,\hcd))$ decrypts $\wsct$ as follows:
  \begin{enumerate}
  \item For every $k$, compute:
    \[ \smallset{\tgckey'_{I,B}[k]}_{I,B} = \otmsg_{k,\gccir[k]} = \otoutput(\otflowtwo_{k,\gccir[k]}, \gccir[k], \otrandone'_{k}) \enspace. \]
  \item For every $i$ and $j$, compute:
    \[ \smallset{\tgckey'_{I,B}[i,j]}_{I,B} = \otmsg_{i,j,\gckey'[i]_j} = \otoutput(\otflowtwo_{i,j,\gckey'[i]_j}, \gckey'[i]_j, \otrandone'_{i,j}) \enspace. \]
  \item For every $I \in [\hcciroutlen]$ and $B = \hcout_I$, evaluate the garble circuit $\tgccir_{I,B}$:
    \[ \wsmsg[I,B] = \tgceval(\tgccir_{I,B},(\smallset{\tgckey'_{I,B}[k]}_k,\smallset{\tgckey'_{I,B}[i,j]}_{i,j})) \]
  \end{enumerate}
  and return $\wsmsg = \smallset{\wsmsg[I,\hcout_I]}_{I \in [\hcciroutlen]}$.
\end{description}

We have the following theorem.
\begin{theorem}
  If $\ot$ is correct, receiver-private, and semi-malicious sender-private, then the equivocable functional commitment scheme $\ehc$ defined above is correct, semi-malicious functionally binding, and simulatable.
  Furthermore, the associated witness selector $\ws$ is correct and semantically secure.
\end{theorem}

\begin{proof}
  As in the proof of \cref{th:sec-hc-ws}, correctness is straightforward and semi-malicious functional binding follows from the semantic security of the witness selector.
  Furthermore simulatability of $\ehc$ and semantic security of the witness selector $\ws$ can be proven similarly as in the proof of \cref{th:sec-hc-ws}.
\end{proof}



%%% Local Variables:
%%% mode: latex
%%% TeX-master: "../main"
%%% End:

\section{2-Round Malicious MPC}
\label{sec:malicious}

%%% Local Variables:
%%% mode: latex
%%% TeX-master: "../main"
%%% End:

% !TEX root =../main.tex


\section{Instantiations}
\label{sec:instantiations}

\subsection{Semi-Malicious OT}

In this section we demonstrate the feasibility of two-round adaptively secure semi-malicious UC oblivious transfer from any two-round malicious oblivious transfer with sender and receiver oblivious sampling. 


\BD [OT with sender and receiver oblivious sampling] A 2-round oblivious transfer (OT) protocol with sender and receiver oblivious sampling
is a 2-round
oblivious OT (see Definition \ref{def:2ot}) with the additional tuple of PPT algorithms $(\ootsendone,\ootsendtwo,\iotsendone,\iotsendtwo)$ satisfying the following additional properties:

\BI
\item {\bf Indistinguishability of Receiver's oblivious flow.} For any message $\otsel$, consider the experiment $\crs \getsr \OTSetup(1^\secpar)$, $\tilde\mu^1\getsr \ootsendone(\crs;\tilde\otrandone)$, $\otflowone = \otsendone(\crs,\otsel;\otrandone)$, $\tilde{\otrandone}' \getsr\iotsendone(\crs, \mu^1)$ then 
\[(\crs,\tilde\otrandone,\tilde\mu^1,\otsel)\compind(\crs,\tilde{\otrandone}',\otflowone,\otsel)\]

\item {\bf Indistinguishability of Sender's oblivious flow.} For any message $\otmsg$, consider the experiment $\crs \getsr \OTSetup(1^\secpar)$, $\tilde\mu^2\getsr \ootsendtwo(\crs;\tilde\zeta)$, $\otflowtwo = \otsendtwo(\crs,\otflowone,\otmsg;\zeta)$, $\tilde{\zeta}' \getsr\iotsendtwo(\crs, \mu^2)$
 then 
\[(\crs,\tilde{\zeta}',\tilde\mu^2,\otmsg)\compind(\crs,\zeta,\otflowtwo,\otmsg)\]
\Fnote{I'm not sure this definition is strong enough: I think we need to allow the adversary to choose $\mu^1$. $\mu^1$ might be maliciously generated, as even a semi-malicious adversary may claim it as ``obliviously sampled''. Now, I'm not sure there is any issue with re-usability of the first flow}

\EI

\ED


\anti{we need to add the extraction property as well to the OT}

\paragraph{Parallel oblivious transfer.} We consider a strengthening of $\ot$ in which M OT executions can run in parallel. We assume that m is a fixed, possibly a-priori bounded, polynomial
in the security parameter $\secpar$.\anti{update this}
\Fnote{Actually, now, I'm not sure this is an issue. See comment above}

%We consider a strengthening of $\ot$ in which a maliciously generated receiver OT flow can be reused in M OT sender flows. We assume that m is a fixed, possibly a-priori bounded, polynomial in the security parameter $\secpar$

\BD[Non-committing encryption] A non-committing (bit)
encryption scheme consists of a tuple $(\NCGen, \NCEnc, \NCDec, \NCSim)$ where
$(\NCGen, \NCEnc, \NCDec)$ is an encryption scheme and $\NCSim$ is the simulation satisfying the following property: for $b\in\bits$ 
 the following distributions are computationally indistinguishable:
     \begin{align*}
    \big\{(\pk, c, \sigma_G, \sigma_E) : (\pk, \sk) \getsr \NCGen(1^\secpar; \sigma_G),& c = \NCEnc_\pk(b; \sigma_E)\big\}_{\secpar,b}   \enspace, \\
       \big\{(\pk, c, \sigma^b_G, \sigma^b_E) :  (\pk, c, \sigma^0_G, \sigma^0_E,  \sigma^1_G,\sigma^1_E)&\getsr\NCSim(1^\secpar) \big\}_{\secpar,b}\enspace.
    \end{align*}
 

 
\ED
\Mnote{modified theorem statement below. }
\BT\label{thm:comp1}

%Assuming the existence of a two-round extractable malicious oblivious transfer protocol with sender oblivious sampling in the CRS model, there exists a two-round malicious oblivious transfer protocol with sender equivocality in the CRS model. 
Assume the existence of injective one-way functions and two-round oblivious transfer with the following properties:
\BI
\item UC-Security against static corruption of the receiver by an active adversary. 
\item Oblivious sampleability of sender's algorithm. 
\EI
Then there exists a two-round oblivious transfer protocol with the following properties:
\BI
\item UC-Security against static corruption of the receiver by an active adversary and adaptive corruption of the sender by a passive adversary. 
\EI
Additionally, the compilation preserves oblivious sampleability of the receiver if the original protocol satisfies it. \Mnote{I think we also need to say that if the original OT achieves semi-malicious sender security, then the compiler preserves it.}
\ET

\noindent\textbf{Intuition:} \Mnote{Write high-level description here:}


\Mnote{Instead of $O_2$, use $Obl$, and $O_2^{-1}$ use $InvObl$}


{\small\BPR [Malicious $\qot$ protocol with sender equivocality]\label{prot:compiler1}
Protocol $\qot = (\qotsetup,\qotsendone,\qotsendtwo,\qotoutput,\qotequivS)$ is run between sender S and receiver R and uses a two-round extractable malicious oblivious transfer protocol $\eot = (\eotsetup,\eotsendone,\eotsendtwo,\eotoutput,\eootsendtwo,\iotsendtwo,\eotextr)$ with sender oblivious sampling and a non-interactive equivocal commitment scheme $\qcom=(\qcomsetup,\qqcomsetup,\qcomcom,\qcomopen,\qcomsim,\qcomequiv)$.
\BI
  \item \textsc{Inputs:} S holds two strings $\otmsg_0, \otmsg_1 \in \bits^\secpar$ and R holds a bit $\otsel$.
    
   \item \textsc{Setup} $\qotsetup(1^\secpar)$:
  
  \BI
  \item[] Run $\crs' \getsr \eotsetup(1^\secpar)$, $\ck \getsr \qcomsetup(1^\secpar)$ and set $\crs=(\crs',\ck)$. 
  \EI
  
  \item\textsc{Setup}  $\qOTsetup(1^\secpar)$:
  
  \BI
    \item Run  $(\crs', \tex) \getsr \eotsetup(1^\secpar)$ and $(\ck, \teq) \getsr \qqcomsetup(1^\secpar)$ and set $\crs=(\crs',\ck)$
      \EI

    
   \item \textsc{First Round} $\qotsendone(\crs,\otsel;\otrandone)$: 
 \BI
 \item [] Execute $\otflowone = \eotsendone(\crs,\otsel;\otrandone)$ and output $\otflowone$.
 \EI
    \item \textsc{Second Round} $\qotsendtwo(\crs,\otflowone,\otmsg_0,\otmsg_1;\{r_b,s_b,\zeta_{b,\delta},\tilde\zeta_{b,\delta}\}_{b,\delta\in\bits})$:
  \BE
  \item Generate $\com_0 = \qcomcom(\otmsg_0;r_0), \com_1 = \qcomcom(\otmsg_1;r_1)$;
  \item Generate $\delta_0 \getsr \bits$. For $\delta \in \bits$, generate $\otflowtwo_{0,\delta}$ as follows: 
    \[
      \otflowtwo_{0,\delta} = \begin{cases}
        \eotsendtwo(\crs,\otflowone,\otmsg_0||r_0,s_0 ;\zeta_{0,\delta}), s_0 \getsr\bits^\secpar & \text{if } \delta=\delta_0 \\
        \eootsendtwo(\crs,\tilde\zeta_{0,\delta})                                  & \text{otherwise}      %
      \end{cases}
    \]
  \item Generate $\delta_1 \getsr \bits$. For $\delta \in \bits$, generate $\otflowtwo_{1,\delta}$ as follows: 
    \[
      \otflowtwo_{1,\delta} = \begin{cases}
        \eotsendtwo(\crs,\otflowone,s_1,\otmsg_1||r_1 ;\zeta_{1,\delta}), s_1 \getsr\bits^\secpar  & \text{if } \delta=\delta_1 \\
        \eootsendtwo(\crs,\tilde\zeta_{1,\delta})                                  & \text{otherwise}      %
      \end{cases}
    \]
   

       
       \item Output the message $(\com_0,\com_1,\smallset{\otflowtwo_{0,\delta},\otflowtwo_{1,\delta}})$.
  \EE

 
  
  \item    \textsc{Output} $\qotoutput((\com_0,\com_1,\smallset{\otflowtwo_{0,\delta},\otflowtwo_{1,\delta}}),\otsel,\otrandone)$: 
   \BE
  \item Run $\eotoutput(\star, \otsel,\otrandone)$ for messages $\smallset{\otflowtwo_{\otsel,\delta}}_{\delta\in\bits}$. Let the outputs be $y_1,y_2$. Output $\otmsg$ for which $y_i=\otmsg||r$ for some $i\in\bits$ and $\com_\otsel = \qcomcom(\otmsg;r)$.


  \EE

  \item \textsc{Sender Equivocation} $\qotequivS_1({\crs},{\teq},\otflowone,\otsel,\otmsg_\otsel;\{r_b,\rho_b,s_b,s_{b,\delta},\zeta_{b,\delta},\tilde\zeta_{b,\delta},\zeta_{c,b,\delta},\zeta'_{c,b,\delta}\}_{b,c,\delta\in\bits,})$:
    
    \BE 
    
  \item Generate $\com_\otsel = \qcomcom(\otmsg_\otsel;r_\otsel)$ and $(\com_{1-\otsel},\state^\cc) \getsr \qcomsim(\crs,\teq)$.
  \item Generate $\delta_0 \getsr \bits$. For $\delta \in \bits$, generate $\otflowtwo_{0,\delta}$ as follows: 
    \[
      \otflowtwo_{0,\delta} = \begin{cases}
        \eotsendtwo(\crs,\otflowone,\otmsg_\otsel||r_0,s_0 ;\zeta_{0,\delta}), s_0 \getsr\bits^\secpar & \text{if } \delta=\delta_0,\ \otsel=0 \\
        \eootsendtwo(\crs,\tilde\zeta_{0,\delta})                                        & \text{if } \delta \neq \delta_0,\ \otsel=0 \\
        \eotsendtwo(\crs,\otflowone,0||\rho_0,s_{0,0} ;\zeta_{0,0,\delta}), s_{0,0} \getsr\bits^\secpar     & \text{if } \delta=\delta_0,\ \otsel=1 \\
        \eotsendtwo(\crs,\otflowone,1||\rho_1,s_{0,1} ;\zeta'_{0,0,\delta}), s_{0,1} \getsr\bits^\secpar     & \text{if } \delta\neq\delta_0,\ \otsel=1 \\
      \end{cases}
    \]
  \item Generate $\delta_1 \getsr \bits$. For $\delta \in \bits$, generate $\otflowtwo_{1,\delta}$ as follows: 
    \[
      \otflowtwo_{1,\delta} = \begin{cases}
        \eotsendtwo(\crs,\otflowone,s_1,\otmsg_\otsel||r_1 ;\zeta_{1,\delta}), s_1 \getsr\bits^\secpar  & \text{if } \delta=\delta_1,\ \otsel=1 \\
        \eootsendtwo(\crs,\tilde\zeta_{1,\delta})                                        & \text{if } \delta \neq \delta_1,\ \otsel=1 \\
        \eotsendtwo(\crs,\otflowone,s_{1,0},0||\rho_0 ;\zeta_{1,0,\delta}), s_{1,0} \getsr\bits^\secpar     & \text{if } \delta=\delta_1,\ \otsel=0 \\
        \eotsendtwo(\crs,\otflowone,s_{1,1},1||\rho_1 ;\zeta'_{1,0,\delta}), s_{1,1} \getsr\bits^\secpar     & \text{if } \delta\neq\delta_1,\ \otsel=0 \\
      \end{cases}
    \]
 

    
       \item Output the message $(\com_0,\com_1,\smallset{\otflowtwo_{0,\delta},\otflowtwo_{1,\delta}})$.
    
    \EE
    
    
 
  


\EI


  \EPR
}


\begin{proof}[Proof of \cref{thm:comp1}] Completeness follows directly. %UC-security against static corruption by an active adversary of $\qot$ also follows from the corresponding property of the $\eot$ protocol. \Mnote{Write something about how sender message is simulated and that the same receiver OT flow is used four times.}
Without loss of generality, the only corruption scenario to consider here is an adversary that actively corrupts the receiver at the beginning of the execution and passively corrupts the sender post-execution.
\Mnote{Prove UC-security of this scenario. At the end mention, how the compiler preserves oblivious receiver sampleability and semi-malicious sender security.}

\subparagraph{Sender Equivocation:} We need to prove that for any polynomial-time circuit family $\advA =\smallset{\advA_\secpar}_{\secpar \in \N}$,
    there exists a negligible function $\negl$, such that for any $\secpar \in N$ and for any $\otsel \in
    \bits$:
    \begin{multline*}
      \Bigg| \Pr\bigg[
      \advA_\secpar(\state,\crs,\otflowtwo,\zeta) = 1
      \ : \
      \begin{array}{l}
        \ \crs \getsr \qotsetup(1^\secpar); \\
        (\state,\otmsg_0,\otmsg_1,\otflowone) \getsr \advA_\secpar(\crs); \ \zeta \getsr \bits^{\poly(\secpar)};\\  \otflowtwo=\qotsendtwo(\crs,\otflowone,\otmsg_0,\otmsg_1;\zeta)
              \end{array}
      \bigg] - \\
      \Pr\bigg[
      \advA_\secpar(\state,\crs,\otflowtwo,\zeta') = 1
      \ : \
      \begin{array}{l}
       (\crs,\teq) \getsr \qOTsetup(1^\secpar); \\
        (\state,\otmsg_0,\otmsg_1,\otflowone) \getsr \advA_\secpar(\crs);\ \\
        \otflowtwo\getsr\qotequivS_1({\crs},{\teq},\otflowone,\otsel,\otmsg)\\
      \zeta' = \qotequivS_2({\crs},{\teq},\otflowone,\otflowtwo,\otsel,\otmsg_\otsel,\otmsg_{1-\otsel})
      \end{array}
      \bigg]
      \Bigg| \le \negl(\secpar) \enspace.
    \end{multline*}
    \anti{update this def using the Extr and have the state}

    
    
    

For that, we consider the following hybrid distributions where $\hyb{0}$ consider the real execution: 

\paragraph{$\hyb{1}$:} In the first hybrid, the challenger extracts the input $\otsel$ of the receiver running the extraction algorithm $ (\otsel;\otrandone) \getsr\eotextr(\crs,\otflowone)$. 

\begin{lemma}
$\hyb{0}\compind \hyb{1}$.
\label{lemma:hyb1}
\end{lemma}
\begin{proof}
The indistinguishability of $\hyb{1}$ from $\hyb{0}$ follow from the extractability property of $\eot$. 
\end{proof}

\paragraph{$\hyb{1}$:} In this hybrid we change the way the public parameters of the OT are generated. In particular, the challenger runs the setup algorithm $(\crs,\teq) \getsr \qOTsetup(1^\secpar)$ and stores trapdoors $(\teq,\tex)$.

\anti{update the numbering of the hybrids}

\begin{lemma}
$\hyb{0}\compind \hyb{1}$.
\label{lemma:hyb1}
\end{lemma}
\begin{proof}
The indistinguishability of $\hyb{1}$ from $\hyb{0}$ follows from the indistinguishability of public parameters of $\eot$. 
\end{proof}



\paragraph{$\hyb{2}$:} In this hybrid we change the way the challenger generates the commitments ob behalf of the sender. In particular, the challenger generates:  

\[\com_\otsel = \qcomcom(\otmsg_\otsel;r_\otsel),(\com_{1-\otsel},\state^\cc) \getsr \qcomsim(\crs,\teq)\]. 


% = \qcomcom(0;\rho_0)= \qcomcom(1;\rho_1)$.\anti{update this}




\begin{lemma}
$\hyb{1}\compind \hyb{2}$.
\label{lemma:hyb1}
\end{lemma}
\begin{proof}
The indistinguishability of $\hyb{1}$ from $\hyb{2}$ follow from the computational hiding of the commitment scheme $\com$.
\end{proof}


\paragraph{$\hyb{2}$:} This hybrid is similar to $\hyb{1}$ except that the sender's messages are simulated: 

\[\smallset{\otflowtwo_{0,\delta},\otflowtwo_{1,\delta}}\getsr\qotequivS_1({\crs},{\teq},\otflowone,\otsel,\otmsg)\]


\begin{lemma}
$\hyb{1}\compind \hyb{2}$.
\label{lemma:hyb1}
\end{lemma}
\begin{proof}
The indistinguishability of $\hyb{1}$ from $\hyb{2}$ follow from the sender privacy of $\eot$.
 \anti{do them 1 by 1}\end{proof}


\paragraph{$\hyb{2}$:}In this hybrid, we change how the internal randomness of the corrupted party is explained to $\cA$ on being adaptively corrupted. Specifically we change the randomness that is used to explain the commitment $\cS$ generates on behalf of parties. In particular,  upon corruption the challenger receives the real input $(\otmsg_0,\otmsg_1)$ and computes $\rho_{1-\otsel}\getsr \qcomequiv ({\ck},{\teq},\com,\state^\cc, \otmsg_{1-\otsel})$ such that $\com_{1-\otsel} = \qcomcom(\otmsg_{1-\otsel};\rho_{1-\otsel})$.

\begin{lemma}
$\hyb{1}\compind \hyb{2}$.
\label{lemma:hyb1}
\end{lemma}
\begin{proof}
The indistinguishability of $\hyb{1}$ from $\hyb{2}$ follow from the equivocation property of the commitment scheme.\end{proof}


\paragraph{$\hyb{2}$:} This hybrid is similar to  $\hyb{1}$ except that upon corruption the challenger receives the real input $(\otmsg_0,\otmsg_1)$. In this case the challenger prepares the internal state on behalf of the sender such that it will be consistent with the OT messages $\smallset{\otflowtwo_{0,\delta},\otflowtwo_{1,\delta}}$ that it had provided to $\cA$ earlier. In particular, the challenger runs 
$\zeta'_\otsel\getsr\qotequivS_2({\crs},{\teq},\smallset{\otflowtwo_{0,\delta},\otflowtwo_{1,\delta}},\otsel,\otmsg_\otsel,\otmsg_{1-\otsel})$. \anti{Move the rest in Protocol 2}. Set $\zeta'_\otsel=\{u_{0,\delta},u_{1,\delta}\}_{\delta\in\bits}\cup \{\delta_0,\delta_1\}$ where 


  \[
      u_{0,\delta} = \begin{cases}
      (s_0,\zeta_{0,\delta}) & \text{if } \delta=\delta_0,\ \otsel=0 \\
       \zeta_{0,\delta}                                    & \text{if } \delta \neq \delta_0,\ \otsel=0 \\
         \iotsendtwo(\alpha) & \text{if } \delta=\delta_0,\ \otsel=1 \\
        ( 1||\rho_1,\zeta'_{0,0,\delta})     & \text{if } \delta\neq\delta_0,\ \otsel=1 \\

      \end{cases}
    \]

\anti{where $\alpha=\eotsendtwo(\crs,\otflowone,0||\rho_0,s_{0,1} ;\zeta'_{0,0,\delta})$}
    \[
      u_{1,\delta} = \begin{cases}
        
          (s_1,\zeta_{1,\delta}) & \text{if } \delta=\delta_1,\ \otsel=1 \\
        \tilde\zeta_{1,\delta}                                        & \text{if } \delta \neq \delta_1,\ \otsel=1 \\
        (0||\rho_0, \zeta_{1,0,\delta}) & \text{if } \delta=\delta_1,\ \otsel=0  \\
        \iotsendtwo(\beta) & \text{if } \delta\neq\delta_1,\ \otsel=0 \\
      \end{cases}
    \]
    
    \anti{where $\beta=\eotsendtwo(\crs,\otflowone,s_{1,1},1||\rho_1 ;\zeta'_{1,0,\delta}))$}





\begin{lemma}
$\hyb{2}\compind \hyb{3}$.
\label{lemma:hyb1}
\end{lemma}
\begin{proof}
The indistinguishability of $\hyb{2}$ from $\hyb{3}$ follow from the sender oblivious sampling property of $\eot$\anti{add details}.
\end{proof}



\end{proof}

\Mnote{Write it like prev thm.}\Mnote{Add injective OWFs. Also remove NCE. Just say we assume a private channel from sender to the receiver and remark that this can be easily implemented via NCE where the receiver provides a public-key in the first round. Remark that we dont have a NCE from Receiver to Sender because that will cost another round.}
\BT\label{comp2}

Assuming the existence of a two-round parallel and extractable malicious oblivious transfer protocol with receiver oblivious sampling and sender equivocality in the CRS model, there exists a two-round semi-malicious adaptive oblivious transfer protocol in the CRS model. 
\ET




{\small\BPR [ Semi-Malicious $\ot$ protocol with receiver and sender equivocality]\label{prot:compiler2}
Protocol $\ot = (\OTSetup,\otsendone,\otsendtwo,\otoutput,\ROTequiv,\SOTequiv)$ is run between sender S and receiver R and uses a two-round extractable malicious OT protocol with receiver and sender oblivious sampling and sender equivocality $\qot = (\qotsetup,\qotsendone,\qotsendtwo,\qotoutput,\obothsend,\iobothsend,\qotextr,\qotequivS)$, a non-committing encryption scheme $\NC=(\NCGen, \NCEnc, \NCDec, \NCSim)$, a garbled circuit scheme $\gc=(\gcgen,\gcgarble,\gceval,\gcsim)$ for the class of all
  $\cirsize$-size circuits with a sufficiently large polynomial bound $\cirsize$ and a non-interactive equivocal commitment scheme $\qcom=(\qcomsetup,\qqcomsetup,\qcomcom,\qcomopen,\qcomsim,\qcomequiv)$.   
  \BI
  \item \textsc{Inputs:} S holds two strings $\otmsg_0, \otmsg_1 \in \bits^\secpar$ and R holds a bit $\otsel$.
    
     \item \textsc{Setup} $\OTSetup(1^\secpar)$:
  
  \BI
  \item[] Run $\crs' \getsr \qotsetup(1^\secpar)$, $\ck \getsr \qcomsetup(1^\secpar)$ and set $\crs=(\crs',\ck)$. 
  \EI
  
  \item\textsc{Setup}  $\OTSetupq(1^\secpar)$:
  
  \BI
    \item Run $(\crs', \tex) \getsr \qotsetup(1^\secpar)$, $(\crs', \teq)\getsr \qOTsetup(1^\secpar)$ and $(\ck, \cteq) \getsr \qqcomsetup(1^\secpar)$ and set $\crs=(\crs',\ck)$.      \EI

    
    
    
   \item \textsc{First Round} $\otsendone(\crs,\otsel;\{\otrandone_\delta,\tilde\otrandone_\delta\}_{\delta\in\bits},r)$: 
 \BE
   \item Generate $\com = \qcomcom(\otsel;r)$;
 %  \item Choose a random subset $S \subseteq [4\secpar]$ of size $\secpar$.
   \item For $\delta \in \bits$, generate $\otflowone_{\delta}$ as follows: 

    \[\otflowone_\delta= \begin{cases} \qotsendone(\crs,r;\otrandone_\delta) & \text{if } \delta=\otsel \\
                      \qootsendone(\crs;\tilde\otrandone_\delta)                                  & \text{otherwise}      %
        \end{cases}\]

   \Mnote{Expand OTs since $r$ will be a string}
   \item Generate key pair $(\sk,\pk) \getsr \NCGen(1^\secpar)$. 
   
 \item Output $(\pk,\com,\{\otflowone_\delta\}_{\delta\in\bits})$.
 \EE
    \item \textsc{Second Round} $\otsendtwo(\crs,\{\otflowone_\delta\}_{\delta\in\bits},\otmsg_0,\otmsg_1;r,r_E,\{\zeta_{\delta,j}\}_{\delta\in\bits,j\in[\cirsize]})$:
  \BE
  
 % \item Choose a random subset $T \subseteq [4\secpar]$ of size $\secpar$.
  \item For $\delta \in \bits$ generate circuit $\cir_\delta$ as follows: 
  
    \[ \cir_\delta(r) = \big\{  \text{output}~ \otmsg_\delta ~~\text{if } \com = \qcomcom(\delta;r) \big\}
     \]
    
 and generate the garble circuit $\giccir_\delta$, $\{\keys^\delta_j\}_{j\in[\cirsize]} \getsr \gcgen(1^\secpar)$: 
  \[ \giccir_\delta \getsr \gcgarble(\{\keys^\delta_j\}_j, \cir_\delta) \]
  
  \item For $\delta \in \bits$, generate $\otflowtwo_{\delta,j}$ for $j\in[\cirsize]$ as follows: 
  
      \[\otflowtwo_{\delta,j} =  \qotsendtwo(\crs,\otflowone_\delta,\keys^\delta_j;\zeta_{\delta,j}) 
           %          \qootsendtwo(\crs;\zeta_{\delta,j})                                & \text{otherwise}    %
        \]

  
\item Generate and output $ct=\NCEnc_\pk(\{\giccir_\delta ,\otflowtwo_{\delta,j}\}_{\delta\in\bits,j\in[\cirsize]};r_E)$. 
       
  \EE


  \item \textsc{Output} $\otoutput(\ct,\otsel,r,\{\rho_i\}_{i\in\bits})$: 
  
  \BE
  \item Decrypt $\{\giccir_i ,\otflowtwo_{i,j}\}_{i\in\bits,j\in[\cirsize]}\}=\NCDec_\sk(ct)$. 
  \item Receive the garble labels $\{\keys^i_{j}[r]\}_{j\in[\cirsize]}\getsr\qotoutput(\otflowtwo_{i,j}\}_{i\in\bits,j\in[\cirsize]}, \otsel,\rho_i,r)$. 
  \item Evaluate the garble circuits $x_i = \gceval(\gccir_i, \keys^i_{j}[r])$. 
  \item If $x_i \neq \perp$ for an $i\in\bits$ then output $x_i$ \Mnote{, otherwise output the default value}. 
  \EE

   \item \textsc{Receiver Equivocation} $\ROTequiv_1({\crs},{\teq};\{\otrandone_\delta\}_{\delta\in\bits})$
   \BE
  \item Generate $\com \getsr \qcomsim(\crs,\teq)$.
  \item For $\delta \in \bits$, compute $r_\delta \gets \qcomequiv(\crs,\teq,\com,\delta)$.
 %\item  Choose two random disjoint subsets $S_0,S_1 \subseteq [4\secpar]$ of size $\secpar$.
    \item  For $\delta \in \bits$, generate $\otflowone_\delta$ as follows: 
    
    \[\otflowone_\delta=\qotsendone(\crs,r_\delta;\otrandone_\delta)\]
\EE
   \item \textsc{Receiver Equivocation} $\ROTequiv_2({\crs},{\teq},\otsel,r_\otsel)$
   \BE

    \item For $\delta \in \bits$,  generate $\{\zeta_{\delta}\}_{\delta\in\bits}$ as follows

    \[\zeta_\delta= \begin{cases} r_\otsel;\otrandone_\otsel & \text{if } \delta=\sigma \\
   \qiotsendone(  \crs,\mu^1_{1-\sigma} )& \text{otherwise} \\
       %             \tilde\rho_i                             & \text{otherwise}      %
        \end{cases}\]
        
    %    \item Choose $T$ such that $T\cap S_{1-\otsel}=\emptyset$
\EE


\EI


   
   
   
    \EPR
}


    \subparagraph{Receiver Equivocation:} We need to prove that for any polynomial-time circuit family $\advA =\smallset{\advA_\secpar}_{\secpar \in \N}$,
    there exists a negligible function $\negl$, such that for any $\secpar \in N$ and for any $\otsel \in
    \bits$:
    \begin{multline*}
      \Bigg| \Pr\bigg[
      \advA_\secpar(\state,\crs,\otflowone,\otrandone) = 1
      \ : \
      \begin{array}{l}
        \ \crs \getsr \qotsetup(1^\secpar); \\
        (\state,\otsel) \getsr \advA_\secpar(\crs); \ \otrandone \getsr \bits^{\poly(\secpar)};\\  \otflowone = \otsendone(\crs,\otsel;\otrandone)
              \end{array}
      \bigg] - \\
      \Pr\bigg[
      \advA_\secpar(\state,\crs,\otflowone,\otrandone') = 1
      \ : \
      \begin{array}{l}
       (\crs,\teq) \getsr \qOTsetup(1^\secpar); \\
        (\state,\otsel) \getsr \advA_\secpar(\crs);\ \\
        \otflowone\getsr\ROTequiv_1({\crs},{\teq})\\
     \otrandone' \getsr\ROTequiv_2({\crs},{\teq},\otflowone,\otsel)
      \end{array}
      \bigg]
      \Bigg| \le \negl(\secpar) \enspace.
    \end{multline*}
    
    \anti{make them qOT}

For that, we consider the following hybrid distributions where $\hyb{0}$ consider the real execution:


 \anti{update the numbering of the hybrids}
\paragraph{$\hyb{1}$:} In this hybrid we change the way the public parameters of the OT are generated. In particular, the challenger runs:
$(\crs', \teq)\getsr \qOTsetup(1^\secpar)$ and $(\ck, \cteq) \getsr \qqcomsetup(1^\secpar)$ and stores $(\teq, \cteq)$. 


\begin{lemma}
$\hyb{0}\compind \hyb{1}$.
\label{lemma:hyb1}
\end{lemma}
\begin{proof}
The indistinguishability of $\hyb{1}$ from $\hyb{0}$ follows from the indistinguishability of public parameters of $\qot$ and commitment $\qcom$. 
\end{proof}


\paragraph{\bf Receiver is honest at the onset of the protocol.} 

\paragraph{$\hyb{2}$:} In this hybrid we change the way the challenger generates the commitment on behalf of the sender. The challenger generates $\com \getsr \qcomsim(\crs,\teq)$.  



% = \qcomcom(0;\rho_0)= \qcomcom(1;\rho_1)$.\anti{update this}




\begin{lemma}
$\hyb{1}\compind \hyb{2}$.
\label{lemma:hyb1}
\end{lemma}
\begin{proof}
The indistinguishability of $\hyb{1}$ from $\hyb{2}$ follow from the computational hiding of the commitment scheme $\com$.
\end{proof}


\paragraph{$\hyb{2}$:} This hybrid is similar to $\hyb{1}$ except that the receiver's messages are simulated as follows:  
$\{\otflowone_i\}_{i\in[4\secpar]}\getsr\ROTequiv_1({\crs},{\teq})$.


\begin{lemma}
$\hyb{1}\compind \hyb{2}$.
\label{lemma:hyb1}
\end{lemma}
\begin{proof}
The indistinguishability of $\hyb{1}$ from $\hyb{2}$ follow from the receiver privacy of $\qot$.
 \anti{do them 1 by 1}\end{proof}

\paragraph{\bf Receiver gets corrupted at the end of the protocol:} 



\paragraph{$\hyb{2}$:}In this hybrid, we change how the internal randomness of the corrupted party is explained to $\cA$ on being adaptively corrupted. Upon corruption the challenger receives the real input $\otsel$ and computes $r'\getsr \qcomequiv ({\ck},{\teq},\com,\state^\cc, \otsel)$ such that $\com \getsr \qcomcom(\otsel)$.

\begin{lemma}
$\hyb{1}\compind \hyb{2}$.
\label{lemma:hyb1}
\end{lemma}
\begin{proof}
The indistinguishability of $\hyb{1}$ from $\hyb{2}$ follow from the equivocation property of the commitment scheme.\end{proof}


\paragraph{$\hyb{2}$:} This hybrid is similar to  $\hyb{1}$ except that challenger prepares the internal state on behalf of the receiver such that it will be consistent with $\{\otflowone_i\}_{i\in[4\secpar]}$. That said, the challenger runs $\zeta'_\otsel\getsr\ROTequiv_2({\crs},{\teq},\otsel)$.



\begin{lemma}
$\hyb{2}\compind \hyb{3}$.
\label{lemma:hyb1}
\end{lemma}
\begin{proof}
The indistinguishability of $\hyb{2}$ from $\hyb{3}$ follow from the receiver oblivious sampling property of $\qot$\anti{add details}.
\end{proof}

\paragraph{\bf Receiver is corrupted and Sender is honest at the onset of the protocol:} 

\paragraph{$\hyb{1}$:} In this hybrid, the challenger extracts the input $\otsel$ of the receiver running the extraction algorithm $ (\otsel||r;\otrandone) \getsr\eotextr(\crs,\otflowone)$. 

\begin{lemma}
$\hyb{0}\compind \hyb{1}$.
\label{lemma:hyb1}
\end{lemma}
\begin{proof}
The indistinguishability of $\hyb{1}$ from $\hyb{0}$ follow from the extractability property of $\eot$. 
\end{proof}

\paragraph{$\hyb{2}$:} In this hybrid we change the way messages $\otflowtwo_{i,j,b}$ are generated. The challenger uses twice the input labels that correspond to the actual inputs $\mathsf{inp}=\otsel||r$, instead of using both possible input labels: 

\[\otflowtwo_{i,j}=\qotsendtwo(\crs,\otflowone_i,(\keys^i_j[\mathsf{inp}_k],\keys^i_j[\mathsf{inp}_k]);\zeta_{i,j})\]

where $k\in[|\mathsf{inp}|]$.


\begin{lemma}
$\hyb{0}\compind \hyb{1}$.
\label{lemma:hyb1}
\end{lemma}
\begin{proof}
The indistinguishability of $\hyb{1}$ from $\hyb{0}$ follow from the sender privacy of $\qot$.  
\end{proof}

\paragraph{$\hyb{2}$:} This hybrid is identical to the previous hybrid except that the challenger simulates the garble circuits: 
\[(\keys,\giccir_i)\getsr \gcsim(1^\secpar, x_i)~ \text{where} ~ x_i=\cir_i(\otsel,r) \]







\begin{lemma}
$\hyb{0}\compind \hyb{1}$.
\label{lemma:hyb1}
\end{lemma}
\begin{proof}
The indistinguishability of $\hyb{1}$ from $\hyb{0}$ follow from the fact tha $\gc$ is simulatable.  
\end{proof}

\paragraph{\bf Sender gets corrupted at the end of the protocol:} 

\paragraph{$\hyb{2}$:} This hybrid is similar to  $\hyb{1}$ except that challenger prepares the internal state on behalf of the sender such that it will be consistent with $\{\otflowtwo_{i,j}\}_{i\in[4\secpar],j\in[\cirsize]}$. That said, the challenger runs: $\zeta'_\otsel\getsr\qotequivS({\crs},{\teq},\otflowone_i,\otsel,\otmsg_\otsel)$. 

\begin{lemma}
$\hyb{0}\compind \hyb{1}$.
\label{lemma:hyb1}
\end{lemma}
\begin{proof}
The indistinguishability of $\hyb{1}$ from $\hyb{0}$ follow from the sender equivocation property of $\qot$. 
\end{proof}

\paragraph{$\hyb{2}$:} This hybrid is similar to $\hyb{1}$ except that the challenger explains randomness for the garble circuits:

\[( \keys^i,\psi_i) \getsr \gcequiv( 
    \gccir_i,\cirin_i, \state)\]
    
\begin{lemma}
$\hyb{0}\compind \hyb{1}$.
\label{lemma:hyb1}
\end{lemma}
\begin{proof}
The indistinguishability of $\hyb{1}$ from $\hyb{0}$ follow from the equivocation property of $\gc$. 
\end{proof}


\anti{rest is not relevant}

\paragraph{$\hyb{2}$:} This hybrid is similar to hybrid $\hyb{1}$, except that the ciphertext $ct$ are simulated: $ct \getsr \NCSim(1^\secpar)$.


\begin{lemma}
$\hyb{1}\compind \hyb{2}$.
\label{lemma:hyb1}
\end{lemma}
\begin{proof}
The indistinguishability of $\hyb{1}$ from $\hyb{2}$ follows from the semantic security of $\NC$. 
\end{proof}


%\subsection{NIZK}

%\subsection{Commitment}

%%% Local Variables:
%%% mode: latex
%%% TeX-master: "../main"
%%% End:


\subsubsection{Construction of 2-Round Receiver-Extractable Semi-Malicious Oblivious Transfer with Sender and Receiver Oblivious Sampling.}
\label{sec:ot-hps}


\Fnote{Name needs to be changed everywhere else}

We now construct 2-round receiver-extractable semi-malicious oblivious transfer protocols with sender and receiver oblivious sampling from IND-CPA encryption schemes and smooth projective hash functions~\cite{EC:CraSho02} (SPHFs) with the following additional properties: ciphertexts of the encryption scheme and projection keys of the SPHF can be obliviously sampled.
Our DDH-based construction is a variant of the one of Naor and Pinkas~\cite{SODA:NaoPin01} (later generalized by Halevi and Kalai~\cite{JC:HalKal12}), where some parameters are generated by a trusted party and written in the CRS, instead of being generated by the sender (in order to allow extraction of the selection bit of the receiver).
Our QR-based construction uses in addition a trick from~\cite{C:HofKil09} to allow for oblivious sampling, without requiring to perform square roots (which requires knowing the factorization of the modulus used in the construction).

Let us first define (IND-CPA) encryption schemes with ciphertext oblivious sampling and SPHFs with projection key oblivious sampling.

\begin{definition}[Encryption with Ciphertext Oblivious Sampling]
  An \emph{encryption scheme with ciphertext oblivious sampling} is a tuple of five polynomial-time algorithms $(\keygen,\enc,\enco,\encoinv,\dec)$:
  \begin{description}
  \item[Key generation:] $(\pk,\sk) \getsr \keygen(1^\secpar)$ generates a public/secret key pair $(\pk,\sk)$;
  \item[Encryption:] $\ct = \enc(\pk,m;r)$ encrypts a message $m \in \mathcal{M}$ under $\pk$ into a ciphertext $\ct$, using random tape $r$;
  \item[Oblivious encryption:] $\ct = \enco(\pk;\tilde r)$ obliviously generate a ciphertext $\ct$ from randomness $\tilde r$; $\tilde r' \gets \encoinv(\pk,\ct)$ explains the randomness for the ciphertext $\pk$;
  \item[Decryption:] $m' = \dec(\sk,\ct)$ decrypts the ciphertext $\ct$ using the secret key $\sk$;
  \end{description}
  satisfying the usual properties: perfect correctness, semantic security (IND-CPA), and:
  \begin{itemize}
  \item \textbf{Correctness.} For any security parameter $\secpar \in \N$, for any key pair $(\pk,\sk)$ in the image of $\keygen(1^\secpar)$, for any message $m \in \mathcal{M}$, for any ciphertext $\ct$ in the image of $\enc(\pk,m)$:
    \[ \dec(\sk,\ct) = m \enspace. \]
  \item \textbf{Semantic Security (IND-CPA).} For any polynomial-time circuit family $\advA =\smallset{\advA_\secpar}_{\secpar \in \N}$,
    there exists a negligible function $\negl$, such that for any $\secpar \in N$:
    \begin{multline*}
      \bigg| 2 \cdot \Pr\bigg[ \advA_\secpar(\state, \ct) \ : \ 
    \end{multline*}
  \item \textbf{Obliviousness.} For any $(\pk,\sk) \getsr \keygen(1^\secpar)$, the following two distributions are computationally indistinguishable:
    \begin{align*}
      &\set{(\ct,\tilde r') \ : \ m \getsr \mathcal{M}; \ \ct \getsr \enc(\pk,m); \ \tilde r' = \encoinv(\pk,\ct)}, \\
      &\set{(\ct,\tilde r) \ : \ \ct = \enco(\pk; \tilde r)}.
    \end{align*}
  \end{itemize}
  \Fnote{TODO write more if time}
\end{definition}

For any encryption scheme, we define the associated collection of NP languages $\smallset{\lang_{\enc,\pk}}_\pk$ of pairs $\word=(\ct,m)$ such that $\ct$ is a valid encryption of $m$ under $\pk$. More formally $(\ct,m) \in \lang_{\enc,\pk}$ if there exists a random tape $r$ such that $\ct = \enc(\pk,m;r)$.
The random tape $r$ is the witness $\wit$.
\Fnote{Setup for language parameter...}

\begin{definition}[SPHF with Projection Key Oblivious Sampling]
  A \emph{smooth projective hash function (SPHF) with projection key oblivious sampling} for a collection of languages $\smallset{\lang_\pk}_\pk$ is a tuple of five algorithms $(\HashKG,\ProjKGO,\ProjKGOinv,\Hash,\allowbreak\ProjHash)$:
  \begin{description}
  \item[Hashing key generation:] $(\hk,\hp) \getsr \HashKG(\word)$ generates a hashing key $\hk$ (i.e., a secret key) and an associated projection key $\hp$ (i.e., a public key) for~$\word$.
  \item[Projection key oblivious sampling:] $\hp = \ProjKGO(\word;\tilde \rho)$ obliviously sample a projection key from randomness $\tilde \rho$; $\tilde \rho' \getsr \ProjKGOinv(\word,\hp)$ explains the randomness for the projection key $\hp$.
  \item[Hashing:] $\HH = \Hash(\hk,\word)$ computes the hash value $\HH \in \bits$ of $\word$ for $\hk$.\footnote{We remark that we just need SPHF where hash values are only a single bit.}
  \item[Projected hashing:] $\projH = \ProjHash(\hp,\word,\wit)$ computes the projected hash value $\projH \in \bits$ of $\word$ for $\hk$.
  \end{description}
  satisfying the following properties:
  \begin{itemize}
  \item \textbf{Correctness.} For any $\word \in \lang$ with witness $\wit$, for any $(\hk,\hp) \getsr \HashKG(\word)$:
    \[ \Hash(\hk,\word) = \ProjHash(\hp,\word,\wit) \]
  \item \textbf{Smoothness.} For any $\word \notin \lang$, the following two distributions are statistically indistinguishable:
    \begin{align*}
      &\set{(\hp,\HH) \ : \ (\hk,\hp) \getsr \HashKG(\word); \ \HH = \Hash(\hk,\word) } \\
      &\set{(\hp,\projH) \ : \ (\hk,\hp) \getsr \HashKG(\word); \ \HH \getsr \bits}
    \end{align*}
  \item \textbf{Obliviousness.} For any $\word$ (whether in $\lang$ or not)\Fnote{to fix in eprint: the $\pk$ part needs be supposed to be well-formed... but no time to write this down formally. This is required for QR, but is not much of an issue. Basically needs a parameter for language and a parameter setup}, the following two distributions are computationally indistinguishable:
    \begin{align*}
      &\set{(\hp,\tilde \rho') \ : \ (\hk,\hp) \getsr \HashKG(\word);\ \tilde \rho' \getsr \ProjKGOinv(\word,\hp)}, \\
      &\set{(\hp,\tilde \rho)  \ : \ \hp = \ProjKGO(\word;\tilde \rho)}.
    \end{align*}
  \end{itemize}
\end{definition}

\begin{theorem}\label{th:ot-hps}
  Assuming the existence of an encryption scheme with ciphertext oblivious sampling and of an SPHF with projection key oblivious sampling for the associated language, there exists a two-round receiver-extractable semi-malicious oblivious transfer protocol with sender and receiver oblivious sampling in the CRS model.
\end{theorem}

{\small\BPR [Semi-Malicious OT protocol $\eot$]\label{prot:hpsot}
  Protocol $\eot = (\eotsetup,\allowbreak\eotsetupb,\allowbreak\eotsendone,\eotsendtwo,
  \allowbreak\eotoutput,\allowbreak\eootsendtwo,\allowbreak\iotsendtwo,\allowbreak\eotextr)$ is run between sender S and receiver R.
  It uses an encryption scheme with ciphertext oblivious sampling $(\keygen,\enc,\enco,\encoinv,\dec)$ and an SPHF with projection key oblivious sampling $(\HashKG,\ProjKGO,\ProjKGOinv,\Hash,\ProjHash)$ for $\lang_\enc$
  \BI
\item \textsc{Inputs:} S holds two bits $\otmsg_0, \otmsg_1 \in \bits$ and R holds a bit $\otsel$.\footnote{Extension to strings $\otmsg_0,\otmsg_1$ can be done by repeating the second flow in parallel, keeping a unique first flow.}
\item \textsc{Setup} $\eotsetup(1^\secpar)$:
  \BI
\item[] Run $(\pk,\sk) \getsr \keygen(1^\secpar)$ and set $(\crs,\tex) = (\pk,\sk)$.
  \EI
\item \textsc{First Round} $\eotsendone(\crs,\otsel;r)$: 
  \BI
\item[] Execute $\otflowone = \ct = \enc(\pk,\otsel;r)$ and output $\otflowone$.
  \EI
\item \textsc{Second Round} $\eotsendtwo(\crs,\otflowone,\otmsg_0,\otmsg_1)$:\Fnote{in similar protocols, randomness is specified but not used, to fix later for eprint version}
  \BE
\item For $b \in \bits$, generate $(\hk_b,\hp_b) = \HashKG((\pk,\ct,b))$;
\item For $b \in \bits$, compute $\HH_b = \Hash(\hk_b,(\pk,\ct,b))$ and set $\otflowtwo_b = \HH_b \xor \otmsg_b$;
\item Output the message $\otflowtwo = (\hp_0,\hp_1,\otflowtwo_0,\otflowtwo_1)$
  \EE
\item \textsc{Output} $\eotoutput((\hp_0,\hp_1,\otflowtwo_{0},\otflowtwo_{1}),\otsel,r)$: 
  \BI
\item[] Compute $\projH = \ProjHash(\hp_\otsel,(\pk,\ct,\otsel),r)$ and output $\projH \xor \otflowtwo_\otsel$;
  \EI
\item \textsc{Receiver Extraction} $\eotextr(\pk,\sk,\ct)$
  \BI
\item[] Output $\otsel$ if $\dec(\sk,\ct) = \otsel \in \bits$ or $\perp$ otherwise;
  \EI
\item \textsc{Receiver's Oblivious Flow:}
  \BI
\item $\ootsendone(\crs;\tilde r) = \enco(\pk; \tilde r)$
\item $\iotsendone(\crs,\otflowone) = \encoinv(\pk, \otflowone)$
  \EI
\item \textsc{Sender's Oblivious Flow:}
  \BI
\item $\ootsendtwo(\crs;\tilde zeta)$ parses $\tilde \zeta = (\tilde \rho_0,\tilde \rho_1,\otflowtwo_0,\otflowtwo_1)$, computes $\hp_b = \ProjKGO((\pk,\ct,b);\tilde \rho_b)$ for $b \in \bits$, and outputs $(\hp_0,\hp_1,\otflowtwo_0,\otflowtwo_1)$
\item $\iotsendtwo(\crs,(\hp_0,\hp_1,\otflowtwo_0,\otflowtwo_1))$ computes $\tilde \rho_b' = \ProjKGOinv((\pk,\ct,b),\hp_b)$ for $b \in \bits$ and outputs $(\tilde \rho_0,\tilde \rho_1,\otflowtwo_0,\otflowtwo_1)$
  \EI
  \EI
  \EPR}

\begin{proof}[Proof of \cref{th:ot-hps}]
  Correctness follows from the correctness of the encryption scheme and of the SPHF.
  Security against a malicious receiver and extraction follows from the perfect correctness of the encryption scheme and the smoothness: if $\dec(\sk,\ct) \neq b$, then smoothness ensures that even given $\hp_b$, $\HH_b$ is statistically close to uniform from the adversarial receiver point of view.
  Security against a semi-malicious sender comes from the fact that the encryption scheme is IND-CPA and thus $\ct$ can be replaced by the encryption of a random message.
  \Fnote{needs to write more in eprint}
\end{proof}

We conclude by exhibiting concrete instantiations from DDH and QR. 

\begin{example}[Instantiation from DDH]
  The encryption scheme is ElGamal in any group $\G$ of prime order $p$ where DDH holds, which is ciphertext oblivious samplable (assuming that we can obliviously sample group elements):
  \begin{itemize}
  \item \textsc{Key generation:} $\keygen$ generate a random generator $g$ of $\G$, and a random scalar $s \getsr \Z_p$. The secret key is $\sk = (g,s)$, while the public key is $\pk = (g,h)$, with $h = g^s$;
  \item \textsc{Encryption:} $\enc(\pk,m)$ samples $r \getsr \Z_p$, and outputs $\ct = (g^r, h^r m)$;
  \item \textsc{Decryption:} $\dec(\sk, (u,v))$ outputs $v / u^s$;
  \item \textsc{Oblivious encryption:} $\enco(\pk;\tilde r)$ obliviously samples from $\tilde r$ two group elements $(u,v)$ and outputs $(u,v)$; $\encoinv$ does the inverse operation.
  \end{itemize}
  The SPHF is the following from~\cite{EC:CraSho02} (we recall that $\word = (\pk,\ct,m)$ and $\wit =r$):
  \begin{itemize}
  \item \textsc{Hashing key generation:} $(\hk,\hp) \getsr \HashKG(\word)$ generates $(\alpha,\beta)\gets \Z_p^2$, outputs $\hk = (\alpha,\beta)$ and $\hp = g^\alpha h^\beta$;
  \item \textsc{Projection key oblivious sampling:} $\hp = \ProjKGO(\word;\tilde \rho)$ obliviously sample a random group element $\hp$ from randomness $\tilde \rho$; $\tilde \rho' \getsr \ProjKGOinv(\word,\hp)$ does the inverse operation.
  \item \textsc{Hashing:} $\HH = \Hash(\hk,\word)$ outputs the hash value $\HH = u^\alpha {(v/m)}^\beta$, where $\ct = (u,v)$.
  \item \textsc{Projected hashing:} $\projH = \ProjHash(\hp,\word,\wit)$ outputs the value $\projH = \hp^r$.
  \end{itemize}
  The ElGamal encryption scheme is IND-CPA under the Decisional Diffie-Hellman (DDH) assumption, so the resulting OT is secure under DDH.
\end{example}

\begin{example}[Instantiation from QR]
  The encryption scheme is a variant of the Goldwasser-Micali~\cite{JCSS:GolMic84} encryption scheme modulo an RSA modulus $N=pq$ product of two safe primes:
  \begin{itemize}
  \item \textsc{Key generation:} $\keygen$ generate a two safe primes $p=2p'+1$ and $q=2q'+1$, where $p'$ and $q'$ are two distinct large enough primes, and generates a generator $g$ of the cyclic group $\Jacob_N$ of elements of Jacobi symbol $1$ and $y$ an element of Jacobi symbol $-1$, and outputs $\pk = (N,g,y)$ and $\sk = (p,q)$.
  \item \textsc{Encryption:} $\enc(\pk,m)$ samples a large enough integer $r$ and outputs $\ct = g^{2r} {(-1)}^m$.\footnote{The squaring of $g$ is used to allow for oblivious sampling of the projection key of the associated SPHF. We remark that $g$ and $g^2$ generate the same subgroup of $\Z_N^*$.}
  \item \textsc{Decryption:} $\dec(\sk, \ct)$ return $0$ if and only if $\ct$ is a quadratic residue (which can be checked efficiently knowing the factorization of $N$); and $1$ otherwise;
  \item \textsc{Oblivious encryption:} $\enco(\pk; \tilde r)$ obliviously samples from $r$ an element of $\Jacob_N$, which can be done by reducing $r$ modulo $N$ and checking if it is in $\Jacob_N$; and if it is not, by multiplying the resulting element by $y$; $\encoinv$ does the inverse operation.
  \end{itemize}
  We recall that $\word = (\pk,\ct,m)$ and $\wit =r$.
  We describe a variant of SPHF with a weaker smoothness property, namely the hash value of a word outside the language has min-entropy 1, instead of being statistically indistinguishable from uniform.
  This can be transformed into a real SPHF by parallel repetitions and using a strong randomness extractor (see \cite{EC:CraSho02}).
  This transformation preserves projection key oblivious sampling.
  The construction is inspired from the QR SPHF of~\cite{EC:CraSho02} but is changed in a way that the membership to the set of projection keys can be publicly verifiable.
  More precisely, in the former construction, projections keys are quadratic residues, while in our constructions, we use signed quadratic residues~\cite{C:HofKil09} which are easily recognizable even without knowing the factorization of $N$.
  The group of signed quadratic residues is $\SQR_N = \set{|x| \ : \ x \in \QR_N}$, where $\QR_N$ is the group of quadratic residues and $|x|$ denotes the absolute value of $x \in \Z_N^*$ when seen as an integer in $\{-(N-1)/2,\dots,(N-1)/2\}$. This group is the same as $\Jacob_N^+ = \set{|x| \ : \ x \in \Jacob_N}$.
  \begin{itemize}
  \item \textsc{Hashing key generation:} $(\hk,\hp) \getsr \HashKG(\word)$ generates a large enough integer $\alpha$, and outputs $\hk = \alpha$ and $\hp = |g^\alpha|$;
  \item \textsc{Projection key oblivious sampling:} $\hp = \ProjKGO(\word;\tilde \rho)$ obliviously sample a random element $\hp \in \SQR_N = \Jacob_N^+$ from randomness $\tilde \rho$, by sampling an element of $\Jacob_N$ as for the oblivious encryption and taking its absolute value; $\tilde \rho' \getsr \ProjKGOinv(\word,\hp)$ does the inverse operation;
  \item \textsc{Hashing:} $\HH = \Hash(\hk,\word)$ outputs the hash value $\HH = {(\ct / {(-1)}^m)}^\alpha$;
  \item \textsc{Projected hashing:} $\projH = \ProjHash(\hp,\word,\wit)$ outputs the value $\projH = \hp^{2r}$.
  \end{itemize}
  Correctness of the resulting SPHF comes from the fact that $\HH = {(\ct / {(-1)}^m)}^\alpha = g^{2r\alpha} = {(|g^\alpha|)}^{2r} = \hp^{2r}$. This is where we see the importance of defining $\ct = g^{2r} {(-1)}^m$ instead of the more classical $\ct = g^r {(-1)}^m$.
  Smoothness of the resulting SPHF is argued exactly like in~\cite{EC:CraSho02}: the hash value of a word $\word \notin \lang$ (i.e., $(\ct / {(-1)}^m) \in \Jacob_N^* \setminus \QR_N$) is uniform among two opposite values when only conditioned on knowing $\hp$.
  Finally, the above encryption scheme is IND-CPA under the quadratic residuosity (QR) assumption, so the resulting OT is secure under QR.
\end{example}

%%% Local Variables:
%%% mode: latex
%%% TeX-master: "../main"
%%% End:

% \input{tex_files/07_NIZK}

\iftoggle{anonymous}{}{
  \subsubsection*{Acknowledgments\iftoggle{llncs}{.}{}}
}

\iftoggle{llncs}{
  \bibliographystyle{splncs03}
}{
%  \bibliographystyle{alpha-short}
  \bibliographystyle{alpha}
}
\bibliography{cryptobib/abbrev3,cryptobib/crypto,add}

\iftoggle{full}{
\appendix

\iftoggle{llncs}{\chapter*{Appendix}}{}

\remove{\section{Appendices from EC paper}

\Fnote{just for information}

% !TEX root =../main.tex

%\section{Additional Definitions}
%\label{A:def-preliminaries}

\subsection{Circuit Classes and Indistinguishability}
\label{A:preliminaries}

\subparagraph{Circuit Classes and Families:}
We recall the definitions of circuit classes and families.

\begin{mydefinition}[Class of $\cirsize$-Size Circuits]
  Let $\cirsize$ be a function from $\N$ to $\N$, a {\em
    $\cirsize$-size circuit class} is a family of sets
  $\circlass = {\{\circlass_\secpar\}}_{\secpar \in
    \N}$ of circuits, satisfying that every circuit $\cir \in
  \circlass_\secpar$ has size at most $\cirsize(\secpar)$. A
  {\em $\poly$-size circuit class} is a $\cirsize$-size circuit class
  for some polynomial $\cirsize$.
  
  Let $\cirinlen$ and $\ciroutlen$ be functions from $\N$ to $\N$. We
  say that $\circlass$ has input and/or output length $\cirinlen$ and
  $\ciroutlen$, if every circuit $\cir \in
  \circlass_\secpar$ has input and/or output length~$\cirinlen(\secpar)$ and~$\ciroutlen(\secpar)$.
\end{mydefinition}

For the sake of simplicity, we suppose that all circuits in $\circlass_\secpar$ have the same input and output lengths. This can be achieved without loss of generality using appropriate paddings.



% Since the
% input and output lengths of any circuit in
% $\circlass^\cirsize_\secpar$ are bounded by $\cirsize(\secpar)$, it
% suffices if the universal circuit $\univcirc^\cirsize_\secpar$ takes
% as input $\cirsize(\secpar) + \cirsize(\secpar)$ bits and outputs
% $\cirsize(\secpar)$ bits, and the inputs and outputs are appropriately
% padded.

\subparagraph{Statistical and Computational Indistinguishability:}
A function $\negl \colon \N \to \N$ is negligible if for any polynomial $p\colon \N \to \N$, for any large enough $\secpar \in \N$, $\negl(\secpar) < 1 / p(\secpar)$.

\begin{mydefinition}[Indistinguishability]
  Let $S = {\{S_\secpar\}}_{\secpar \in \N}$ be an ensemble of subsets
  of $\bits^*$, where every element in set $S_\secpar$ has length
  $\poly(\secpar)$.  Then ensembles $X = {\{X_{\secpar,
      w}\}}_{\secpar \in \N, w \in S_\secpar}$ and $Y = {\{Y_{\secpar,
      w}\}}_{\secpar \in \N, w \in S_\secpar}$ % \todo{F: maybe this should be moved after computationally ind., although there is a parenthesis issue --- by the way, we don't use this notation, but we may use it later.} 
  are \emph{statistically} (resp., \emph{computationally}) \emph{indistinguishable}, denoted as $X \approx_s Y$ (resp., $X \approx Y$), if for any arbitrary-size (resp., polynomial-size)
  circuit family $\dist = {\{\dist_\secpar\}}_{\secpar \in \N}$ and any polynomial-size sequence of index $\smallset{w_\secpar \in S}_{\secpar \in \N}$, there exists
  a negligible function $\negl$ such that, for every $\secpar \in \N$,
  \[ \left| \Pr\left[\dist_{\secpar}(w_\secpar,X_{\secpar,w_\secpar}) = 1\right] -
    \Pr\left[\dist_{\secpar}(w_\secpar,Y_{\secpar,w_\secpar}) = 1\right] \right| \le \negl(\secpar) \enspace.\]
\end{mydefinition}

Two statistically indistinguishable ensembles are also said to be \emph{statistically close}.


\subsection{Garbled Circuit}
\label{sec:gc}

\begin{mydefinition}[Garbled Circuit]
  Let $\circlass = {\{\circlass_\secpar\}}_{\secpar \in \N}$ be a
  $\poly$-size circuit class with input and output lengths $\cirinlen$
  and $\ciroutlen$.  A \emph{garbled circuit} scheme $\gc$ for
  $\circlass$ % with
%   polynomial universal circuits 
  is a tuple of four
  polynomial-time algorithms $\gc=(\gcgen,\gcgarble,\gceval,\allowbreak\gcsim)$: 
  \begin{description}
  \item[Input Labels Generation:] $\gckey \getsr \gcgen(1^\secpar)$
%     1^\cirinlen)$
    generates input labels $\{\gckey_{b}^i\}_{i \in
      [\cirinlen],b \in \bits }$ (with $\gckey_{b}^i \in \bits^\kappa$ being
    the input label corresponding to the value $b$ of the $i$-th input
    wire) for the security parameter $\secpar$, input length
    $\cirinlen$, and input label length $\kappa$; % = \cirinlen(\secpar)$;
  \item[Circuit Garbling:] $\gccir \getsr \gcgarble(\gckey, \cir)$
     garbles the circuit $\cir \in \circlass_\secpar$
    into $\gccir$;
  \item[Evaluation:] $\cirout = \gceval(\gccir, \gckey_{b_i})$  evaluates the
    garbled circuit $\gcgarble$ using input labels
    ${\{\gckey^i_{b_i}\}}_{i \in [n]}$ (where $\gckey'[i] \in \bits^\kappa$)
    and returns the output $\cirout \in \bits^\ciroutlen$; %   with
%      $\ciroutlen = \ciroutlen(\secpar)$
    
  \item[Simulation:] $(\gckey', \gcsimcir) \getsr \gcsim(1^\secpar,
    \cirout)$
    % 1^\cirsize, 1^\cirinlen, \cirout)$
     simulates  input labels $\gckey'$ and
    a garbled circuit $\gcsimcir$ for the security parameter
    $\secpar$ and the output $\cirout \in \bits^\ciroutlen$;
  \end{description}
  satisfying the following security properties:
  \begin{description}
  \item[Correctness:] For any security parameter $\secpar \in \N$, for any circuit $\cir \in \circlass_\secpar$, for any input $\cirin \in \bits^{\cirinlen}$, for any $\gckey$ in the image of $\gcgen(1^\secpar)$ and any $\gccir$ in the image of $\gcgarble(\gckey, \cir)$:
    \[ \gceval(\gccir, {\{\gckey^i_{\cirin_i}\}}_{i \in [\cirinlen]}) = \cir(\cirin)\enspace. \]
  \item[Simulatability:] The following two distributions are computationally indistinguishable:
    \begin{align*}
      \bigg\{
      ({\{\gckey^i_{\cirin_i}\}}_{i \in [\cirinlen]}, \gccir)
      \ &: \
          \begin{array}{l}
            \gckey \getsr \gcgen(1^\secpar); \\
            \gccir \getsr \gcgarble(\gckey, \cir)
          \end{array}
      {\bigg\}}_{\secpar, \cir \in \circlass_\secpar, \cirin \in \bits^{\cirinlen}} \enspace, \\
      \big\{
      (\gckey', \gccir)
      \ &: \
          \begin{array}{l}
            (\gckey',\cir) \getsr \gcsim(1^\secpar, \cir(\cirin))
          \end{array}
          {\big\}}_{\secpar, \cir \in \circlass_\secpar, \cirin \in
          \bits^{\cirinlen}} \enspace.
    \end{align*}
  \end{description}
\end{mydefinition}

We recall that garbled circuit schemes can be constructed from one-way
functions.

For the sake of simplicity, if $\cirin \in \bits^\cirinlen$ and $\gckey = {\{\gckey[i, b]\}}_{i \in
      [\cirinlen], b \in \bits}$, we define
$\gckey[\cirin] = \smallset{\gckey[i,\cirin_i]}_{i \in
  [\cirinlen]}$.
  
  \remove{
We extend this notation when the input is a tuple: for example, if $\cirin=(u,v) \in \bits^{\cirinlen_1} \times \bits^{\cirinlen_2}$, we define $\gckey[u] = \smallset{\gckey[i,u_i]}_{i \in [\cirinlen_1]}$ and $\gckey[v] = \smallset{\gckey[\cirinlen_1 + i,v_i]}_{i \in [\cirinlen_2]}$.
We also abuse notation and define $\gckey[[u]]$ (resp., $\gckey[[v]]$) to be the $2\cirinlen_1$ (resp., $2\cirinlen_2$) input labels corresponding to the input wires for $u$ and $v$: $\gckey[[u]] = \smallset{\gckey[i,b]}_{i \in [\cirinlen_1],b \in \bits}$ and $\gckey[[v]] = \smallset{\gckey[\cirinlen_1 + i,b]}_{i \in [\cirinlen_2], b \in \bits}$.
This notation is also used for $\gckey' = {\{\gckey[i]\}}_{i \in [\cirinlen]}$: $\gckey'[[u]] = \smallset{\gckey'[i]}_{i \in [\cirinlen_1]}$ and $\gckey'[[v]] = \smallset{\gckey'[\cirinlen_1 + i]}_{i \in [\cirinlen_2]}$.}


\subsection{Multiparty Computation Protocols}
\label{A:def-MPC}

We recall the definition of semi-honest multi-party computation (MPC) protocols essentially from~\cite{Goldreich04}.

\subsubsection{Syntax}
\begin{mydefinition}[Functionality] Let $N$ be a positive integer.  An $N$-party functionality
  $\mpcfunc$ is a deterministic function from $\bigcup_{\mpcinlen \in
    \N} {(\bits^\mpcinlen)}^N$ to ${(\bits^*)}^N$.
\end{mydefinition}

For any $i \in [N]$, we write $f_i(\mpcvecin)$ the $i$-th element of the output tuple of $f$ on input $\mpcvecin \in \bigcup_{\mpcinlen \in \N} {(\bits^\mpcinlen)}^N$.
For any $I \subseteq [N]$, we write $f_I(\mpcvecin) = {\{f_i(\mpcvecin)\}}_{i \in I}$.
Similarly, $\mpcvecin_I = \smallset{x_i}_{i \in I}$.

We consider MPC protocols where at each round $\round$, each party $P_i$ broadcasts a message $m_i^\round$ to all the other parties.

\begin{mydefinition}[MPC Protocol]
  \label{def:mpc}
  Let $N$ be a positive integer, % Let $\mpcinlen =
  % \mpcinlen(\secpar)$, $\mpcrandlen = \mpcrandlen(\
  $\nbrounds = \nbrounds(\secpar)$ a polynomial in the security
  parameter, and $\mpcfunc$ an $N$-party functionality.  An
  $\nbrounds$-round \emph{MPC protocol} $\mpc$ for $\mpcfunc$ is a
  tuple of two deterministic polynomial-time algorithms
  $\mpc=(\mpcnext, \mpcoutput)$:

  \begin{description}
  \item[Next Message:] $\msg_{i}^\round =
    \mpcnext_i(1^\secpar,\mpcin_i,\mpcrand_i,\vec \msg^{<\round})$ is the message
  broadcasted by party $P_i$ for $i \in [N]$ in round $\round \in
  [\nbrounds]$, on input $\mpcin_i \in \bits^\mpcinlen$, on random
  tape $\mpcrand_i \in \bits^{\mpcrandlen}$, after receiving the
  messages $\vec \msg^{<\round} = {\{\msg_j^{\round'}\}}_{j
      \in [N],\round' < \round}$, where $\msg_j^{\round'}$ is the message broadcasted by party $P_j$ on round $\round' \in [\round-1]$, and where the input length $\mpcinlen$
  and random tape length $\mpcrandlen$ are polynomial in the security parameter $\secpar$;
  \item[Output:] $\mpcout_i = \mpcoutput_i(1^\secpar,
    \mpcin_i,\mpcrand_i,\vec \msg)$ is the output of party $P_i$ for $i
  \in [N]$, on input $\mpcin_i \in
  \bits^\mpcinlen$, on random tape $\mpcrand_i \in
  \bits^{\mpcrandlen}$, after receiving the messages
  $\vec \msg = \smallset{\msg_j^\round}_{j \in [N], \round \in [\nbrounds]}$ as defined above;
  \end{description}
  satisfying the following property:
  \begin{description}
  \item[Correctness:] For any security parameter
    $\secpar \in \N$, for any inputs $(\mpcin_1,\dots,\mpcin_N) \in
    {(\bits^\mpcinlen)}^N$, 
    \begin{multline*}
      \Pr\Big[
        \left\{\mpcoutput_i(1^\secpar,\mpcin_i,\mpcrand_i,\vec\msg)\right\}_{i \in [N]} \ne
        \mpcfunc(\mpcin_1,\dots,\mpcin_N) \ : \ 
      \mpcvecrand \getsr {(\bits^{\mpcrandlen})}^N
        \Big] =0~,
    \end{multline*}
    where $\msg_{i}^\round =
    \mpcnext_i(1^\secpar,\mpcin_i,\mpcrand_i,\vec \msg^{<\round})$ for $i \in [N]$ and $\round \in
    [\nbrounds]$.
  \end{description}
\end{mydefinition}


\begin{mydefinition}[View and Output]
  Let $N$ be a positive integer. Let $\mpcfunc$ be an $N$-party functionality.
  Let $\mpc = (\mpcnext, \mpcoutput)$ be an MPC protocol for $\mpcfunc$.
  Let $I \subseteq [N]$.

  \begin{itemize}
  \item   The \emph{view} of parties ${\{P_i\}}_{i \in I}$ during an execution of $\mpc$ with security parameter~$\secpar$, input length $\mpcinlen$, inputs $\mpcvecin = (\mpcin_1,\dots,\mpcin_N) \in {(\bits^{\mpcinlen})}^N$, random tapes $\mpcvecrand = (\mpcrand_1,\dots,\mpcrand_N) \in {(\bits^{\mpcrandlen})}^N$ is:
  \[ \mpcview_I(1^\secpar,\mpcvecin,\mpcvecrand) = {\left( \mpcvecin_i, \mpcvecrand_I, \vec \msg \right)}\enspace, \]
where $\vec \msg$ is defined as in \cref{def:mpc}.

\item The \emph{output} of the protocol for the parties ${\{P_i\}}_{i \in I}$ is:
\[ \mpcoutput_I(\mpcvecin,\mpcvecrand) = \smallset{\mpcout_i}_{i \in I}\enspace, \]
where $\mpcout_i = \mpcoutput_i(1^\secpar, \mpcin_i,\mpcrand_i, \vec \msg)$.

  \end{itemize}

\end{mydefinition}

In the sequel, the unary representation $1^\secpar$ of the security parameter $\secpar$ is often omitted from the parameters of $\mpcnext$, $\mpcoutput$, and $\mpcview$ to simplify notation.


\subsubsection{Security against Semi-Honest Adversaries}

\begin{mydefinition}[Security against Semi-Honest Adversaries]
  \label{def:semi-honest-mpc}
  Let $N$ be a positive integer. Let $\mpcfunc$ be an $N$-party functionality.
  Let $\mpc$ be an MPC protocol for~$\mpcfunc$.
  Then $\mpc$ is \emph{secure against semi-honest adversaries} if there exists a probabilistic polynomial-time algorithm $\mpcsim$ such that for the following two distributions are computationally indistinguishable:
  \begin{align*}
    &{\left\{
      \left(
      \mpcview_I(1^\secpar, \mpcvecin, \mpcvecrand),\;
      \mpcoutput_I(\mpcvecin,\mpcvecrand)\right)
      \ : \
      \mpcvecrand \getsr {(\bits^{\mpcrandlen})}^N
      \right\}}_{\secpar,I \subseteq [N],\mpcvecin} \enspace, \\
    &{\left\{
      \left(\mpcsim(1^\secpar, I, \mpcvecin_{I},
      \mpcfunc_{I}(\mpcvecin)),\;
      \mpcfunc_I(\mpcvecin)\right)
      \right\}}_{\secpar,I \subseteq [N],\mpcvecin} \enspace.
  \end{align*}
\end{mydefinition}

\subsubsection{Security against Malicious Adversaries}
We now recall the notion of security against malicious adversary. We
focus on the case with static corruptions and security with abortion.
We also recall that we assume that parties have access to a simultaneous broadcast channel.

We first need to define the notions of ideal execution
$\mpcideal_{I,\mpcsim}(1^\secpar,\mpcvecin)$ against a simulator
$\mpcsim$ simulating malicious parties $\smallset{P_i}_{i \in I}$ and
of real execution $\mpcreal_{I,\advA}(1^\secpar,\mpcvecin)$ against an
adversary $\advA$ playing the roles of malicious parties
$\smallset{P_i}_{i \in I}$.  Simulators $\mpcsim$ are defined as
non-uniform \emph{expected-poly-time} interactive Turing machines
while adversaries $\advA$ are defined as non-uniform poly-time
interactive Turing machines.

\paragraph{Ideal Execution.}
$\mpcideal_{I,\mpcsim}(1^\secpar,\mpcvecin)$ is defined by playing the following game with the simulator $\mpcsim$:
\begin{enumerate}
\item The simulator is given $I$ and $\mpcvecin_I$.
\item The simulator chooses a vector
  $\mpcvecin'_I = \smallset{\mpcvecin'_i}_{i \in I}$ intuitively corresponding to the extracted inputs of the malicious parties. We set $\mpcin'_i = \mpcin_i$ for $i \in \bar{I}$, where $\bar{I} = [N] \setminus I$ corresponds to the set of honest parties. As usual, $\mpcvecin' = \smallset{\mpcvecin'_i}_{i \in [N]}$. % such that $\mpcvecin'_{\bar{I}} = \mpcvecin_{\bar{I}}$, where $\bar{I} = [N] \setminus I$ corresponds to the set of honest parties.
\item The simulator is given $\mpcfunc_I(\mpcvecin')$.
\item The simulator can then decide to abort or proceed. If it aborts, we set $\vec \mpcout_{\bar{I}} = (\perp,\dots,\perp)$, otherwise, we set $\vec \mpcout_{\bar{I}} = \mpcfunc_{\bar{I}}(\mpcvecin')$.
\item $\mpcideal_{I,\mpcsim}(1^\secpar,\mpcvecin)$ is defined as $(\vec \mpcout_{\bar{I}}, z)$ where $z$ is the output of the simulator.
\end{enumerate}

\paragraph{Real Execution.}
$\mpcreal_{I,\advA}(1^\secpar,\mpcvecin)$ is defined by running the MPC protocol where the adversary $\advA$ controls the malicious parties $\smallset{P_i}_{i \in I}$ while the honest parties $\smallset{P_i}_{i \in \bar{I}}$ follow the protocol. It is then defined as the pair $(\vec \mpcout_{\bar{I}}, z)$, where $\vec\mpcout_{\bar{I}}$ is the vector of outputs of the honest parties while $z$ is the output of the adversary.
The adversary can be rushing: in each round, it can wait for all the messages from the honest parties before sending its own messages.

%\paragraph{Security.}

\begin{mydefinition}[Malicious Security]
  Let $N$ be a positive integer. Let $\mpcfunc$ be an $N$-party functionality.
  Let $\mpc$ be an MPC protocol for~$\mpcfunc$.
  Then $\mpc$ is \emph{secure against malicious adversaries} if for any non-uniform poly-time interactive Turing machine $\advA$, there exists a non-uniform expected-poly-time interactive Turing machine $\mpcsim = \smallset{\mpcsim_\secpar}_{\secpar \in \N}$ such that:
  \[ \smallset{\mpcideal_{I,\mpcsim}(1^\secpar,\mpcvecin)}_{\secpar,I,\mpcvecin} \approx \smallset{\mpcreal_{I,\advA}(1^\secpar,\mpcvecin)}_{\secpar,I,\mpcvecin} \enspace. \]
\end{mydefinition}


\subsubsection{Security against Semi-Malicious Adversaries}
A semi-malicious adversary~\cite{EC:AJLTVW12} $\advA$ is similar to a malicious adversary, except that after each round, it has to write on a special \emph{witness tape}, pairs $(\mpcin_i,\mpcrand_i)$ of input $\mpcin_i$ and randomness $\mpcrand_i$ explaining all the messages of the malicious party $P_i$, for each $i \in I$.
The witnesses given in each round do not need to be consistent, and the adversary is rushing: in each round, it can choose its message and witness $(\mpcin_i,\mpcrand_i)$ after having seen the messages of the other parties.

More formally, we define $\mpcrealsm_{I,\advA}(1^\secpar,\mpcvecin)$ as $\mpcreal_{I,\advA}
(1^\secpar,\mpcvecin)$ except that if at some round $\round$ one witness is invalid, then honest parties all abort (do not send any more messages) and output $\perp$.

\begin{mydefinition}[Semi-Malicious Security]
  Let $N$ be a positive integer. Let $\mpcfunc$ be an $N$-party functionality.
  Let $\mpc$ be an MPC protocol for~$\mpcfunc$.
  Then $\mpc$ is \emph{secure against malicious adversaries} if for any non-uniform poly-time interactive Turing machine $\advA$ (with an extra witness tape), there exists a non-uniform poly-time interactive Turing machine $\mpcsim$ such that:
  \[ \smallset{\mpcideal_{I,\mpcsim}(1^\secpar,\mpcvecin)}_{\secpar,I,\mpcvecin} \approx \smallset{\mpcrealsm_{I,\advA}(1^\secpar,\mpcvecin)}_{\secpar,I,\mpcvecin} \enspace. \]
\end{mydefinition}

\subsubsection{Delayed-Semi-Malicious Security}
\label{sec:def-def-mpc}
Haitner~\cite{TCC:Haitner08} introduced the notion of defensible
security for constructing malicious OT from semi-honest OT in a
black-box way. In his definition, a defensible adversary is one that
outputs at the end of the protocol execution a ``defense,'' which is a
pair of input and randomness, and is valid if an honest player with
this pair of input and randomness would produce exactly the same
messages as what the adversary has sent. In other words, a defensible
adversary is like a semi-malicious adversary, except that it only
needs to provide a witness (as defined above) at the end of the
execution. Haitner then gave an indistinguishability-based definition
of OT privacy against defensible adversaries.

In this work, we consider a variant of defensible adversaries,
called \emph{delayed-semi-malicious} who are required to provide a
witness in the {\em second last round}, and security only holds if
this witness explains the messages of the corrupted players in {\em
  all} rounds.  Furthermore, we define simulation-based security
against these adversaries with a {\em universal} simulator that can
simulate the view of the adversaries by interacting them as black-box
in a {\em straight-line}. In slightly more detail,
\begin{itemize}
\item The real world is defined identically as the real world for
  semi-malicious security, except that, the adversary $\advA$ is only
  required to provide a witness in the {\em second last round}, that
  is, round $\nbrounds-1$. If the witness is invalid w.r.t.\ messages
  of the corrupted players in the first $\nbrounds-1$ rounds, then
  honest parties all abort (do not send any more messages) and output
  $\perp$ after round $\nbrounds-1$. In addition, if the witness is
  invalid w.r.t.\ messages of the corrupted parties in the last round
  $\nbrounds$, then honest parties again output $\perp$.
  $\mpcrealdef_{I,\advA}(1^\secpar,\mpcvecin)$ denotes the outputs of
  honest players and the adversary.

\item The ideal world is defined identically as the ideal world for
  semi-malicious security, except that, the universal simulator
  $\mpcsim$ on input $(1^\secpar, I)$ interacts with adversary $\advA$
  (as a black-box) in a {\em straight line}, and receives the witness
  that $\advA$ outputs after round $\nbrounds
  -1$. $\mpcideal_{I,\mpcsim\leftrightarrow\advA}(1^\secpar,\mpcvecin)$
  denotes the output of honest players and $\mpcsim$.

\end{itemize}
% More formally, we define $\mpcrealdef_{I,\advA}(1^\secpar,\mpcvecin)$
% as $\mpcreal_{I,\advA} (1^\secpar,\mpcvecin)$ except that
% honest parties all abort and output $\perp$ if the witness provided at round $\nbrounds-1$
% is invalid w.r.t.\ a message sent by the adversary at any round (including the last round $\nbrounds$).
%, as
%described above, if the witness is invalid at the round $\nbrounds-1$
%or $\nbrounds$, then honest parties all abort (do not send any more
%messages) and output $\perp$.

\begin{mydefinition}
  Let $N$ be a positive integer. Let $\mpcfunc$ be an $N$-party
  functionality.  Let $\mpc$ be an MPC protocol for~$\mpcfunc$.  Then
  $\mpc$ is \emph{delayed-semi-maliciously secure} if there exists a
  non-uniform expected-poly-time interactive Turing machine $\mpcsim$,
  such that, for every non-uniform poly-time interactive Turing
  machine $\advA$:
  \[
  \smallset{\mpcideal_{I,\mpcsim\leftrightarrow\advA}(1^\secpar,\mpcvecin)}_{\secpar,I}
  \approx
  \smallset{\mpcrealdef_{I,\advA}(1^\secpar,\mpcvecin)}_{\secpar,I,\mpcvecin}
  \enspace. \]
\end{mydefinition}

%%% Local Variables:
%%% mode: latex
%%% TeX-master: "../main"
%%% End:

% \section{Security Proof of the Construction of 2-Round Semi-Honest MPC}
% \label{A:sec-sh-mpc}

% \subsection{Proof of \cref{th:sec-sh-mpc}}
% \label{A:th-sec-sh-mpc}
\remove{
\begin{proof}[Proof of \cref{th:sec-sh-mpc}]
  
  Correctness is straightforward. Let us prove security against semi-honest adversaries.
  
  We need to exhibit a polynomial-time simulator of the view of any subset $I \subseteq [N]$ of corrupted parties, namely:
  \[ \tmpcview_I(1^\secpar, I, \mpcvecin, \tmpcvecrand) = (\mpcvecin_I,\tmpcvecrand_I,\tvecmsg) \]
  where $\tmpcvecrand = \smallset{\tmpcrand_i}_{i \in [N]}$ are honestly-generated random tapes of the parties. We recall that $\mpcvecin_I = \smallset{\mpcin_i}_{i \in I}$ and $\tmpcvecrand_I = \smallset{\tmpcrand_i}_{i \in I}$.

  The simulator first run the simulator of the inner MPC protocol and get $(\mpcvecrand_I, \vecmsg)$. It then simulates all the messages $\tvecmsg$ together with the random tapes $\tmpcvecrand_I$ of the corrupted parties as follows.
  \vspace{\baselineskip}

  \noindent \ul{\em First round}:
  \begin{itemize}
  \item For each corrupted party $P_\iii$ with $\iii \in I$, generate the commitments $\hcc_{\iii}^\round = \hccom(1^\secpar,
    (\mpcin_{\iii},\mpcrand_{\iii}); \hcr_\iii^\round)$ and the first message $\tmsg_{\iii}^1 =
    \smallset{\hcc^\round_{\iii}}_{\round \in [\nbrounds]}$ as in the real protocol.
  \item For each honest party $P_\iii$ with $\iii \notin I$, simulate the commitments:
    \[ (\hcc_\iii^\round,\hcd_\iii^\round) \getsr \hcsim(1^\secpar,G_\iii^\round,\msg_\iii^\round)\enspace, \]
    for $\round \in [\nbrounds]$ and for the circuit $\G_\iii^\round$ defined by $G_\iii^{\round-1}(\star,\star) = \mpcnext_\iii(\star,\star,\vec \msg^{< \round-1})$. Then set the first message $\tmsg_{\iii}^1 =
    \smallset{\hcc^\round_{\iii}}_{\round \in [\nbrounds]}$.
  \end{itemize}

  \noindent \ul{\em Second round}:
  \begin{itemize}
  \item For each corrupted party $P_\iii$ with $\iii \in I$, generate the garbled interactive circuit $ \giccir_\iii \getsr \gicgarble(1^\secpar, \icnext_\iii)$ and the second message $\tmsg_{\iii}^2 = \giccir_\iii$, as in the real protocol.
  \item For each honest party $P_\iii$ with $\iii \notin I$, compute $\vec q^\round = \smallset{\hcc_j^{\round-1}, G_j^{\round-1}}_{j \in [N]}$, $\vec w^\round = \smallset{\msg_j^\round,\hcd_j^\round}_{j \in [N]}$, and $o^\round = (\msg_\iii^\round,\hcd_\iii^\round)$, for $\round \in [\nbrounds]$, and simulate the garbled interactive circuit:
    \[ \gicsimcir_\iii \getsr \gicsim(1^\secpar, \smallset{\vec q^\round,\vec w^\round,o^\round}_{\round \in [\nbrounds]}) \enspace. \]
    The second message is $\tmsg_{\iii}^2 = \gicsimcir_\iii$.
  \end{itemize}

  We now need to prove that the simulated view is indistinguishable from the real view.
  More formally we need to prove that the following two distributions are computationally indistinguishable:
  \begin{align*}
    \distr_0 &= {\left\{
               \left(
               \mpcview_I(1^\secpar, \mpcvecin, \mpcvecrand),\;
               \mpcoutput_I(\mpcvecin,\mpcvecrand)\right)
               \ : \
               \mpcvecrand \getsr {(\bits^{\mpcrandlen})}^N
               \right\}}_{\secpar,I \subseteq [N],\mpcvecin} \enspace, \\
    \distr_1 &= {\left\{\left(\mpcsim(1^\secpar, I, \mpcvecin_{I},
          \mpcfunc_{I}(\mpcvecin)),\;
          \mpcfunc_I(\mpcvecin)\right)\right\}}_{\secpar,I \subseteq [N],\mpcvecin} \enspace.
  \end{align*}

  For that, we consider a sequence of $N+N^2$ hybrids $\smallset{\hybrid_{1,\istar}}_{\istar \in [N]}$ and $\smallset{\hybrid_{2,(\roundstar,\jstar)}}_{\roundstar \in [\nbrounds],\jstar \in [N]}$:
  \begin{description}
  \item[Hybrid $\hybrid_{1,\istar}$:] This hybrid is similar to $\distr_0$ (the real protocol), except that for the second messages of parties $P_i$ for $i \le \istar$ which are simulated as in $\distr_1$: when $i \notin I$, $\tmsg_{i}^2 = \gicsimcir_i = \gicsim(1^\secpar, \smallset{\vec q^\round,\vec w^\round,o^\round}_{\round \in [\nbrounds]})$.

    Let $\hybrid_{1,0} = \distr_0$. We have the following claim.
    \begin{claim}
      If $\gic$ is simulatable, then for any $\istar \in [N]$, $\hybrid_{1,\istar-1}$ and $\hybrid_{1,\istar}$ are computationally indistinguishable.
    \end{claim}
    \begin{proof}
      First, if $P_\istar$ is corrupted ($\istar \in I$), then $\hybrid_{1,\istar-1}$ and $\hybrid_{1,\istar}$ are actually the same distribution. Let us focus on the case $\istar \notin I$.

      % Since $\hc$ is semi-honest functionally binding, $\ndadistr^\hc$ is unique-answer and $\icdistr$ is unique-transcript.
      The only difference between  $\hybrid_{1,\istar-1}$ and $\hybrid_{1,\istar}$ is that $\tmsg_{i}^2 = \gicsimcir_i$ is simulated in the latter distribution. Thus, these two distributions are computationally indistinguishable if $\gic$ is simulatable.
    \end{proof}
    
  \item[Hybrid $\hybrid_{2,(\roundstar,\jstar)}$:] We consider the lexicographic order $\prec$ (or any linear order) over the pairs $(\roundstar,\jstar) \in [\nbrounds] \times [N]$, and we define ${(\roundstar,\jstar)}^{-}$ to be the predecessor of $(\roundstar,\jstar)$.

    The hybrid $\hybrid_{2,(\roundstar,\jstar)}$ is similar to $\distr_1$ (the simulated protocol), except that for all $(\round,j) \succ (\roundstar,\jstar)$, $\hcc_j^\round$ and $\hcd_j^\round$ are generated as in the real protocol ($\distr_0$):
    \[ \hcc_{j}^\round = \hccom(1^\secpar,
      (\mpcin_{j},\mpcrand_{j}); \hcr_j^\round); \
      \hcd_j^\round = \hcopen(\hcc_j^\round, G_j^\round,\msg^\round_{j}, \hcr_j^\round)
      \enspace, \]
    where $\hcr_j^\round$ is a uniform random tape.

    Let $\hybrid_{1,N} = \hybrid_{2,{(1,1)}^{-}}$. We have the straightforward following claim.
    \begin{claim}
      If $\hc$ is simulatable, then for any $(\roundstar,\jstar) \in [\nbrounds] \times [N]$, $\hybrid_{2,{(\roundstar,\jstar)}^-}$ and $\hybrid_{2,(\roundstar,\jstar)}$ are computationally indistinguishable.
    \end{claim}

    Furthermore, the only difference between $\hybrid_{2,(N,N)}$ and $\distr_1$ is that in the latter distribution, the inner MPC messages $\vec \msg$ are simulated by the inner MPC simulator.
    Thus, we have the following claim.
    \begin{claim}
      If the inner MPC is secure against semi-honest adversaries, then $\hybrid_{2,(N,N)}$ and $\distr_1$ are computationally indistinguishable.
    \end{claim}
  \end{description}


  We remark that we do not directly use the semi-honest functional binding property of the FC scheme, as it is implied by the simulatability property of the GIC scheme.
\end{proof}}

%%% Local Variables:
%%% mode: latex
%%% TeX-master: "../main"
%%% End:

%\section{Proof of the Construction of Garbled Interactive Circuit}
%\label{A:sec-cons-gic}
\remove{
\begin{proof}[Proof of \cref{th:sec-cons-gic}]
  Correctness is straightforward. Let us prove simulatability.
  
  \subparagraph{Simulatability:} We need to prove the computational indistinguishability of the following two distributions:
  \begin{align*}
    \distr_0 = &\bigset{\giccir \ : \ (\iC, \vec w, \aux) \getsr \icdistr_{\secpar,
                 \icind}; \ \giccir \getsr \gicgarble(1^\secpar, \iC) }_{\secpar, \icind}\enspace, \\
    \distr_1 = &\bigset{\gicsimcir \ : \ (\iC, \vec w, \aux) \getsr \icdistr_{\secpar, \icind}; \ 
                  \gicsimcir \getsr \gicsim(1^\secpar, \trans(\iC, \nda_\secpar, \vec w)) 
                 }_{\secpar, \icind} \enspace.
  \end{align*}

  For that, we introduce $2 \nbrounds+2$ hybrid distributions $\hybrid_{0,0},\hybrid_{0,1},\hybrid_{1,0},\hybrid_{1,1},\hybrid_{2,0},\dots,\hybrid_{\nbrounds,1}$:
  \begin{description}
  \item[Hybrid $\hybrid_{\round,0}$:] This hybrid is similar to $\distr_1$, except that $\gcicaugnext^{> \round}$ and $\gckey^{> \round}$ (thus in particular $\vec \wsct^{>\round}$ is not defined) are generated as in $\distr_0$.

  We have the following straightforward claim.
  
  \begin{claim}
    $\hybrid_{0,0}$ and $\distr_{0}$ are the same distribution.
  \end{claim}
  
  \item[Hybrid $\hybrid_{\round,1}$: ]
    For $\round = 0$, this hybrid is the same as $\hybrid_{0,0}$.
    
    For $\round \ge 1$, this hybrid is similar to $\hybrid_{\round,0}$, except that:
    $\vec\wsct^{\round}$ is computed as:
    \[ \wsct^\round_k \getsr \wsenc(1^\secpar, q^\round_k, \gckey''^{\round+1}[[a^\round_k]])\enspace, \]
    where $\gckey''^{\round+1}[i,b] = \gckey[i,a^\round_{k,j}]$ for each $b \in \bits$ and each input wire $i$ corresponding to the $j$-th bit of the input $a^\round_k$ in $\icaugnext^\round$.
    In other words, for each input wire of the answers $\vec a^\round$, instead of encrypting both possible input labels with the witness selector, we encrypt twice the input label which is actually used.
    
    Thanks to consistency between the distributions $\icdistr$ and $\ndadistr$, the semantic security of the witness selector ensures that this hybrid is indistinguishable from the previous one.
    \begin{claim}
      If $\icdistr$ and $\ndadistr$ are consistent and if $\ws$ is semantically secure, then for any $\round \in [\nbrounds]$, $\hybrid_{\round,0}$ and $\hybrid_{\round,1}$ are computationally indistinguishable.
    \end{claim}

    Let $\hybrid_{\nbrounds+1,0}$ be the distribution $\distr_1$.
    As the only difference between $\hybrid_{\round,1}$ and $\hybrid_{\round+1,0}$ is that in the latter hybrid, $\gcicaugnext^{\round+1}$ and $\gckey^{\round+1}$ are simulated via $\gcsim$ instead of being generated via $\gcgarble$, we have the following claim.
    
    \begin{claim}
      If $\gc$ is simulatable, then for any $\round \in \{0,\dots,\nbrounds\}$, $\hybrid_{\round,1}$ and $\hybrid_{\round+1,0}$ are computationally indistinguishable.
    \end{claim}
  \end{description}
\end{proof}}

%%% Local Variables:
%%% mode: latex
%%% TeX-master: "../main"
%%% End:

% \section{Security Proof of the Construction of Functional Commitment with Witness Selector}
% \label{A:sec-hc-ws}

\begin{proof}[Proof of \cref{th:sec-hc-ws}]
  Correctness is straightforward. Let us now prove semi-honest functional binding of $\hc$, simulatability of $\hc$, and semantic security of $\ws$.
  
  \subparagraph{Semi-Honest Functional Binding:}
  Semi-honest functional binding follows directly from the semantic security of the witness selector, which we prove below.
  Indeed, an adversary $\advA$ against semi-honest functional binding generate a functional decommitment $\hcd$ to some value $\hccirout$  for some circuit $\hccir$, on input $\hcr$ and $\hcc = \hccom(1^\secpar,\hcmsg;\hcr)$, such that $\hcout \neq \hccir(\hcmsg)$ (and $\hcd$ is indeed a valid functional decommitment: $\hcver(\hcc,\hccir,\hccirout,\hcd)=1$.)
  This pair $(\hcout,\hcd)$ can be used to decrypt any ciphertext $\wsct \getsr \wsenc(1^\secpar, (\hcc,\hccir), \wsmsg)$ to $\wsmsg' = \smallset{\wsmsg[I,\hcout_I]}_{I \in [\hcciroutlen]}$, which breaks semantic security of the witness selector, as $\hcout \neq \hccir(\hcmsg)$.
  
  \subparagraph{Simulatability:} We need to prove the computational indistinguishability of the following two distributions:
  \begin{align*}
    \distr_0 &= {\left\{
               (\hcc,\hcd)
               \ : \
               \hcr \getsr \bits^\hcrlen; \ \hcc =
               \hccom(1^\secpar,\hcmsg;\hcr); \ \hcd = \hcopen(\hcc, \hccir, \hcmsg,
               \hcr) 
               \right\}}_{\secpar,\hcmsg,\hccir}\enspace, \\
    \distr_1 &= {\left\{  (\hcc,\hcd)
               \ : \ 
               (\hcc, \hcd) \getsr
               \hcsim(1^\secpar, \hccir, \hccir(\hcmsg))
               \right\}}_{\secpar,\hcmsg,\hccir}\enspace.
  \end{align*}
  For that, let us introduce the hybrid distribution $\hybrid$, where $(\hcc,\hcd)$ is generated as in $\distr_0$ except for $\otflowone_{i,b,j}$ for $i \in [S]$, $b = 1-\hccir[i]$, and $j \in [|\gckey[i,b]|]$ that is generated as follows:
  \[ \otflowone_{i,b,j} = \otsendone(1^\secpar,0;\otrandone_{i,b,j})\enspace.  \]

  As $\otrandone_{i,b,j}$ is never revealed, we have the following claim.
  \begin{claim}
    If $\ot$ is receiver-private, then $\distr_0$ and $\hybrid$ are computationally indistinguishable.
  \end{claim}

  We also remark that in $\hybrid$, the input labels $\smallset{\gckey[i,1-\hccir[i]]}$ are never used: only the input labels $\smallset{\gckey'[i]} = \smallset{\gckey[i,\hccir[i]]}$ are used. The only difference between $\hybrid$ and $\distr_1$ is that in $\hybrid$, $(\gckey',\gccir)$ is generated honestly using $\gcgarble$, while in $\distr_1$, it is simulated by $\gcsim$.
  Thus we have the following claim.

  \begin{claim}
    If $\gc$ is simulatable, then $\hybrid$ and $\distr_1$ are computationally indistinguishable.
  \end{claim}

  \subparagraph{Semantic Security of the Witness Selector:}  We need to prove the computational indistinguishability of the following two distributions:
  \begin{align*}
    \distr_0 &= \bigset{
    ((\hcc,\hccir), (\hcout,\hcd), \hcr, \wsct)
    \ : \
               \begin{array}{l}
                 \hcr \getsr \bits^\hcrlen;
                 \hcc = \hccom(1^\secpar,\hcmsg;\hcr); \\
                 \hccirout = \hccir(\hcmsg); \hcd = \hcopen(\hcc,\hccir,\hccirout,\hcr); \\
                 \wsct \getsr \wsenc(1^\secpar,(\hcc,\hccir),\wsmsg)
               \end{array}
    }_{\secpar,\hccir,\hcmsg,\wsmsg} \enspace, \\
    \distr_1 &= \bigset{
    ((\hcc,\hccir), (\hcout,\hcd), \hcr, \wsct)
    \ : \
               \begin{array}{l}
                 \hcr \getsr \bits^\hcrlen; 
                 \hcc = \hccom(1^\secpar,\hcmsg;\hcr); \\
                 \hccirout = \hccir(\hcmsg); \hcd = \hcopen(\hcc,\hccir,\hcout,\hcr); \\
                 \smallset{\wsmsg'[I,B] = \wsmsg[I, \hccirout_I]}_{I,B} \\
                 \wsct \getsr \wsenc(1^\secpar,(\hcc,\hccir),\wsmsg')
               \end{array}
    }_{\secpar,\hccir,\hcmsg,\wsmsg} \enspace. \\
  \end{align*}

  For that, we consider the following hybrid distributions:
  \begin{description}
  \item[Hybrid $\hybrid_1$:] This hybrid is similar to $\distr_0$, except that the second flows of the OT protocol $\otflowtwo_{i,j}$ (generated by $\wsenc$) are now generated as follows: for $i \in [S]$ and $j \in [|\gckey[1,0]|]$:
    \[ \otflowtwo_{i,j} \getsr \otsendtwo(\otflowone_{i,\hccir[i],j},\otmsg_{i,j,\gckey[i,\hccir[i]]_j},\otmsg_{i,j,\gckey[i,\hccir[i]]_j}) \enspace. \]
    As the first flow $\otflowone_{i,\hccir[i],j}$ is generated as $\otflowone_{i,\hccir[i],j} = \otsendone(1^\secpar,\gckey[i,\hccir[i]]_j; \otrandone_{i,b,j}$, we have the following claim.

    \begin{claim}
      If $\ot$ is sender-private, then $\distr_0$ and $\hybrid_1$ are computationally indistinguishable.
    \end{claim}

    We recall that $\otmsg_{i,j,b} = \smallset{\tgckey_{I,B}[i,j,b]}_{I,B}$.
    We remark that in this hybrid, the input labels $\tgckey_{I,B}[i,j,1-\gckey[i,\hccir[i]]_j]$ are not used. Let us write $\tgckey'_{I,B}[i,j] = \tgckey_{I,B}[i,j,\gckey[i,\hccir[i]]_j]$
  \item[Hybrid $\hybrid_2$:] This hybrid is similar to $\hybrid_1$ except that the garbled circuits $\tgccir_{I,B}$ and its input labels $\tgckey'_{I,B}[i,j]$ are simulated: for every $I \in [\hcciroutlen]$ and $B \in \bits$:
    \[ (\tgckey'_{I,B}, \tgcsimcir_{I,B}) \getsr \tgcsim(1^\secpar,\tilde{y}_{I,B})\enspace, \]
    and $\tgccir_{I,B}$ is replaced by $\tgcsimcir_{I,B}$,
    where:
    \begin{align*}
    \tilde{y}_{I,B} &= \tcir_{I,B}(\smallset{\gckey[i,\hccir[i]]}_{i \in [S]}) \\
      &=
      \begin{cases}
        \wsmsg[I,B] &\text{if } \hcout'_I = B \text{ where } \hcout' = \gceval(\gccir,\smallset{\gckey[i,\hccir[i]]}_{i \in [S]}), \\
        \perp &\text{otherwise.}
      \end{cases}
    \end{align*}
    
    We have the following straightforward claim.
    \begin{claim}
      If $\tgc$ is simulatable, then $\hybrid_1$ and $\hybrid_2$ are computationally indistinguishable.
    \end{claim}

    We recall that by definition of $\gccir$ and by correctness of garbling:
    \[ \gceval(\gccir,\smallset{\gckey[i,\hccir[i]]}_{i \in [S]}) = \cir(\hccir) = \univcirc_\secpar(\hccir,\hcmsg) = \hccir(\hcmsg) = \hcout \enspace. \]
    In other words:
    \[ \tilde{y}_{I,B}       =
      \begin{cases}
        \wsmsg[I,B] &\text{if } \hcout_I = B \\
        \perp &\text{otherwise.}
      \end{cases}
    \]
    This hybrid distribution thus only depends on $\smallset{\wsmsg[I,\hcout_I]}_I$.
  \item[Hybrid $\hybrid_3$:] This hybrid is defined with regards to $\distr_1$ exactly as $\hybrid_1$ is defined with regards to $\distr_0$.

    We have the two following immediate claims.
    \begin{claim}
      If $\tgc$ is simulatable, then $\hybrid_2$ and $\hybrid_3$ are computationally indistinguishable.
    \end{claim}

    \begin{claim}
      If $\ot$ is sender-private, then $\hybrid_3$ and $\distr_1$ are computationally indistinguishable.
    \end{claim}
  \end{description}
\end{proof}


%%% Local Variables:
%%% mode: latex
%%% TeX-master: "../main"
%%% End:

}
}{}


\end{document}




%%% Local Variables: 
%%% mode: latex
%%% TeX-master: t
%%% End: 
